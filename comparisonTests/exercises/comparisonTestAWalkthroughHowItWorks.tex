\documentclass{ximera}

%\usepackage{todonotes}
%\usepackage{mathtools} %% Required for wide table Curl and Greens
%\usepackage{cuted} %% Required for wide table Curl and Greens
\newcommand{\todo}{}

\usepackage{multicol}

\usepackage{esint} % for \oiint
\ifxake%%https://math.meta.stackexchange.com/questions/9973/how-do-you-render-a-closed-surface-double-integral
\renewcommand{\oiint}{{\large\bigcirc}\kern-1.56em\iint}
\fi

\graphicspath{
  {./}
  {ximeraTutorial/}
  {basicPhilosophy/}
  {functionsOfSeveralVariables/}
  {normalVectors/}
  {lagrangeMultipliers/}
  {vectorFields/}
  {greensTheorem/}
  {shapeOfThingsToCome/}
  {dotProducts/}
  {partialDerivativesAndTheGradientVector/}
  {../ximeraTutorial/}
  {../productAndQuotientRules/exercises/}
  {../motionAndPathsInSpace/exercises/}
  {../normalVectors/exercisesParametricPlots/}
  {../continuityOfFunctionsOfSeveralVariables/exercises/}
  {../partialDerivativesAndTheGradientVector/exercises/}
  {../directionalDerivativeAndChainRule/exercises/}
  {../commonCoordinates/exercisesCylindricalCoordinates/}
  {../commonCoordinates/exercisesSphericalCoordinates/}
  {../greensTheorem/exercisesCurlAndLineIntegrals/}
  {../greensTheorem/exercisesDivergenceAndLineIntegrals/}
  {../shapeOfThingsToCome/exercisesDivergenceTheorem/}
  {../greensTheorem/}
  {../shapeOfThingsToCome/}
  {../separableDifferentialEquations/exercises/}
  {../dotproducts/}
  {../functionsOfSeveralVariables/}
  {../lagrangeMultipliers/}
  {../partialDerivativesAndTheGradientVector/}
  {../normalVectors/}
  {../vectorFields/}
}
\def\xmNotExpandableAsAccordion{true}

\newcommand{\mooculus}{\textsf{\textbf{MOOC}\textnormal{\textsf{ULUS}}}}

\usepackage{tkz-euclide}\usepackage{tikz}
\usepackage{tikz-cd}
\usetikzlibrary{arrows}
\tikzset{>=stealth,commutative diagrams/.cd,
  arrow style=tikz,diagrams={>=stealth}} %% cool arrow head
\tikzset{shorten <>/.style={ shorten >=#1, shorten <=#1 } } %% allows shorter vectors

\usetikzlibrary{backgrounds} %% for boxes around graphs
\usetikzlibrary{shapes,positioning}  %% Clouds and stars
\usetikzlibrary{matrix} %% for matrix
\usepgfplotslibrary{polar} %% for polar plots
\usepgfplotslibrary{fillbetween} %% to shade area between curves in TikZ
%\usetkzobj{all} %% obsolete

\usepackage[makeroom]{cancel} %% for strike outs
%\usepackage{mathtools} %% for pretty underbrace % Breaks Ximera
%\usepackage{multicol}
\usepackage{pgffor} %% required for integral for loops

\usepackage{tkz-tab}  %% for sign charts

%% http://tex.stackexchange.com/questions/66490/drawing-a-tikz-arc-specifying-the-center
%% Draws beach ball
\tikzset{pics/carc/.style args={#1:#2:#3}{code={\draw[pic actions] (#1:#3) arc(#1:#2:#3);}}}



\usepackage{array}
\setlength{\extrarowheight}{+.1cm}
\newdimen\digitwidth
\settowidth\digitwidth{9}
\def\divrule#1#2{
\noalign{\moveright#1\digitwidth
\vbox{\hrule width#2\digitwidth}}}





\newcommand{\RR}{\mathbb R}
\newcommand{\R}{\mathbb R}
\newcommand{\N}{\mathbb N}
\newcommand{\Z}{\mathbb Z}

\newcommand{\sagemath}{\textsf{SageMath}}


\renewcommand{\d}{\,d}
%\def\d{\mathop{}\!d}
%\def\d{\,d}

\AddToHook{begindocument}{%
  \renewcommand{\d}{\,d}     % lualatex redefines \d to underdot !!!
}

\pgfplotsset{
    every axis/.style={
        scale only axis,
        enlargelimits=false,
        trim axis left,
        trim axis right,
        clip=true,
    }
}

\newcommand{\dd}[2][]{\frac{\d #1}{\d #2}}
\newcommand{\pp}[2][]{\frac{\partial #1}{\partial #2}}
\renewcommand{\l}{\ell}
\newcommand{\ddx}{\frac{d}{\d x}}

\newcommand{\zeroOverZero}{\ensuremath{\boldsymbol{\tfrac{0}{0}}}}
\newcommand{\inftyOverInfty}{\ensuremath{\boldsymbol{\tfrac{\infty}{\infty}}}}
\newcommand{\zeroOverInfty}{\ensuremath{\boldsymbol{\tfrac{0}{\infty}}}}
\newcommand{\zeroTimesInfty}{\ensuremath{\small\boldsymbol{0\cdot \infty}}}
\newcommand{\inftyMinusInfty}{\ensuremath{\small\boldsymbol{\infty - \infty}}}
\newcommand{\oneToInfty}{\ensuremath{\boldsymbol{1^\infty}}}
\newcommand{\zeroToZero}{\ensuremath{\boldsymbol{0^0}}}
\newcommand{\inftyToZero}{\ensuremath{\boldsymbol{\infty^0}}}



\newcommand{\numOverZero}{\ensuremath{\boldsymbol{\tfrac{\#}{0}}}}
\newcommand{\dfn}{\textbf}
%\newcommand{\unit}{\,\mathrm}
\newcommand{\unit}{\mathop{}\!\mathrm}
\newcommand{\eval}[1]{\bigg[ #1 \bigg]}
\newcommand{\seq}[1]{\left( #1 \right)}
\renewcommand{\epsilon}{\varepsilon}
\renewcommand{\phi}{\varphi}


\renewcommand{\iff}{\Leftrightarrow}

\DeclareMathOperator{\arccot}{arccot}
\DeclareMathOperator{\arcsec}{arcsec}
\DeclareMathOperator{\arccsc}{arccsc}
\DeclareMathOperator{\si}{Si}
\DeclareMathOperator{\scal}{scal}
\DeclareMathOperator{\sign}{sign}


%% \newcommand{\tightoverset}[2]{% for arrow vec
%%   \mathop{#2}\limits^{\vbox to -.5ex{\kern-0.75ex\hbox{$#1$}\vss}}}
\newcommand{\arrowvec}[1]{{\overset{\rightharpoonup}{#1}}}
%\renewcommand{\vec}[1]{\arrowvec{\mathbf{#1}}}
\renewcommand{\vec}[1]{{\overset{\boldsymbol{\rightharpoonup}}{\mathbf{#1}}}\hspace{0in}}

\newcommand{\point}[1]{\left(#1\right)} %this allows \vector{ to be changed to \vector{ with a quick find and replace
\newcommand{\pt}[1]{\mathbf{#1}} %this allows \vec{ to be changed to \vec{ with a quick find and replace
\newcommand{\Lim}[2]{\lim_{\point{#1} \to \point{#2}}} %Bart, I changed this to point since I want to use it.  It runs through both of the exercise and exerciseE files in limits section, which is why it was in each document to start with.

\DeclareMathOperator{\proj}{\mathbf{proj}}
\newcommand{\veci}{{\boldsymbol{\hat{\imath}}}}
\newcommand{\vecj}{{\boldsymbol{\hat{\jmath}}}}
\newcommand{\veck}{{\boldsymbol{\hat{k}}}}
\newcommand{\vecl}{\vec{\boldsymbol{\l}}}
\newcommand{\uvec}[1]{\mathbf{\hat{#1}}}
\newcommand{\utan}{\mathbf{\hat{t}}}
\newcommand{\unormal}{\mathbf{\hat{n}}}
\newcommand{\ubinormal}{\mathbf{\hat{b}}}

\newcommand{\dotp}{\bullet}
\newcommand{\cross}{\boldsymbol\times}
\newcommand{\grad}{\boldsymbol\nabla}
\newcommand{\divergence}{\grad\dotp}
\newcommand{\curl}{\grad\cross}
%\DeclareMathOperator{\divergence}{divergence}
%\DeclareMathOperator{\curl}[1]{\grad\cross #1}
\newcommand{\lto}{\mathop{\longrightarrow\,}\limits}

\renewcommand{\bar}{\overline}

\colorlet{textColor}{black}
\colorlet{background}{white}
\colorlet{penColor}{blue!50!black} % Color of a curve in a plot
\colorlet{penColor2}{red!50!black}% Color of a curve in a plot
\colorlet{penColor3}{red!50!blue} % Color of a curve in a plot
\colorlet{penColor4}{green!50!black} % Color of a curve in a plot
\colorlet{penColor5}{orange!80!black} % Color of a curve in a plot
\colorlet{penColor6}{yellow!70!black} % Color of a curve in a plot
\colorlet{fill1}{penColor!20} % Color of fill in a plot
\colorlet{fill2}{penColor2!20} % Color of fill in a plot
\colorlet{fillp}{fill1} % Color of positive area
\colorlet{filln}{penColor2!20} % Color of negative area
\colorlet{fill3}{penColor3!20} % Fill
\colorlet{fill4}{penColor4!20} % Fill
\colorlet{fill5}{penColor5!20} % Fill
\colorlet{gridColor}{gray!50} % Color of grid in a plot

\newcommand{\surfaceColor}{violet}
\newcommand{\surfaceColorTwo}{redyellow}
\newcommand{\sliceColor}{greenyellow}




\pgfmathdeclarefunction{gauss}{2}{% gives gaussian
  \pgfmathparse{1/(#2*sqrt(2*pi))*exp(-((x-#1)^2)/(2*#2^2))}%
}


%%%%%%%%%%%%%
%% Vectors
%%%%%%%%%%%%%

%% Simple horiz vectors
\renewcommand{\vector}[1]{\left\langle #1\right\rangle}


%% %% Complex Horiz Vectors with angle brackets
%% \makeatletter
%% \renewcommand{\vector}[2][ , ]{\left\langle%
%%   \def\nextitem{\def\nextitem{#1}}%
%%   \@for \el:=#2\do{\nextitem\el}\right\rangle%
%% }
%% \makeatother

%% %% Vertical Vectors
%% \def\vector#1{\begin{bmatrix}\vecListA#1,,\end{bmatrix}}
%% \def\vecListA#1,{\if,#1,\else #1\cr \expandafter \vecListA \fi}

%%%%%%%%%%%%%
%% End of vectors
%%%%%%%%%%%%%

%\newcommand{\fullwidth}{}
%\newcommand{\normalwidth}{}



%% makes a snazzy t-chart for evaluating functions
%\newenvironment{tchart}{\rowcolors{2}{}{background!90!textColor}\array}{\endarray}

%%This is to help with formatting on future title pages.
\newenvironment{sectionOutcomes}{}{}



%% Flowchart stuff
%\tikzstyle{startstop} = [rectangle, rounded corners, minimum width=3cm, minimum height=1cm,text centered, draw=black]
%\tikzstyle{question} = [rectangle, minimum width=3cm, minimum height=1cm, text centered, draw=black]
%\tikzstyle{decision} = [trapezium, trapezium left angle=70, trapezium right angle=110, minimum width=3cm, minimum height=1cm, text centered, draw=black]
%\tikzstyle{question} = [rectangle, rounded corners, minimum width=3cm, minimum height=1cm,text centered, draw=black]
%\tikzstyle{process} = [rectangle, minimum width=3cm, minimum height=1cm, text centered, draw=black]
%\tikzstyle{decision} = [trapezium, trapezium left angle=70, trapezium right angle=110, minimum width=3cm, minimum height=1cm, text centered, draw=black]


\author{Jim Talamo}
\license{Creative Commons 3.0 By-bC}


\outcome{}


\begin{document}
\begin{exercise}

Consider the series $\sum_{k=1}^{\infty} \frac{k}{k^3+2k+1}$ and pretend that you know only the definition of convergence for series and none of the convergence tests.  

If we were only concerned about the terms in the \emph{sequence}, $\frac{k}{k^3+2k+1}$, we are trying to sum, we could use growth rates to determine that the dominant term in the numerator is $k$ and that the dominant term in the denominator is

\begin{multipleChoice}
\choice{$1$}
\choice{$2k$}
\choice[correct]{$k^3$}
\end{multipleChoice}

Thus, the rational expression $\frac{k}{k^3+2k+1}$ should behave like $\frac{k}{k^3} = \frac{1}{k^2}$ for large $k$.  

While this lets us determine exactly what the limit of the \emph{sequence} $\frac{k}{k^3+2k+1}$ would be, we have to remember that for the \emph{series} $\sum_{k=1}^{\infty} \frac{k}{k^3+2k+1}$:

\begin{multipleChoice}
\choice[correct]{we are adding infinitely many terms together, so the small discrepancies between $\frac{k}{k^3+2k+1}$ and $\frac{1}{k^2}$ may be important!}
\choice{since the difference between $\frac{k}{k^3+2k+1}$ and $\frac{1}{k^2}$ approaches $0$ as $k$ increases, no further exploration is necessary.}
\end{multipleChoice}

Indeed, note that while it is true that $\frac{k}{k^3+2k+1} $ and $\frac{1}{k^2}$ become arbitrarily close as $n$ grows arbitrarily large, there is a small difference between these for each $n$, and the fact that we are trying to add infinitely many such terms together may be problematic!  Indeed, we think of this as:

\[
\frac{k}{k^3+2k+1} \to \frac{1}{k^2} \quad \textrm{ so } \quad \frac{k}{k^3+2k+1} - \frac{1}{k^2} \to 0  
\]

Since we have an infinite series, the consequence of adding infinitely many such terms tells us that heuristically we are adding infinitely many terms whose difference approaches $0$, or that:

\[
\sum^{\infty} \frac{k}{k^3+2k+1} - \frac{1}{k^2} \to \infty \cdot \answer{0}  
\]

There is no reason \emph{a priori} to conclude that we can gain any insight into a \emph{series} by only studying the sequence whose terms we are adding!  

\begin{exercise}
Indeed, the terms in the sequences $\{a_n\}$ given by $a_n =  \frac{1}{2^n}$ and $\{b_n \}$ given by: $b_n=\frac{1}{2^n} + \frac{1}{n}$ are sequences whose terms become arbitrarily close together as $n$ increases since:

\[
a_n-b_n =  \frac{1}{2^n} - \left(\frac{1}{2^n} + \frac{1}{n}\right) = \answer{-\frac{1}{n}}
\]
and thus the difference tends to $0$.  However:

\begin{multipleChoice}
\choice[correct]{$\sum_{k=1}^n a_k$ is a geometric series with $|r| <1$.  It converges.}
\choice{$\sum_{k=1}^n a_k$ is a geometric series with $|r| >1$.  It diverges.}
\end{multipleChoice}

If $\sum_{k=1}^{\infty} b_k$ converges, then $\sum_{k=1}^{\infty} a_k-b_k$ would converge.  However, $\sum_{k=1}^{\infty} a_k-b_k = \sum_{k=1}^{\infty} -\frac{1}{k}$, which: 

\begin{multipleChoice}
\choice[correct]{diverges; it's a $p$-series with $p =1$.}
\choice{converges; it's a $p$-series with $p=1$.}
\end{multipleChoice}

Hence, $\sum_{k=1}^{\infty} b_k$ diverges.  

What conclusion should we draw?

\begin{multipleChoice}
\choice[correct]{We cannot use the fact that the terms in the \emph{sequences} $\{a_n\}$ and $\{b_n\}$ become arbitrarily close to determine if the \emph{series} $\sum_{k=1}^{\infty} a_k$ and $\sum_{k=1}^{\infty} b_k$ either both converge or both diverge.}
\choice{If two sequences $\{a_n\}$ and $\{b_n\}$ have terms that become arbitrarily close, then $\sum_{k=1}^{\infty} a_k$ and $\sum_{k=1}^{\infty} b_k$ will either both converge or both diverge.}
\end{multipleChoice}

\end{exercise}
We need to establish a sufficient way to compare the behavior of the \emph{series} $\sum_{k=1}^{\infty} \frac{k}{k^3+2k+1}$ and $\sum_{k=1}^{\infty} \frac{1}{k^2}$ by comparing the terms in the  \emph{sequences} $\frac{k}{k^3+2k+1}$ and $\frac{1}{k^2}$.  The comparison tests precisely establish this, and the logic behind the comparison test is explored in the context of a specific example in this exercise.

\begin{exercise}
To begin, recall that the series $\sum_{k=1}^{\infty}  a_k$ converges if:

\begin{multipleChoice}
\choice{$\lim_{n \to \infty} a_n = 0$.}
\choice{$\lim_{n \to \infty} a_n$ exists.}
\choice[correct]{$\lim_{n \to \infty} s_n$ exists, where $s_n = \sum_{k=1}^n a_k$.}
\end{multipleChoice}

There are two strategies for determining if $\lim_{n \to \infty} s_n$ exists:

\begin{itemize}
\item[1.] Find an explicit formula for $s_n$ and use it to compute $\lim_{n \to \infty} s_n$ or determine that it does not exist.
\item[2.] Determine if $\{s_n\}$ is bounded and/or monotonic.   
\end{itemize}

In the present case, it would be very difficult to establish an explicit formula for $s_n$, so we will try to establish existence using the second strategy.  First, recall the previous important ideas regarding the second strategy:

\begin{selectAll}
\choice{If $\{s_n\}$ is monotonic $\lim_{n \to \infty} s_n$ exists.}
\choice{If $\{s_n\}$ is bounded $\lim_{n \to \infty} s_n$ exists.}
\choice[correct]{If $\{s_n\}$ is bounded and monotonic, $\lim_{n \to \infty} s_n$ exists.}
\choice[correct]{If $\{s_n\}$ is increasing, we can show $\lim_{n \to \infty} s_n$ does not exist by showing it is not bounded above.}
\choice{If $\{s_n\}$ is increasing, we can show $\lim_{n \to \infty} s_n$ does not exist by showing it is not bounded below.}
\end{selectAll}

Since we have reasoned before that the \emph{sequences} $\frac{k}{k^3+2k+1}$ and $\frac{1}{k^2}$ should be asymptotically similar (behave the same way as $k$ grows arbitrarily large, it is reasonable to hope that:

\begin{multipleChoice}
\choice[correct]{$\sum_{k=1}^{\infty} \frac{k}{k^3+2k+1}$ converges.}
\choice{$\sum_{k=1}^{\infty} \frac{k}{k^3+2k+1} = \sum_{k=1}^{\infty} \frac{1}{k^2}$.}
\choice{$\sum_{k=1}^{\infty} \frac{k}{k^3+2k+1}$ diverges.}
\end{multipleChoice}

Thus, of the strategies we have for considering the limit of $\{s_n\}$ when we cannot find an explicit formula, we will try to establish:

\begin{selectAll}
\choice[correct]{$\{s_n\}$ is bounded and monotonic, so $\lim_{n \to \infty} s_n$ exists.}
\choice{$\{s_n\}$ is increasing, so we can show $\lim_{n \to \infty} s_n$ does not exist by showing it is not bounded above.}
\end{selectAll}

Since $a_n = \frac{n}{n^3+2n+1} > 0$ for all $n \geq 1$ and $s_n = \sum_{k=1}^n a_k$,  we see that:

\begin{multipleChoice}
\choice[correct]{$\{s_n\}$ is increasing.}
\choice{$\{s_n\}$ is decreasing.}
\choice{$\{s_n\}$ is neither increasing nor decreasing.}
\end{multipleChoice}

Also, note that $n^3 < n^3+2n+1$ for all $n \geq 1$, so:
\begin{multipleChoice}
\choice{$\frac{1}{n^3} < \frac{1}{n^3+2n+1}$}
\choice[correct]{$\frac{1}{n^3} > \frac{1}{n^3+2n+1}$}
\end{multipleChoice}

Thus, multiplying both sides by $n$ gives:

\[
\frac{n}{n^3} > \frac{n}{n^3+2n+1}
\]

Note that $\frac{n}{n^3} = \frac{1}{n^2}$, and since this holds for every value of $n$, it certainly holds for the sums up to this value:

\[
\sum_{k=1}^n \frac{k}{k^3+2k+1} < \sum_{k=1}^n \frac{1}{k^2}
\]

Since $\frac{1}{k^2}$ is positive for every $k$, we know that $\sum_{k=1}^n \frac{1}{k^2} < \sum_{k=1}^{\infty} \frac{1}{k^2}$.  But, $\sum_{k=1}^{\infty} \frac{1}{k^2}$:

\begin{multipleChoice}
\choice[correct]{is a $p$-series with $p>1$, it converges.}
\choice{is a $p$-series with $p>1$, it diverges.}
\choice{is a $p$-series with $p<1$, it converges.}
\choice{is a $p$-series with $p<1$, it diverges.}
\end{multipleChoice}

Thus, since:

\[
s_n = \sum_{k=1}^n \frac{n}{n^3+2n+1} < \sum_{k=1}^{\infty} \frac{1}{k^2}
\]

we see that $\{s_n\}$ is bounded above!

We have thus established that $\{s_n\}$ is bounded and monotonic, so $\lim_{n \to \infty} s_n$ must exist, and hence  $\sum_{k=1}^{\infty} \frac{k}{k^3+2k+1}$ converges.

\begin{exercise}
This logic is precisely the logic that can be used to give a formal proof of the comparison test.  Indeed, if we require all of the $a_k$ terms we are summing to be positive, we can establish $s_n$ is increasing:

\begin{itemize}
\item  If we expect the series to converge, we then are tasked with finding a sequence whose terms are larger that converges (like we did here) to establish an upper bound.  
\item If we expect the series to diverge, we must find a sequence whose terms are smaller that diverges.
\end{itemize}

While the sequence of partial sums does not explicitly appear in the statement of the test, it is nonetheless behind the scenes!  
\end{exercise}
\end{exercise}
\end{exercise}
\end{document}
