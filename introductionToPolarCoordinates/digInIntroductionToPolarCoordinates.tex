\documentclass{ximera}

%\usepackage{todonotes}
%\usepackage{mathtools} %% Required for wide table Curl and Greens
%\usepackage{cuted} %% Required for wide table Curl and Greens
\newcommand{\todo}{}

\usepackage{multicol}

\usepackage{esint} % for \oiint
\ifxake%%https://math.meta.stackexchange.com/questions/9973/how-do-you-render-a-closed-surface-double-integral
\renewcommand{\oiint}{{\large\bigcirc}\kern-1.56em\iint}
\fi

\graphicspath{
  {./}
  {ximeraTutorial/}
  {basicPhilosophy/}
  {functionsOfSeveralVariables/}
  {normalVectors/}
  {lagrangeMultipliers/}
  {vectorFields/}
  {greensTheorem/}
  {shapeOfThingsToCome/}
  {dotProducts/}
  {partialDerivativesAndTheGradientVector/}
  {../ximeraTutorial/}
  {../productAndQuotientRules/exercises/}
  {../motionAndPathsInSpace/exercises/}
  {../normalVectors/exercisesParametricPlots/}
  {../continuityOfFunctionsOfSeveralVariables/exercises/}
  {../partialDerivativesAndTheGradientVector/exercises/}
  {../directionalDerivativeAndChainRule/exercises/}
  {../commonCoordinates/exercisesCylindricalCoordinates/}
  {../commonCoordinates/exercisesSphericalCoordinates/}
  {../greensTheorem/exercisesCurlAndLineIntegrals/}
  {../greensTheorem/exercisesDivergenceAndLineIntegrals/}
  {../shapeOfThingsToCome/exercisesDivergenceTheorem/}
  {../greensTheorem/}
  {../shapeOfThingsToCome/}
  {../separableDifferentialEquations/exercises/}
  {../dotproducts/}
  {../functionsOfSeveralVariables/}
  {../lagrangeMultipliers/}
  {../partialDerivativesAndTheGradientVector/}
  {../normalVectors/}
  {../vectorFields/}
}
\def\xmNotExpandableAsAccordion{true}

\newcommand{\mooculus}{\textsf{\textbf{MOOC}\textnormal{\textsf{ULUS}}}}

\usepackage{tkz-euclide}\usepackage{tikz}
\usepackage{tikz-cd}
\usetikzlibrary{arrows}
\tikzset{>=stealth,commutative diagrams/.cd,
  arrow style=tikz,diagrams={>=stealth}} %% cool arrow head
\tikzset{shorten <>/.style={ shorten >=#1, shorten <=#1 } } %% allows shorter vectors

\usetikzlibrary{backgrounds} %% for boxes around graphs
\usetikzlibrary{shapes,positioning}  %% Clouds and stars
\usetikzlibrary{matrix} %% for matrix
\usepgfplotslibrary{polar} %% for polar plots
\usepgfplotslibrary{fillbetween} %% to shade area between curves in TikZ
%\usetkzobj{all} %% obsolete

\usepackage[makeroom]{cancel} %% for strike outs
%\usepackage{mathtools} %% for pretty underbrace % Breaks Ximera
%\usepackage{multicol}
\usepackage{pgffor} %% required for integral for loops

\usepackage{tkz-tab}  %% for sign charts

%% http://tex.stackexchange.com/questions/66490/drawing-a-tikz-arc-specifying-the-center
%% Draws beach ball
\tikzset{pics/carc/.style args={#1:#2:#3}{code={\draw[pic actions] (#1:#3) arc(#1:#2:#3);}}}



\usepackage{array}
\setlength{\extrarowheight}{+.1cm}
\newdimen\digitwidth
\settowidth\digitwidth{9}
\def\divrule#1#2{
\noalign{\moveright#1\digitwidth
\vbox{\hrule width#2\digitwidth}}}





\newcommand{\RR}{\mathbb R}
\newcommand{\R}{\mathbb R}
\newcommand{\N}{\mathbb N}
\newcommand{\Z}{\mathbb Z}

\newcommand{\sagemath}{\textsf{SageMath}}


\renewcommand{\d}{\,d}
%\def\d{\mathop{}\!d}
%\def\d{\,d}

\AddToHook{begindocument}{%
  \renewcommand{\d}{\,d}     % lualatex redefines \d to underdot !!!
}

\pgfplotsset{
    every axis/.style={
        scale only axis,
        enlargelimits=false,
        trim axis left,
        trim axis right,
        clip=true,
    }
}

\newcommand{\dd}[2][]{\frac{\d #1}{\d #2}}
\newcommand{\pp}[2][]{\frac{\partial #1}{\partial #2}}
\renewcommand{\l}{\ell}
\newcommand{\ddx}{\frac{d}{\d x}}

\newcommand{\zeroOverZero}{\ensuremath{\boldsymbol{\tfrac{0}{0}}}}
\newcommand{\inftyOverInfty}{\ensuremath{\boldsymbol{\tfrac{\infty}{\infty}}}}
\newcommand{\zeroOverInfty}{\ensuremath{\boldsymbol{\tfrac{0}{\infty}}}}
\newcommand{\zeroTimesInfty}{\ensuremath{\small\boldsymbol{0\cdot \infty}}}
\newcommand{\inftyMinusInfty}{\ensuremath{\small\boldsymbol{\infty - \infty}}}
\newcommand{\oneToInfty}{\ensuremath{\boldsymbol{1^\infty}}}
\newcommand{\zeroToZero}{\ensuremath{\boldsymbol{0^0}}}
\newcommand{\inftyToZero}{\ensuremath{\boldsymbol{\infty^0}}}



\newcommand{\numOverZero}{\ensuremath{\boldsymbol{\tfrac{\#}{0}}}}
\newcommand{\dfn}{\textbf}
%\newcommand{\unit}{\,\mathrm}
\newcommand{\unit}{\mathop{}\!\mathrm}
\newcommand{\eval}[1]{\bigg[ #1 \bigg]}
\newcommand{\seq}[1]{\left( #1 \right)}
\renewcommand{\epsilon}{\varepsilon}
\renewcommand{\phi}{\varphi}


\renewcommand{\iff}{\Leftrightarrow}

\DeclareMathOperator{\arccot}{arccot}
\DeclareMathOperator{\arcsec}{arcsec}
\DeclareMathOperator{\arccsc}{arccsc}
\DeclareMathOperator{\si}{Si}
\DeclareMathOperator{\scal}{scal}
\DeclareMathOperator{\sign}{sign}


%% \newcommand{\tightoverset}[2]{% for arrow vec
%%   \mathop{#2}\limits^{\vbox to -.5ex{\kern-0.75ex\hbox{$#1$}\vss}}}
\newcommand{\arrowvec}[1]{{\overset{\rightharpoonup}{#1}}}
%\renewcommand{\vec}[1]{\arrowvec{\mathbf{#1}}}
\renewcommand{\vec}[1]{{\overset{\boldsymbol{\rightharpoonup}}{\mathbf{#1}}}\hspace{0in}}

\newcommand{\point}[1]{\left(#1\right)} %this allows \vector{ to be changed to \vector{ with a quick find and replace
\newcommand{\pt}[1]{\mathbf{#1}} %this allows \vec{ to be changed to \vec{ with a quick find and replace
\newcommand{\Lim}[2]{\lim_{\point{#1} \to \point{#2}}} %Bart, I changed this to point since I want to use it.  It runs through both of the exercise and exerciseE files in limits section, which is why it was in each document to start with.

\DeclareMathOperator{\proj}{\mathbf{proj}}
\newcommand{\veci}{{\boldsymbol{\hat{\imath}}}}
\newcommand{\vecj}{{\boldsymbol{\hat{\jmath}}}}
\newcommand{\veck}{{\boldsymbol{\hat{k}}}}
\newcommand{\vecl}{\vec{\boldsymbol{\l}}}
\newcommand{\uvec}[1]{\mathbf{\hat{#1}}}
\newcommand{\utan}{\mathbf{\hat{t}}}
\newcommand{\unormal}{\mathbf{\hat{n}}}
\newcommand{\ubinormal}{\mathbf{\hat{b}}}

\newcommand{\dotp}{\bullet}
\newcommand{\cross}{\boldsymbol\times}
\newcommand{\grad}{\boldsymbol\nabla}
\newcommand{\divergence}{\grad\dotp}
\newcommand{\curl}{\grad\cross}
%\DeclareMathOperator{\divergence}{divergence}
%\DeclareMathOperator{\curl}[1]{\grad\cross #1}
\newcommand{\lto}{\mathop{\longrightarrow\,}\limits}

\renewcommand{\bar}{\overline}

\colorlet{textColor}{black}
\colorlet{background}{white}
\colorlet{penColor}{blue!50!black} % Color of a curve in a plot
\colorlet{penColor2}{red!50!black}% Color of a curve in a plot
\colorlet{penColor3}{red!50!blue} % Color of a curve in a plot
\colorlet{penColor4}{green!50!black} % Color of a curve in a plot
\colorlet{penColor5}{orange!80!black} % Color of a curve in a plot
\colorlet{penColor6}{yellow!70!black} % Color of a curve in a plot
\colorlet{fill1}{penColor!20} % Color of fill in a plot
\colorlet{fill2}{penColor2!20} % Color of fill in a plot
\colorlet{fillp}{fill1} % Color of positive area
\colorlet{filln}{penColor2!20} % Color of negative area
\colorlet{fill3}{penColor3!20} % Fill
\colorlet{fill4}{penColor4!20} % Fill
\colorlet{fill5}{penColor5!20} % Fill
\colorlet{gridColor}{gray!50} % Color of grid in a plot

\newcommand{\surfaceColor}{violet}
\newcommand{\surfaceColorTwo}{redyellow}
\newcommand{\sliceColor}{greenyellow}




\pgfmathdeclarefunction{gauss}{2}{% gives gaussian
  \pgfmathparse{1/(#2*sqrt(2*pi))*exp(-((x-#1)^2)/(2*#2^2))}%
}


%%%%%%%%%%%%%
%% Vectors
%%%%%%%%%%%%%

%% Simple horiz vectors
\renewcommand{\vector}[1]{\left\langle #1\right\rangle}


%% %% Complex Horiz Vectors with angle brackets
%% \makeatletter
%% \renewcommand{\vector}[2][ , ]{\left\langle%
%%   \def\nextitem{\def\nextitem{#1}}%
%%   \@for \el:=#2\do{\nextitem\el}\right\rangle%
%% }
%% \makeatother

%% %% Vertical Vectors
%% \def\vector#1{\begin{bmatrix}\vecListA#1,,\end{bmatrix}}
%% \def\vecListA#1,{\if,#1,\else #1\cr \expandafter \vecListA \fi}

%%%%%%%%%%%%%
%% End of vectors
%%%%%%%%%%%%%

%\newcommand{\fullwidth}{}
%\newcommand{\normalwidth}{}



%% makes a snazzy t-chart for evaluating functions
%\newenvironment{tchart}{\rowcolors{2}{}{background!90!textColor}\array}{\endarray}

%%This is to help with formatting on future title pages.
\newenvironment{sectionOutcomes}{}{}



%% Flowchart stuff
%\tikzstyle{startstop} = [rectangle, rounded corners, minimum width=3cm, minimum height=1cm,text centered, draw=black]
%\tikzstyle{question} = [rectangle, minimum width=3cm, minimum height=1cm, text centered, draw=black]
%\tikzstyle{decision} = [trapezium, trapezium left angle=70, trapezium right angle=110, minimum width=3cm, minimum height=1cm, text centered, draw=black]
%\tikzstyle{question} = [rectangle, rounded corners, minimum width=3cm, minimum height=1cm,text centered, draw=black]
%\tikzstyle{process} = [rectangle, minimum width=3cm, minimum height=1cm, text centered, draw=black]
%\tikzstyle{decision} = [trapezium, trapezium left angle=70, trapezium right angle=110, minimum width=3cm, minimum height=1cm, text centered, draw=black]

\outcome{Convert between polar and Cartesian coordinates.}
\outcome{Convert between the Cartesian and polar reprentation of a curve.}
\outcome{Determine whether different polar representations represent the same point in the $(x,y)$-plane} 
\outcome{Use the Cartesian to polar method to plot polar graphs.}
\outcome{Understand the difference between a curve and the choice of coordinates used to describe the curve}

\title[Dig-In:]{Introduction to polar coordinates}

\begin{document}
\begin{abstract}
Polar coordinates are coordinates based on an angle and a radius.
\end{abstract}
\maketitle

\section{Polar coordinates}

%Now we focus on a special type of parametric equations, those of the
%form:
%\begin{align*} 
%  x(\theta) &= r(\theta) \cdot \cos(\theta)\\
%  y(\theta) &= r(\theta) \cdot \sin(\theta)
%\end{align*}
%where $r(\theta)$ is a function of $\theta$.  When working with
%parametric equations of this form, it is common to notate
%\[
%(r \cdot \cos(\theta), r\cdot \sin(\theta)) \text{ as } (r,\theta)
%\]
%and state that we are working in \textit{polar coordinates}.

\begin{definition}
  An ordered pair consisting of a radius and an angle $(r,\theta)$
  can be graphed as
  \begin{align*}
    x &= r\cdot \cos(\theta)\\
    y &= r\cdot \sin(\theta)
  \end{align*}
  meaning:
  \begin{image}[2in]
    \begin{tikzpicture}
	\draw[thick,->] (0,0) node [below] {$O$} -- (3,0) node [below] {horizontal axis} ;
	\filldraw (0,0) circle (1.5pt);
	\filldraw [rotate=55] (2,0) circle (1.5pt);
	\draw [thick,rotate=55] (0,0)-- node [rotate=55,pos=.5,above] {$r$} (2,0) node [above] {$(r,\theta)$};
	\draw [->] (.75,0) arc(0:55:.75); 
	\draw [rotate=27.5] (1,0) node {$\theta$};
    \end{tikzpicture}
  \end{image}
  Coordinates of this type are called \dfn{polar coordinates}.
\end{definition}

Polar coordinates are great for certain situations. However, there is
a price to pay. Every point in the plane has more than one of
description in polar coordinates.

\begin{question}
  Which of the following represent the origin, $(0,0)$, in
  $(x,y)$-coordinates?
  \begin{selectAll}
    \choice[correct]{$(0,0)$}
    \choice[correct]{$(0,\pi)$}
    \choice[correct]{$(0,-\pi)$}
  \end{selectAll}
  \begin{feedback}
    All of these represent the origin, since $(0,\theta)$ represents
    the origin for any angle $\theta$.
  \end{feedback}
\end{question}


\begin{example}
Plot the following points in polar coordinates:
\[
A =(1,\pi/4)\quad B=(1.5,\pi)\quad C = (2,-\pi/3)\quad D = (-1,\pi/4)
\]
\begin{explanation}
  It helps to use a ``polar grid'' to plot these points:
  \begin{image}
    \begin{tikzpicture}
      \begin{polaraxis}[
          xmin=0,xmax=360, ymin=0,ymax=3,
          xtick={0,30,45,60,90,120,135,150,180,210,225,240,270,300,315,330,360},
          xticklabels={$0$,$\frac{\pi}{6}$,$\frac{\pi}{4}$,$\frac{\pi}{3}$,$\frac{\pi}{2}$,$\frac{2\pi}{3}$,$\frac{3\pi}{4}$,$\frac{5\pi}{6}$,$\pi$,$\frac{7\pi}{6}$,$\frac{5\pi}{4}$,$\frac{4\pi}{3}$,$\frac{3\pi}{2}$,$\frac{5\pi}{3}$,$\frac{7\pi}{4}$,$\frac{11\pi}{6}$,$2\pi$},
          ytick={.5,1,...,2.5},%yticklabels={},
        ]
        %\addplot+[draw=none, mark=none,penColor,domain=0:360,samples=100,smooth] {1};
      \end{polaraxis}
    \end{tikzpicture}
  \end{image}
  To place the point $A$, go out $1$ unit along the horizontal axis
  (putting you on the inner circle shown on the grid), then rotate
  \wordChoice{\choice{clockwise}\choice[correct]{counterclockwise}}
  $\pi/4$ radians (or $45^\circ$).
  
  To plot $B$, go out $1.5$ units along the horizontal axis and rotate
  $\pi$ radians ($180^\circ$).
  
  To plot $C$, go out 2 units along the initial ray then rotate
  \wordChoice{\choice[correct]{clockwise}\choice{counterclockwise}}
  $\pi/3$ radians, as the angle given is negative.

  To plot $D$, move along the initial ray ``$-1$'' units, in other
  words, ``back up'' $1$ unit, then rotate
  \wordChoice{\choice{clockwise}\choice[correct]{counterclockwise}} by
  $\pi/4$.
  \begin{hint}
    \begin{image}
      \begin{tikzpicture}
        \begin{polaraxis}[
            xmin=0,xmax=360, ymin=0,ymax=3,
            xtick={0,30,45,60,90,120,135,150,180,210,225,240,270,300,315,330,360},
            xticklabels={$0$,$\frac{\pi}{6}$,$\frac{\pi}{4}$,$\frac{\pi}{3}$,$\frac{\pi}{2}$,$\frac{2\pi}{3}$,$\frac{3\pi}{4}$,$\frac{5\pi}{6}$,$\pi$,$\frac{7\pi}{6}$,$\frac{5\pi}{4}$,$\frac{4\pi}{3}$,$\frac{3\pi}{2}$,$\frac{5\pi}{3}$,$\frac{7\pi}{4}$,$\frac{11\pi}{6}$,$2\pi$},
            ytick={.5,1,...,2.5},%yticklabels={},
          ]
          \filldraw [rotate=45] (100,0) circle (1.5pt) node [above right] {$A$};
          \filldraw [rotate=180] (150,0) circle (1.5pt) node [above] {$B$};
          \filldraw [rotate=-60] (200,0) circle (1.5pt) node [below right] {$C$};
          \filldraw [rotate=45] (-100,0) circle (1.5pt) node [below] {$D$};
        \end{polaraxis}
      \end{tikzpicture}
    \end{image}
  \end{hint}
\end{explanation}
\end{example}


It is useful to recognize both the rectangular (or, Cartesian)
coordinates of a point in the plane and its polar coordinates.

\begin{theorem}
Given a point $P=(r,\theta)$ in polar coordinates, rectangular
coordinates are given by
\[
x=r\cos \theta\qquad y=r\sin \theta.
\]
Given a point $Q=(x,y)$ in rectangular coordinates, polar coordinates
are given by
\[
r^2=x^2+y^2\qquad \tan \theta = \frac yx.
\]
\end{theorem}

\begin{question}
  Let $P=(2,2\pi/3)$ be a point in polar coordinates. Describe $P$ in
  rectangular coordinates.
  \begin{prompt}
    \[
    P = (\answer{2\cos (2\pi/3)}, \answer{2\sin (2\pi/3)})
    \]
  \end{prompt}
  \begin{question}
  Let $Q=(-1,5\pi/4)$ be a point in polar coordinates. Describe $Q$ in
  rectangular coordinates.
  \begin{prompt}
    \[
    Q = (\answer{-1\cos (5\pi/4)}, \answer{-1\sin (5\pi/4)})
    \]
  \end{prompt}
\end{question}
\end{question}

\begin{question}
  Let $P=(1,2)$ be a point in rectangular coordinates. Describe $P$ in
  polar coordinates.
  \begin{prompt}
    \[
    P = (\answer{\sqrt{5}}, \answer{\arctan(2)})
    \]
  \end{prompt}
  \begin{question}
  Let $Q=(-1,1)$ be a point in rectangular coordinates. Describe $Q$ in
  polar coordinates.
  \begin{prompt}
    \[
    Q = (\answer{-\sqrt{2}}, \arctan(-1))
    \]
    \begin{hint}
      We'll tell you the angle, you think about the radius.
    \end{hint}
  \end{prompt}
\end{question}
\end{question}

\section{Polar graphs}

Let's talk about how to plot polar functions. A polar function
$r(\theta)$ corresponds to the parametric function:
\begin{align*} 
  x(\theta) &= r(\theta) \cdot \cos(\theta)\\
  y(\theta) &= r(\theta) \cdot \sin(\theta)
\end{align*}
However, if you are sketching a polar function by hand, there are some
tricks that can help. If you want to sketch $r(\theta)$, it is often
useful to first set $\theta = x$, and plot $y=r(x)$ in rectangular
coordinates. Let's just work examples. It is my belief that ``doing
things'' is better than ``describing.''

\begin{example}
  Sketch the polar function $r=1+\cos \theta$ on $[0,2\pi]$.
  \begin{explanation}
    While one could make a table of values and plot them in polar
    coordinates, it is often more useful to first set $\theta=x$ and
    then plot $y=r(x)$. This is what we'll do, starting with $x = \pi/4$:
    \begin{image}%% 45
      \begin{tikzpicture}
	\begin{axis}[
            domain=0:2*pi,
            xmin=-.3,xmax=6.32,ymin=-.3,ymax=2.3,
            axis lines =middle, xlabel=$x$, ylabel=$y$,
            every axis y label/.style={at=(current axis.above origin),anchor=south},
            every axis x label/.style={at=(current axis.right of origin),anchor=west},
            xtick={0,.785,...,6.28},
            xticklabels={$0$,$\frac{\pi}{4}$,$\frac{\pi}{2}$,$\frac{3\pi}{4}$,$\pi$,$\frac{5\pi}{4}$,$\frac{3\pi}{2}$,$\frac{7\pi}{4}$,$2\pi$},
            ytick={0,1,2},
          ]
          \addplot [dashed, smooth,] {1+cos(deg(x)};
	  \addplot [very thick, penColor, smooth,domain=0:.785] {1+cos(deg(x)};
        \end{axis}
      \end{tikzpicture}
      \qquad
       \begin{tikzpicture}
          \begin{polaraxis}[
              xtick={0,45,...,360},
              xticklabels={$0$,$\frac{\pi}{4}$,$\frac{\pi}{2}$,$\frac{3\pi}{4}$,$\pi$,$\frac{5\pi}{4}$,$\frac{3\pi}{2}$,$\frac{7\pi}{4}$,$2\pi$},
              ytick={.5,1,...,2},
            ]
            \addplot+[very thick, mark=none,penColor,domain=0:45,samples=100,smooth] {1+cos(x)};
          \end{polaraxis}
         \end{tikzpicture}
    \end{image}

    
    \begin{image}%% 90
      \begin{tikzpicture}
	\begin{axis}[
            domain=0:2*pi,
            xmin=-.3,xmax=6.32,ymin=-.3,ymax=2.3,
            axis lines =middle, xlabel=$x$, ylabel=$y$,
            every axis y label/.style={at=(current axis.above origin),anchor=south},
            every axis x label/.style={at=(current axis.right of origin),anchor=west},
            xtick={0,.785,...,6.28},
            xticklabels={$0$,$\frac{\pi}{4}$,$\frac{\pi}{2}$,$\frac{3\pi}{4}$,$\pi$,$\frac{5\pi}{4}$,$\frac{3\pi}{2}$,$\frac{7\pi}{4}$,$2\pi$},
            ytick={0,1,2},
          ]
          \addplot [dashed, smooth,] {1+cos(deg(x)};
	  \addplot [very thick, penColor, smooth,domain=0:1.57] {1+cos(deg(x))};
        \end{axis}
      \end{tikzpicture}
      \qquad
      \begin{tikzpicture}
          \begin{polaraxis}[
              xtick={0,45,...,360},
              xticklabels={$0$,$\frac{\pi}{4}$,$\frac{\pi}{2}$,$\frac{3\pi}{4}$,$\pi$,$\frac{5\pi}{4}$,$\frac{3\pi}{2}$,$\frac{7\pi}{4}$,$2\pi$},
              ytick={.5,1,...,2},
            ]
            \addplot+[very thick, mark=none,penColor,domain=0:90,samples=100,smooth] {1+cos(x)};
          \end{polaraxis}
         \end{tikzpicture}
    \end{image}
    
    
    \begin{image}%% 135
      \begin{tikzpicture}
	\begin{axis}[
            domain=0:2*pi,
            xmin=-.3,xmax=6.32,ymin=-.3,ymax=2.3,
              axis lines =middle, xlabel=$x$, ylabel=$y$,
              every axis y label/.style={at=(current axis.above origin),anchor=south},
              every axis x label/.style={at=(current axis.right of origin),anchor=west},
              xtick={0,.785,...,6.28},
              xticklabels={$0$,$\frac{\pi}{4}$,$\frac{\pi}{2}$,$\frac{3\pi}{4}$,$\pi$,$\frac{5\pi}{4}$,$\frac{3\pi}{2}$,$\frac{7\pi}{4}$,$2\pi$},
              ytick={0,1,2},
            ]
            \addplot [dashed, smooth,] {1+cos(deg(x)};
	    \addplot [very thick, penColor, smooth,domain=0:2.355] {1+cos(deg(x))};
          \end{axis}
        \end{tikzpicture}
        \qquad
        \begin{tikzpicture}
          \begin{polaraxis}[
              xtick={0,45,...,360},
              xticklabels={$0$,$\frac{\pi}{4}$,$\frac{\pi}{2}$,$\frac{3\pi}{4}$,$\pi$,$\frac{5\pi}{4}$,$\frac{3\pi}{2}$,$\frac{7\pi}{4}$,$2\pi$},
              ytick={.5,1,...,2},
            ]
            \addplot+[very thick, mark=none,penColor,domain=0:135,samples=100,smooth] {1+cos(x)};
          \end{polaraxis}
         \end{tikzpicture}
      \end{image}

    
       \begin{image}%% 180
         \begin{tikzpicture}
	   \begin{axis}[
               domain=0:2*pi,
               xmin=-.3,xmax=6.32,ymin=-.3,ymax=2.3,
               axis lines =middle, xlabel=$x$, ylabel=$y$,
               every axis y label/.style={at=(current axis.above origin),anchor=south},
              every axis x label/.style={at=(current axis.right of origin),anchor=west},
              xtick={0,.785,...,6.28},
              xticklabels={$0$,$\frac{\pi}{4}$,$\frac{\pi}{2}$,$\frac{3\pi}{4}$,$\pi$,$\frac{5\pi}{4}$,$\frac{3\pi}{2}$,$\frac{7\pi}{4}$,$2\pi$},
              ytick={0,1,2},
             ]
            \addplot [dashed, smooth,] {1+cos(deg(x)};
	    \addplot [very thick, penColor, smooth,domain=0:3.14] {1+cos(deg(x))};
           \end{axis}
        \end{tikzpicture}
        \qquad
         \begin{tikzpicture}
          \begin{polaraxis}[
              xtick={0,45,...,360},
              xticklabels={$0$,$\frac{\pi}{4}$,$\frac{\pi}{2}$,$\frac{3\pi}{4}$,$\pi$,$\frac{5\pi}{4}$,$\frac{3\pi}{2}$,$\frac{7\pi}{4}$,$2\pi$},
              ytick={.5,1,...,2},
            ]
            \addplot+[very thick, mark=none,penColor,domain=0:180,samples=100,smooth] {1+cos(x)};
          \end{polaraxis}
         \end{tikzpicture}
      \end{image}

       
       \begin{image}%% 225
         \begin{tikzpicture}
	   \begin{axis}[
               domain=0:2*pi,
               xmin=-.3,xmax=6.32,ymin=-.3,ymax=2.3,
               axis lines =middle, xlabel=$x$, ylabel=$y$,
               every axis y label/.style={at=(current axis.above origin),anchor=south},
              every axis x label/.style={at=(current axis.right of origin),anchor=west},
              xtick={0,.785,...,6.28},
              xticklabels={$0$,$\frac{\pi}{4}$,$\frac{\pi}{2}$,$\frac{3\pi}{4}$,$\pi$,$\frac{5\pi}{4}$,$\frac{3\pi}{2}$,$\frac{7\pi}{4}$,$2\pi$},
              ytick={0,1,2},
             ]
            \addplot [dashed, smooth,] {1+cos(deg(x)};
	    \addplot [very thick, penColor, smooth,domain=0:3.93] {1+cos(deg(x))};
           \end{axis}
        \end{tikzpicture}
        \qquad
        \begin{tikzpicture}
          \begin{polaraxis}[
              xtick={0,45,...,360},
              xticklabels={$0$,$\frac{\pi}{4}$,$\frac{\pi}{2}$,$\frac{3\pi}{4}$,$\pi$,$\frac{5\pi}{4}$,$\frac{3\pi}{2}$,$\frac{7\pi}{4}$,$2\pi$},
              ytick={.5,1,...,2},
            ]
            \addplot+[very thick, mark=none,penColor,domain=0:225,samples=100,smooth] {1+cos(x)};
          \end{polaraxis}
         \end{tikzpicture}
       \end{image}

       
       \begin{image}%% 270
         \begin{tikzpicture}
	   \begin{axis}[
               domain=0:2*pi,
               xmin=-.3,xmax=6.32,ymin=-.3,ymax=2.3,
               axis lines =middle, xlabel=$x$, ylabel=$y$,
               every axis y label/.style={at=(current axis.above origin),anchor=south},
              every axis x label/.style={at=(current axis.right of origin),anchor=west},
              xtick={0,.785,...,6.28},
              xticklabels={$0$,$\frac{\pi}{4}$,$\frac{\pi}{2}$,$\frac{3\pi}{4}$,$\pi$,$\frac{5\pi}{4}$,$\frac{3\pi}{2}$,$\frac{7\pi}{4}$,$2\pi$},
              ytick={0,1,2},
             ]
            \addplot [dashed, smooth,] {1+cos(deg(x)};
	    \addplot [very thick, penColor, smooth,domain=0:4.71] {1+cos(deg(x))};
           \end{axis}
        \end{tikzpicture}
        \qquad
        \begin{tikzpicture}
          \begin{polaraxis}[
              xtick={0,45,...,360},
              xticklabels={$0$,$\frac{\pi}{4}$,$\frac{\pi}{2}$,$\frac{3\pi}{4}$,$\pi$,$\frac{5\pi}{4}$,$\frac{3\pi}{2}$,$\frac{7\pi}{4}$,$2\pi$},
              ytick={.5,1,...,2},
            ]
            \addplot+[very thick, mark=none,penColor,domain=0:270,samples=100,smooth] {1+cos(x)};
          \end{polaraxis}
         \end{tikzpicture}
       \end{image}

       
       \begin{image}%% 315
         \begin{tikzpicture}
	   \begin{axis}[
               domain=0:2*pi,
               xmin=-.3,xmax=6.32,ymin=-.3,ymax=2.3,
               axis lines =middle, xlabel=$x$, ylabel=$y$,
               every axis y label/.style={at=(current axis.above origin),anchor=south},
              every axis x label/.style={at=(current axis.right of origin),anchor=west},
              xtick={0,.785,...,6.28},
              xticklabels={$0$,$\frac{\pi}{4}$,$\frac{\pi}{2}$,$\frac{3\pi}{4}$,$\pi$,$\frac{5\pi}{4}$,$\frac{3\pi}{2}$,$\frac{7\pi}{4}$,$2\pi$},
              ytick={0,1,2},
             ]
            \addplot [dashed, smooth,] {1+cos(deg(x)};
	    \addplot [very thick, penColor, smooth,domain=0:5.5] {1+cos(deg(x))};
           \end{axis}
        \end{tikzpicture}
        \qquad
        \begin{tikzpicture}
          \begin{polaraxis}[
              xtick={0,45,...,360},
              xticklabels={$0$,$\frac{\pi}{4}$,$\frac{\pi}{2}$,$\frac{3\pi}{4}$,$\pi$,$\frac{5\pi}{4}$,$\frac{3\pi}{2}$,$\frac{7\pi}{4}$,$2\pi$},
              ytick={.5,1,...,2},
            ]
            \addplot+[very thick, mark=none,penColor,domain=0:315,samples=100,smooth] {1+cos(x)};
          \end{polaraxis}
         \end{tikzpicture}
       \end{image}

       
       \begin{image}%% 360
         \begin{tikzpicture}
	   \begin{axis}[
               domain=0:2*pi,
               xmin=-.3,xmax=6.32,ymin=-.3,ymax=2.3,
               axis lines =middle, xlabel=$x$, ylabel=$y$,
               every axis y label/.style={at=(current axis.above origin),anchor=south},
              every axis x label/.style={at=(current axis.right of origin),anchor=west},
              xtick={0,.785,...,6.28},
              xticklabels={$0$,$\frac{\pi}{4}$,$\frac{\pi}{2}$,$\frac{3\pi}{4}$,$\pi$,$\frac{5\pi}{4}$,$\frac{3\pi}{2}$,$\frac{7\pi}{4}$,$2\pi$},
              ytick={0,1,2},
             ]
            \addplot [dashed, smooth,] {1+cos(deg(x)};
	    \addplot [very thick, penColor, smooth,domain=0:6.28] {1+cos(deg(x))};
           \end{axis}
        \end{tikzpicture}
        \qquad
        \begin{tikzpicture}
          \begin{polaraxis}[
              xtick={0,45,...,360},
              xticklabels={$0$,$\frac{\pi}{4}$,$\frac{\pi}{2}$,$\frac{3\pi}{4}$,$\pi$,$\frac{5\pi}{4}$,$\frac{3\pi}{2}$,$\frac{7\pi}{4}$,$2\pi$},
              ytick={.5,1,...,2},
            ]
            \addplot+[very thick, mark=none,penColor,domain=0:360,samples=100,smooth] {1+cos(x)};
          \end{polaraxis}
         \end{tikzpicture}
       \end{image}
  \end{explanation}
\end{example}











\begin{example}
  Sketch the polar function $r=\cos(2 \theta)$ on $[0,2\pi]$.
  \begin{explanation}
    While one could make a table of values and plot them in polar
    coordinates, it is often more useful to first set $\theta=x$ and
    then plot $y=r(x)$. This is what we'll do, starting with $x = \pi/4$:
    \begin{image}%% 45
      \begin{tikzpicture}
	\begin{axis}[
            domain=0:2*pi,
            xmin=-.3,xmax=6.32,ymin=-1.3,ymax=1.3,
            axis lines =middle, xlabel=$x$, ylabel=$y$,
            every axis y label/.style={at=(current axis.above origin),anchor=south},
            every axis x label/.style={at=(current axis.right of origin),anchor=west},
            xtick={0,.785,...,6.28},
            xticklabels={$0$,$\frac{\pi}{4}$,$\frac{\pi}{2}$,$\frac{3\pi}{4}$,$\pi$,$\frac{5\pi}{4}$,$\frac{3\pi}{2}$,$\frac{7\pi}{4}$,$2\pi$},
            ytick={0,1,2},
          ]
          \addplot [dashed, smooth,] {cos(2*deg(x))};
	  \addplot [very thick, penColor, smooth,domain=0:.785] {cos(2*deg(x)};
        \end{axis}
      \end{tikzpicture}
      \qquad
      \begin{tikzpicture}
          \begin{polaraxis}[
              xtick={0,45,...,360},
              xticklabels={$0$,$\frac{\pi}{4}$,$\frac{\pi}{2}$,$\frac{3\pi}{4}$,$\pi$,$\frac{5\pi}{4}$,$\frac{3\pi}{2}$,$\frac{7\pi}{4}$,$2\pi$},
              ytick={.25,.5,...,1},
            ]
            \addplot+[very thick, mark=none,penColor,domain=0:45,samples=100,smooth] {cos(2*x)};
          \end{polaraxis}
         \end{tikzpicture}
    \end{image}

    
    \begin{image}%% 90
      \begin{tikzpicture}
	\begin{axis}[
            domain=0:2*pi,
            xmin=-.3,xmax=6.32,ymin=-1.3,ymax=1.3,
            axis lines =middle, xlabel=$x$, ylabel=$y$,
            every axis y label/.style={at=(current axis.above origin),anchor=south},
            every axis x label/.style={at=(current axis.right of origin),anchor=west},
            xtick={0,.785,...,6.28},
            xticklabels={$0$,$\frac{\pi}{4}$,$\frac{\pi}{2}$,$\frac{3\pi}{4}$,$\pi$,$\frac{5\pi}{4}$,$\frac{3\pi}{2}$,$\frac{7\pi}{4}$,$2\pi$},
            ytick={0,1,2},
          ]
          \addplot [dashed, smooth,] {cos(2*deg(x))};
	  \addplot [very thick, penColor, smooth,domain=0:1.57] {cos(2*deg(x))};
        \end{axis}
      \end{tikzpicture}
      \qquad
      \begin{tikzpicture}
          \begin{polaraxis}[
              xtick={0,45,...,360},
              xticklabels={$0$,$\frac{\pi}{4}$,$\frac{\pi}{2}$,$\frac{3\pi}{4}$,$\pi$,$\frac{5\pi}{4}$,$\frac{3\pi}{2}$,$\frac{7\pi}{4}$,$2\pi$},
              ytick={.25,.5,...,1},
            ]
            \addplot+[very thick, mark=none,penColor,domain=0:90,samples=100,smooth] {cos(2*x)};
          \end{polaraxis}
         \end{tikzpicture}
    \end{image}
    
    
    \begin{image}%% 135
      \begin{tikzpicture}
	\begin{axis}[
            domain=0:2*pi,
            xmin=-.3,xmax=6.32,ymin=-1.3,ymax=1.3,
              axis lines =middle, xlabel=$x$, ylabel=$y$,
              every axis y label/.style={at=(current axis.above origin),anchor=south},
              every axis x label/.style={at=(current axis.right of origin),anchor=west},
              xtick={0,.785,...,6.28},
              xticklabels={$0$,$\frac{\pi}{4}$,$\frac{\pi}{2}$,$\frac{3\pi}{4}$,$\pi$,$\frac{5\pi}{4}$,$\frac{3\pi}{2}$,$\frac{7\pi}{4}$,$2\pi$},
              ytick={0,1,2},
            ]
            \addplot [dashed, smooth,] {cos(2*deg(x))};
	    \addplot [very thick, penColor, smooth,domain=0:2.355] {cos(2*deg(x))};
          \end{axis}
        \end{tikzpicture}
        \qquad
        \begin{tikzpicture}
          \begin{polaraxis}[
              xtick={0,45,...,360},
              xticklabels={$0$,$\frac{\pi}{4}$,$\frac{\pi}{2}$,$\frac{3\pi}{4}$,$\pi$,$\frac{5\pi}{4}$,$\frac{3\pi}{2}$,$\frac{7\pi}{4}$,$2\pi$},
              ytick={.25,.5,...,1},
            ]
            \addplot+[very thick, mark=none,penColor,domain=0:135,samples=100,smooth] {cos(2*x)};
          \end{polaraxis}
         \end{tikzpicture}
      \end{image}

    
       \begin{image}%% 180
         \begin{tikzpicture}
	   \begin{axis}[
               domain=0:2*pi,
               xmin=-.3,xmax=6.32,ymin=-1.3,ymax=1.3,
               axis lines =middle, xlabel=$x$, ylabel=$y$,
               every axis y label/.style={at=(current axis.above origin),anchor=south},
              every axis x label/.style={at=(current axis.right of origin),anchor=west},
              xtick={0,.785,...,6.28},
              xticklabels={$0$,$\frac{\pi}{4}$,$\frac{\pi}{2}$,$\frac{3\pi}{4}$,$\pi$,$\frac{5\pi}{4}$,$\frac{3\pi}{2}$,$\frac{7\pi}{4}$,$2\pi$},
              ytick={0,1,2},
             ]
            \addplot [dashed, smooth,] {cos(2*deg(x))};
	    \addplot [very thick, penColor, smooth,domain=0:3.14] {cos(2*deg(x))};
           \end{axis}
        \end{tikzpicture}
        \qquad
        \begin{tikzpicture}
          \begin{polaraxis}[
              xtick={0,45,...,360},
              xticklabels={$0$,$\frac{\pi}{4}$,$\frac{\pi}{2}$,$\frac{3\pi}{4}$,$\pi$,$\frac{5\pi}{4}$,$\frac{3\pi}{2}$,$\frac{7\pi}{4}$,$2\pi$},
              ytick={.25,.5,...,1},
            ]
            \addplot+[very thick, mark=none,penColor,domain=0:180,samples=100,smooth] {cos(2*x)};
          \end{polaraxis}
         \end{tikzpicture}
      \end{image}

       
       \begin{image}%% 225
         \begin{tikzpicture}
	   \begin{axis}[
               domain=0:2*pi,
               xmin=-.3,xmax=6.32,ymin=-1.3,ymax=1.3,
               axis lines =middle, xlabel=$x$, ylabel=$y$,
               every axis y label/.style={at=(current axis.above origin),anchor=south},
              every axis x label/.style={at=(current axis.right of origin),anchor=west},
              xtick={0,.785,...,6.28},
              xticklabels={$0$,$\frac{\pi}{4}$,$\frac{\pi}{2}$,$\frac{3\pi}{4}$,$\pi$,$\frac{5\pi}{4}$,$\frac{3\pi}{2}$,$\frac{7\pi}{4}$,$2\pi$},
              ytick={0,1,2},
             ]
            \addplot [dashed, smooth,] {cos(2*deg(x))};
	    \addplot [very thick, penColor, smooth,domain=0:3.93] {cos(2*deg(x))};
           \end{axis}
        \end{tikzpicture}
        \qquad
        \begin{tikzpicture}
          \begin{polaraxis}[
              xtick={0,45,...,360},
              xticklabels={$0$,$\frac{\pi}{4}$,$\frac{\pi}{2}$,$\frac{3\pi}{4}$,$\pi$,$\frac{5\pi}{4}$,$\frac{3\pi}{2}$,$\frac{7\pi}{4}$,$2\pi$},
              ytick={.25,.5,...,1},
            ]
            \addplot+[very thick, mark=none,penColor,domain=0:225,samples=100,smooth] {cos(2*x)};
          \end{polaraxis}
         \end{tikzpicture}
       \end{image}

       
       \begin{image}%% 270
         \begin{tikzpicture}
	   \begin{axis}[
               domain=0:2*pi,
               xmin=-.3,xmax=6.32,ymin=-1.3,ymax=1.3,
               axis lines =middle, xlabel=$x$, ylabel=$y$,
               every axis y label/.style={at=(current axis.above origin),anchor=south},
              every axis x label/.style={at=(current axis.right of origin),anchor=west},
              xtick={0,.785,...,6.28},
              xticklabels={$0$,$\frac{\pi}{4}$,$\frac{\pi}{2}$,$\frac{3\pi}{4}$,$\pi$,$\frac{5\pi}{4}$,$\frac{3\pi}{2}$,$\frac{7\pi}{4}$,$2\pi$},
              ytick={0,1,2},
             ]
            \addplot [dashed, smooth,] {cos(2*deg(x))};
	    \addplot [very thick, penColor, smooth,domain=0:4.71] {cos(2*deg(x))};
           \end{axis}
        \end{tikzpicture}
        \qquad
        \begin{tikzpicture}
          \begin{polaraxis}[
              xtick={0,45,...,360},
              xticklabels={$0$,$\frac{\pi}{4}$,$\frac{\pi}{2}$,$\frac{3\pi}{4}$,$\pi$,$\frac{5\pi}{4}$,$\frac{3\pi}{2}$,$\frac{7\pi}{4}$,$2\pi$},
              ytick={.25,.5,...,1},
            ]
            \addplot+[very thick, mark=none,penColor,domain=0:270,samples=100,smooth] {cos(2*x)};
          \end{polaraxis}
         \end{tikzpicture}		
       \end{image}

       
       \begin{image}%% 315
         \begin{tikzpicture}
	   \begin{axis}[
               domain=0:2*pi,
               xmin=-.3,xmax=6.32,ymin=-1.3,ymax=1.3,
               axis lines =middle, xlabel=$x$, ylabel=$y$,
               every axis y label/.style={at=(current axis.above origin),anchor=south},
              every axis x label/.style={at=(current axis.right of origin),anchor=west},
              xtick={0,.785,...,6.28},
              xticklabels={$0$,$\frac{\pi}{4}$,$\frac{\pi}{2}$,$\frac{3\pi}{4}$,$\pi$,$\frac{5\pi}{4}$,$\frac{3\pi}{2}$,$\frac{7\pi}{4}$,$2\pi$},
              ytick={0,1,2},
             ]
            \addplot [dashed, smooth,] {cos(2*deg(x))};
	    \addplot [very thick, penColor, smooth,domain=0:5.5] {cos(2*deg(x))};
           \end{axis}
        \end{tikzpicture}
        \qquad
        \begin{tikzpicture}
          \begin{polaraxis}[
              xtick={0,45,...,360},
              xticklabels={$0$,$\frac{\pi}{4}$,$\frac{\pi}{2}$,$\frac{3\pi}{4}$,$\pi$,$\frac{5\pi}{4}$,$\frac{3\pi}{2}$,$\frac{7\pi}{4}$,$2\pi$},
              ytick={.25,.5,...,1},
            ]
            \addplot+[very thick, mark=none,penColor,domain=0:315,samples=100,smooth] {cos(2*x)};
          \end{polaraxis}
         \end{tikzpicture}
       \end{image}

       
       \begin{image}%% 360
         \begin{tikzpicture}
	   \begin{axis}[
               domain=0:2*pi,
               xmin=-.3,xmax=6.32,ymin=-1.3,ymax=1.3,
               axis lines =middle, xlabel=$x$, ylabel=$y$,
               every axis y label/.style={at=(current axis.above origin),anchor=south},
              every axis x label/.style={at=(current axis.right of origin),anchor=west},
              xtick={0,.785,...,6.28},
              xticklabels={$0$,$\frac{\pi}{4}$,$\frac{\pi}{2}$,$\frac{3\pi}{4}$,$\pi$,$\frac{5\pi}{4}$,$\frac{3\pi}{2}$,$\frac{7\pi}{4}$,$2\pi$},
              ytick={0,1,2},
             ]
            \addplot [dashed, smooth,] {cos(2*deg(x))};
	    \addplot [very thick, penColor, smooth,domain=0:6.28] {cos(2*deg(x))};
           \end{axis}
        \end{tikzpicture}
        \qquad
        \begin{tikzpicture}
          \begin{polaraxis}[
              xtick={0,45,...,360},
              xticklabels={$0$,$\frac{\pi}{4}$,$\frac{\pi}{2}$,$\frac{3\pi}{4}$,$\pi$,$\frac{5\pi}{4}$,$\frac{3\pi}{2}$,$\frac{7\pi}{4}$,$2\pi$},
              ytick={.25,.5,...,1},
            ]
            \addplot+[very thick, mark=none,penColor,domain=0:360,samples=100,smooth] {cos(2*x)};
          \end{polaraxis}
         \end{tikzpicture}
       \end{image}
  \end{explanation}
\end{example}


\section{Converting to and from polar coordinates}


It is sometimes desirable to refer to a graph via a polar equation,
and other times by a rectangular equation.  Therefore it is necessary
to be able to convert between polar and rectangular functions.  Here
is the basic idea:

Given a function $y=f(x)$ in rectangular coordinates, polar coordinates
are given by setting
\[
x=r\cos(\theta)\qquad y=r\sin(\theta).
\]
and solving for $r$.

Given a function $r(\theta)$ in polar coordinates, rectangular
coordinates harder to find. The basic idea is to ``find'' $r\cdot
\cos(\theta)$ and $r\cdot \sin(\theta)$ and write:
\[
r\cos(\theta) = x\qquad r\sin(\theta) = y.
\]
Sometimes it is useful to remember that:
\[
r^2=x^2+y^2\qquad \tan \theta = \frac yx.
\]


\begin{example}
  Convert $y=x^2$ from rectangular coordinates to polar coordinates.
  \begin{explanation}
    Replace $y$ with $r\sin(\theta)$ and replace $x$ with $r\cos(\theta)$, giving:
      \begin{align*}
	y &=x^2\\
	r\sin(\theta) &= \answer[given]{r^2\cos^2(\theta)}\\
	\answer[given]{\frac{\sin(\theta)}{\cos^2(\theta)}}  &= r.
      \end{align*}
      We have found that $r=\sin\theta/\cos^2\theta =
      \tan\theta\sec\theta$. The domain of this polar function is
      $[\answer[given]{-\pi/2},\answer[given]{\pi/2}]$. Plot a few
      points to see how the familiar parabola is traced out by the
      polar equation.
  \end{explanation}
\end{example}

\begin{example}
  Convert $xy = 1$ from rectangular coordinates to polar coordinates.
  \begin{explanation}
    We again replace $x$ and $y$ using the standard identities and
    work to solve for $r$:
    \begin{align*}
	xy &= 1 \\
	\answer[given]{r\cos\theta\cdot r\sin\theta} & = 1\\
	r^2 & = \frac{1}{\cos\theta\sin\theta}\\
	r & = \frac{1}{\sqrt{\cos\theta\sin\theta}}\\
    \end{align*}
    This function is valid only when the product of
    $\cos\theta\sin\theta$ is positive. This occurs in the first and
    third quadrants, meaning the domain of this polar function is
    $(0,\pi/2)$ with $(\pi,3\pi/2)$.
      
    We can rewrite the original rectangular equation $xy=1$ as
    $y=1/x$. Note it only exists in the first and third quadrants.
  \end{explanation}
\end{example}

\begin{example}
   Convert $r=\frac{2}{\sin \theta-\cos\theta}$ from polar coordinates
   to rectangular coordinates.
   \begin{explanation}
     There is no set way to convert from polar to rectangular; in
     general, we look to form the products $r\cos \theta$ and
     $r\sin\theta$, and then replace these with $x$ and $y$,
     respectively. We start in this problem by multiplying both sides
     by $\sin\theta-\cos\theta$:
     \begin{align*}
       r &= \frac{2}{\sin\theta-\cos\theta} \\
       r(\sin\theta-\cos\theta) &= 2\\
       \answer[given]{r\sin\theta}-r\cos\theta &= 2. \qquad \text{Now replace with $y$ and $x$:}\\
       y-x &= 2\\
	  y &= x+2.
     \end{align*}
     The original polar equation, $r=2/(\sin\theta-\cos\theta)$ does
     not easily reveal that its graph is simply a line. However, our
     conversion shows that it is.
   \end{explanation}
\end{example}

\begin{example}
   Convert $r =2\cos \theta$ from polar coordinates to rectangular
   coordinates.        	
   \begin{explanation}
     By multiplying both sides by $r$, we obtain both an $r^2$ term
     and an $r\cos\theta$ term, which we replace with $x^2+y^2$ and
     $x$, respectively.
     \begin{align*}
       r &=2\cos\theta \\
       r^2 &= 2r\cos\theta \\
       \answer[given]{x^2+y^2} &= \answer[given]{2x}. 
     \end{align*}
     We recognize this as a circle. By completing the square we can
     find its radius and center.
     \begin{align*}
       x^2-2x+y^2 &= 0 \\
       \answer[given]{(x-1)^2 + y^2} &=1.
     \end{align*}
     The circle is centered at $(1,0)$ and has radius $1$.
   \end{explanation}
 \end{example}
\end{document}
