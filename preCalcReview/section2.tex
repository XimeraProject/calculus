\documentclass{ximera}

%\usepackage{todonotes}
%\usepackage{mathtools} %% Required for wide table Curl and Greens
%\usepackage{cuted} %% Required for wide table Curl and Greens
\newcommand{\todo}{}

\usepackage{multicol}

\usepackage{esint} % for \oiint
\ifxake%%https://math.meta.stackexchange.com/questions/9973/how-do-you-render-a-closed-surface-double-integral
\renewcommand{\oiint}{{\large\bigcirc}\kern-1.56em\iint}
\fi

\graphicspath{
  {./}
  {ximeraTutorial/}
  {basicPhilosophy/}
  {functionsOfSeveralVariables/}
  {normalVectors/}
  {lagrangeMultipliers/}
  {vectorFields/}
  {greensTheorem/}
  {shapeOfThingsToCome/}
  {dotProducts/}
  {partialDerivativesAndTheGradientVector/}
  {../ximeraTutorial/}
  {../productAndQuotientRules/exercises/}
  {../motionAndPathsInSpace/exercises/}
  {../normalVectors/exercisesParametricPlots/}
  {../continuityOfFunctionsOfSeveralVariables/exercises/}
  {../partialDerivativesAndTheGradientVector/exercises/}
  {../directionalDerivativeAndChainRule/exercises/}
  {../commonCoordinates/exercisesCylindricalCoordinates/}
  {../commonCoordinates/exercisesSphericalCoordinates/}
  {../greensTheorem/exercisesCurlAndLineIntegrals/}
  {../greensTheorem/exercisesDivergenceAndLineIntegrals/}
  {../shapeOfThingsToCome/exercisesDivergenceTheorem/}
  {../greensTheorem/}
  {../shapeOfThingsToCome/}
  {../separableDifferentialEquations/exercises/}
  {../dotproducts/}
  {../functionsOfSeveralVariables/}
  {../lagrangeMultipliers/}
  {../partialDerivativesAndTheGradientVector/}
  {../normalVectors/}
  {../vectorFields/}
}
\def\xmNotExpandableAsAccordion{true}

\newcommand{\mooculus}{\textsf{\textbf{MOOC}\textnormal{\textsf{ULUS}}}}

\usepackage{tkz-euclide}\usepackage{tikz}
\usepackage{tikz-cd}
\usetikzlibrary{arrows}
\tikzset{>=stealth,commutative diagrams/.cd,
  arrow style=tikz,diagrams={>=stealth}} %% cool arrow head
\tikzset{shorten <>/.style={ shorten >=#1, shorten <=#1 } } %% allows shorter vectors

\usetikzlibrary{backgrounds} %% for boxes around graphs
\usetikzlibrary{shapes,positioning}  %% Clouds and stars
\usetikzlibrary{matrix} %% for matrix
\usepgfplotslibrary{polar} %% for polar plots
\usepgfplotslibrary{fillbetween} %% to shade area between curves in TikZ
%\usetkzobj{all} %% obsolete

\usepackage[makeroom]{cancel} %% for strike outs
%\usepackage{mathtools} %% for pretty underbrace % Breaks Ximera
%\usepackage{multicol}
\usepackage{pgffor} %% required for integral for loops

\usepackage{tkz-tab}  %% for sign charts

%% http://tex.stackexchange.com/questions/66490/drawing-a-tikz-arc-specifying-the-center
%% Draws beach ball
\tikzset{pics/carc/.style args={#1:#2:#3}{code={\draw[pic actions] (#1:#3) arc(#1:#2:#3);}}}



\usepackage{array}
\setlength{\extrarowheight}{+.1cm}
\newdimen\digitwidth
\settowidth\digitwidth{9}
\def\divrule#1#2{
\noalign{\moveright#1\digitwidth
\vbox{\hrule width#2\digitwidth}}}





\newcommand{\RR}{\mathbb R}
\newcommand{\R}{\mathbb R}
\newcommand{\N}{\mathbb N}
\newcommand{\Z}{\mathbb Z}

\newcommand{\sagemath}{\textsf{SageMath}}


\renewcommand{\d}{\,d}
%\def\d{\mathop{}\!d}
%\def\d{\,d}

\AddToHook{begindocument}{%
  \renewcommand{\d}{\,d}     % lualatex redefines \d to underdot !!!
}

\pgfplotsset{
    every axis/.style={
        scale only axis,
        enlargelimits=false,
        trim axis left,
        trim axis right,
        clip=true,
    }
}

\newcommand{\dd}[2][]{\frac{\d #1}{\d #2}}
\newcommand{\pp}[2][]{\frac{\partial #1}{\partial #2}}
\renewcommand{\l}{\ell}
\newcommand{\ddx}{\frac{d}{\d x}}

\newcommand{\zeroOverZero}{\ensuremath{\boldsymbol{\tfrac{0}{0}}}}
\newcommand{\inftyOverInfty}{\ensuremath{\boldsymbol{\tfrac{\infty}{\infty}}}}
\newcommand{\zeroOverInfty}{\ensuremath{\boldsymbol{\tfrac{0}{\infty}}}}
\newcommand{\zeroTimesInfty}{\ensuremath{\small\boldsymbol{0\cdot \infty}}}
\newcommand{\inftyMinusInfty}{\ensuremath{\small\boldsymbol{\infty - \infty}}}
\newcommand{\oneToInfty}{\ensuremath{\boldsymbol{1^\infty}}}
\newcommand{\zeroToZero}{\ensuremath{\boldsymbol{0^0}}}
\newcommand{\inftyToZero}{\ensuremath{\boldsymbol{\infty^0}}}



\newcommand{\numOverZero}{\ensuremath{\boldsymbol{\tfrac{\#}{0}}}}
\newcommand{\dfn}{\textbf}
%\newcommand{\unit}{\,\mathrm}
\newcommand{\unit}{\mathop{}\!\mathrm}
\newcommand{\eval}[1]{\bigg[ #1 \bigg]}
\newcommand{\seq}[1]{\left( #1 \right)}
\renewcommand{\epsilon}{\varepsilon}
\renewcommand{\phi}{\varphi}


\renewcommand{\iff}{\Leftrightarrow}

\DeclareMathOperator{\arccot}{arccot}
\DeclareMathOperator{\arcsec}{arcsec}
\DeclareMathOperator{\arccsc}{arccsc}
\DeclareMathOperator{\si}{Si}
\DeclareMathOperator{\scal}{scal}
\DeclareMathOperator{\sign}{sign}


%% \newcommand{\tightoverset}[2]{% for arrow vec
%%   \mathop{#2}\limits^{\vbox to -.5ex{\kern-0.75ex\hbox{$#1$}\vss}}}
\newcommand{\arrowvec}[1]{{\overset{\rightharpoonup}{#1}}}
%\renewcommand{\vec}[1]{\arrowvec{\mathbf{#1}}}
\renewcommand{\vec}[1]{{\overset{\boldsymbol{\rightharpoonup}}{\mathbf{#1}}}\hspace{0in}}

\newcommand{\point}[1]{\left(#1\right)} %this allows \vector{ to be changed to \vector{ with a quick find and replace
\newcommand{\pt}[1]{\mathbf{#1}} %this allows \vec{ to be changed to \vec{ with a quick find and replace
\newcommand{\Lim}[2]{\lim_{\point{#1} \to \point{#2}}} %Bart, I changed this to point since I want to use it.  It runs through both of the exercise and exerciseE files in limits section, which is why it was in each document to start with.

\DeclareMathOperator{\proj}{\mathbf{proj}}
\newcommand{\veci}{{\boldsymbol{\hat{\imath}}}}
\newcommand{\vecj}{{\boldsymbol{\hat{\jmath}}}}
\newcommand{\veck}{{\boldsymbol{\hat{k}}}}
\newcommand{\vecl}{\vec{\boldsymbol{\l}}}
\newcommand{\uvec}[1]{\mathbf{\hat{#1}}}
\newcommand{\utan}{\mathbf{\hat{t}}}
\newcommand{\unormal}{\mathbf{\hat{n}}}
\newcommand{\ubinormal}{\mathbf{\hat{b}}}

\newcommand{\dotp}{\bullet}
\newcommand{\cross}{\boldsymbol\times}
\newcommand{\grad}{\boldsymbol\nabla}
\newcommand{\divergence}{\grad\dotp}
\newcommand{\curl}{\grad\cross}
%\DeclareMathOperator{\divergence}{divergence}
%\DeclareMathOperator{\curl}[1]{\grad\cross #1}
\newcommand{\lto}{\mathop{\longrightarrow\,}\limits}

\renewcommand{\bar}{\overline}

\colorlet{textColor}{black}
\colorlet{background}{white}
\colorlet{penColor}{blue!50!black} % Color of a curve in a plot
\colorlet{penColor2}{red!50!black}% Color of a curve in a plot
\colorlet{penColor3}{red!50!blue} % Color of a curve in a plot
\colorlet{penColor4}{green!50!black} % Color of a curve in a plot
\colorlet{penColor5}{orange!80!black} % Color of a curve in a plot
\colorlet{penColor6}{yellow!70!black} % Color of a curve in a plot
\colorlet{fill1}{penColor!20} % Color of fill in a plot
\colorlet{fill2}{penColor2!20} % Color of fill in a plot
\colorlet{fillp}{fill1} % Color of positive area
\colorlet{filln}{penColor2!20} % Color of negative area
\colorlet{fill3}{penColor3!20} % Fill
\colorlet{fill4}{penColor4!20} % Fill
\colorlet{fill5}{penColor5!20} % Fill
\colorlet{gridColor}{gray!50} % Color of grid in a plot

\newcommand{\surfaceColor}{violet}
\newcommand{\surfaceColorTwo}{redyellow}
\newcommand{\sliceColor}{greenyellow}




\pgfmathdeclarefunction{gauss}{2}{% gives gaussian
  \pgfmathparse{1/(#2*sqrt(2*pi))*exp(-((x-#1)^2)/(2*#2^2))}%
}


%%%%%%%%%%%%%
%% Vectors
%%%%%%%%%%%%%

%% Simple horiz vectors
\renewcommand{\vector}[1]{\left\langle #1\right\rangle}


%% %% Complex Horiz Vectors with angle brackets
%% \makeatletter
%% \renewcommand{\vector}[2][ , ]{\left\langle%
%%   \def\nextitem{\def\nextitem{#1}}%
%%   \@for \el:=#2\do{\nextitem\el}\right\rangle%
%% }
%% \makeatother

%% %% Vertical Vectors
%% \def\vector#1{\begin{bmatrix}\vecListA#1,,\end{bmatrix}}
%% \def\vecListA#1,{\if,#1,\else #1\cr \expandafter \vecListA \fi}

%%%%%%%%%%%%%
%% End of vectors
%%%%%%%%%%%%%

%\newcommand{\fullwidth}{}
%\newcommand{\normalwidth}{}



%% makes a snazzy t-chart for evaluating functions
%\newenvironment{tchart}{\rowcolors{2}{}{background!90!textColor}\array}{\endarray}

%%This is to help with formatting on future title pages.
\newenvironment{sectionOutcomes}{}{}



%% Flowchart stuff
%\tikzstyle{startstop} = [rectangle, rounded corners, minimum width=3cm, minimum height=1cm,text centered, draw=black]
%\tikzstyle{question} = [rectangle, minimum width=3cm, minimum height=1cm, text centered, draw=black]
%\tikzstyle{decision} = [trapezium, trapezium left angle=70, trapezium right angle=110, minimum width=3cm, minimum height=1cm, text centered, draw=black]
%\tikzstyle{question} = [rectangle, rounded corners, minimum width=3cm, minimum height=1cm,text centered, draw=black]
%\tikzstyle{process} = [rectangle, minimum width=3cm, minimum height=1cm, text centered, draw=black]
%\tikzstyle{decision} = [trapezium, trapezium left angle=70, trapezium right angle=110, minimum width=3cm, minimum height=1cm, text centered, draw=black]


\title{Precalculus Review - Section 2}

\begin{document}
\begin{abstract}
%
\end{abstract}
\maketitle

This section contains review material on:
\begin{itemize}
	\item Solving linear equations in one variable
	\item Solving inequalities in one variable
	\item Solving equations and inequalities involving absolute value
	\item Systems of two equations with two unknowns
\end{itemize}

To ``solve an equation'' means to find all values of the unknown variable(s) (these values are called solutions) that satisfy a given equation. In other words, a solution is a real number that, when substituted into the equation, gives the identity.  For example,$x=3$ is a solution of the equation $x^2-4x = 6-x^2$, since $x^2-4x = (3)^2-4(3) = -3$ and $6-x^2 = 6-(3)^2 = -3$; i.e., both sides are equal to $-3$. On the other hand, $x=4$ is not a solution of the above equation; the left side is $x^2-4x = (4)^2-4(4) = 0$, whereas the right side is $6-x^2 = 6-(4)^2 = -10$.

\section{Solving Linear Equations}
A linear equation $ax+b=0$ (assume that $a \neq 0$) has only one solution, namely $x=-\frac{b}{a}$. (If $a = 0$, we don't really have an equation!)

\begin{example}
Solve the equation $3(x+4) = -4(2-2x)$.
\end{example}
\begin{proof}[Solution]
We first simplify
\begin{align*}
	3(x+4) &= -4(2-2x)\\
	3x+12 &= -8 + 8x
\end{align*}
and then gather the like terms together
\begin{align*}
	3x-8x &= -8 - 12\\
	-5x &= -20.
\end{align*}
Dividing by $(-5)$, we get $x=4$.
\end{proof}

\begin{example}
Solve the equation $\displaystyle \frac{4x-1}{3} + \frac{x}{4} = -2$.
\end{example}
\begin{proof}[Solution]
Multiplying the equation by $12$, we get
\begin{align*}
	4(4x-1) + 3(x) &= -2(12)\\
	16x - 4 + 3x &= -24.
\end{align*}

Thus,
\begin{align*}
	19x &= -20\\
	x &= -\frac{20}{19}.
\end{align*}
\end{proof}

\begin{example}
Solve the equation $\displaystyle \frac{4}{x-3} = \frac{5}{x+2}$.
\end{example}
\begin{proof}[Solution]
Multiplying the given equation by $(x-3)(x+2)$, we get
\begin{align*}
	\frac{4(x-3)(x+2)}{x-3} &= \frac{5(x-3)(x+2)}{x+2}\\
	4(x+2) &= 5(x-3)
\end{align*}
(in other words, we ``cross-multiplied'' the equation).  Thus,
\begin{align*}
	4x+8 &= 5x-15\\
	x &= 23.
\end{align*}

We can check that our solution is correct.  Substituting $x=23$ into the left side, we get $\frac{4}{x-3} = \frac{4}{23-3} = \frac{4}{20} = \frac{1}{5}$;
substituting it into the right side, we get $\frac{5}{x+2} = \frac{5}{23+2} = \frac{5}{25} = \frac{1}{5}$.
\end{proof}

It is important to check that what we compute is really a solution.  The following example serves as a warning to that effect.

\begin{example}
Solve the equation $\displaystyle \frac{x}{x+2} - 1 = \frac{8}{x^2-4}$.
\end{example}
\begin{proof}[Solution]
Multiplying the given equation by $x^2-4 = (x+2)(x-2)$, we get
\begin{align*}
	\frac{x}{x+2}(x+2)(x-2) - 1 (x^2-4) &= \frac{8}{x^2-4}(x^2-4)\\
	x^2-2x-x^2+4 &= 8\\
	-2x &= 4\\
	x &= -2.
\end{align*}
However, $x=-2$ is not a solution, since neither of the two fractions is defined in that case.  Thus, the given equation does not have a solution.
\end{proof}

\section{Solving Quadratic Equations}
A quadratic equation $ax^2 + bx + c = 0$, with $a \neq 0$, can be solved by factoring, by completing the square, or by using the quadratic formula

\begin{theorem}[The Quadratic Formula]
The solutions of $ax^2+bx+c = 0$ are given by $$ x = \frac{-b \pm \sqrt{b^2-4c}}{2a} $$
\end{theorem}

The expression $D = b^2-4ac$ is called the discriminant of the quadratic equation.  If $D>0$ the equation has two distinct real solutions.  If $D=0$ it has one real solution.  If $D < 0$ the equation has no real solutions.

\begin{example}
Solve the equations.
\begin{enumerate}
	\item $x^2 + 5x - 24 = 0$
	\item $x^2 + 2x - 2 = 0$
\end{enumerate}
\end{example}
\begin{proof}[Solution]

\begin{enumerate}
	\item{From $x^2+5x-24 = (x+8)(x-3)=0$, it follows that either $x-3=0$ (in which case $x=3$), or $x+8=0$ (and thus $x= -8$).  So the solutions are $x= -8$ and $x=3$.}
	\item{Using the quadratic formula, we get \[ x = \frac{-2 \pm \sqrt{4+8}}{2} = \frac{-2\pm\sqrt{12}}{2} = \frac{-2\pm 2\sqrt{3}}{2} = -1 \pm \sqrt{3}.\]
		In the above computation, we simplified $\sqrt{12}$ using $\sqrt{12} = \sqrt{4 \cdot 3} = 2\sqrt{3}$.
		}
\end{enumerate}
\end{proof}

\begin{example}
Solve the equation $1 + \sqrt{2-x} = 2x$.
\end{example}
\begin{proof}[Solution]

Rearrange the terms first: $\sqrt{2-x} = 2x - 1$.  Squaring both sides we get
\begin{align*}
	2-x &= 4x^2-4x+1\\
	4x^2-3x-1 &= 0
\end{align*}
Thus,\[ x = \frac{3\pm \sqrt{9+16}}{8} = \frac{3 \pm 5}{8}.\]
So, $x = (3+5)/8 = 1$ and $x=(3-5)/8 = -1/4$ are candidates for solutions.  We have to check whether they really give solutions.
Substituting $x=1$ into the given equation, we get $1 + \sqrt{2-1} = 2(1)$; thus $x=1$ is a solution.  Substituting $x=-1/4$ into the
given equation, we get $1 + \sqrt{2+ 1/4} = 2(-1/4)$; i.e., 1 + 3/2 = -1/2; consequently, $x=-1/4$ is not a solution.
\end{proof}

\section{Inequalities}

Rules for inequalities state that we can apply the same rules that we use for equations, except when we have to multiply or divide an inequality by a negative number.  In that case, we reverse the direction of the inequality.  For example, the inequality $3<4$, multiplied by $(-5)$, gives $-15 > -20$.  Likewise, dividing $-4 \leq -2x < 6$ by $(-2)$, we get $2 \geq x > -3$.

To solve an inequality means to find all values of the unknown variable that satisfy it.  In the case of inequalities, the solution usually consists of an interval (or intervals) of real numbers.

\begin{example}
Solve the inequality $-(3+x) < 2(3x+2)$.
\end{example}
\begin{proof}[Solution]

We simplify first
\begin{align*}
	-(3+x) &< 2(3x+2) \\
	-3-x &< 6x+4\\
	-x-6x &< 4+3\\
	-7x &< 7.
\end{align*}
Dividing by $(-7)$, we get $x > -1$.  Using interval notation, we write the solution as $(-1, \infty)$.
\end{proof}

\begin{example}
Solve the inequalities $-11 < 3x+4 \leq 7$.
\end{example}
\begin{proof}[Solution]

We can work with both inequalities at the same time.  Adding $(-4)$ to both sides and dividing by $3$, we get
\begin{align*}
	-11 &< 3x+4 \leq 7\\
	-15 &< 3x \leq 3\\
	-5 &< x \leq 1.
\end{align*}
Thus, the solution is the interval $(-5, 1]$.
\end{proof}

Sometimes, we have to solve inequalities separately, as in the following example.

\begin{example}
Solve the inequalities $3x+1 \geq x-5 \geq 1+4x$.
\end{example}
\begin{proof}[Solution]

Solving $3x+1 \geq x-5$, we get $2x \geq -6$ and so $x \geq -3$.  Solving $x-5 \geq 1+4x$, we get $-3x \geq 6$ so $x \leq -2$.  It follows that
$x \geq -3$ and $x \leq -2$.  Thus, the solution lies in the interval $[-3, -2]$.
\end{proof}

We have to be careful when working with reciprocals:

\begin{itemize}
  \item If $0 < a < b$, then $\displaystyle \frac{1}{a} > \frac{1}{b}$.

  \item If $a < b < 0$, then $\displaystyle \frac{1}{a} > \frac{1}{b}$.
\end{itemize}

The above formulas do not work when one number is positive and the other is negative.  For example, $-2 < 4$, but $-1/2$ is not greater than $1/4$.
Some methods involving more complicated inequalities are reviewed in the next two examples.

\begin{example}
Solve the inequality $x^2 + 6x - 7 \geq 0$.
\end{example}
\begin{proof}[Solution]

Factoring the left side, we get $(x-1)(x+7) \geq 0$.  The solutions of the equation $(x-1)(x+7)=0$ are $x = 1$ and $x=-7$.  The numbers $-7$ and $1$ divide the number line into three intervals: $(-\infty, -7)$, $(-7, 1)$, and $(1, \infty)$.  Since the expression $x^2+6x-7 = (x-1)(x+7)$ can change its sign only at $-7$ and $1$, it follows that the sign on each of these intervals is constant.

We check the sign of each factor on each interval and record it in the table below.  For example, if $x$ is in $(-7, 1)$, then $x > -7$ and so $x+7 > 0$;
that is why we put the plus sign in the row corresponding to $x+7$ and in the column for the interval $(-7, 1)$.

$$
\begin{array}{c c c c}
	& (-\infty, -7) & (-7, 1) & (1, \infty) \\
  x-1 & - & - & + \\
  x+7 & - & + & + \\
  (x-1)(x+7) & + & - & + \\
\end{array}
$$

It follows that the solution consists of the intervals $(-\infty, -7]$ and $[1,\infty)$, since the value $0$ is allowed.  Our solution is then $(-\infty, -7] \cup [1,\infty)$.
Note that if the inequality were a strict inequality, i.e., $(x-1)(x+7) > 0$, then the solution would have been $(-infty, -7) \cup (1,\infty)$.

Alternatively, we can use test values for each interval.  The rationale is that the polynomial does not change its sign inside each interval.  So, if it is positive/negative
at one point in the interval, then it is positive/negative in the whole interval.  For example, to test the sign of $x+7$ in the interval $(-7,1)$, we can use any
number in $(-7,1)$; say, $x=-4$.  Since $x+7 = -4 + 7 = 3 > 0$, we conclude that $x+7$ is positive in the interval $(-7,1)$.

We can use test values to check the sign of the whole polynomial at once.  Here is how it works.  Test the interval $(-\infty, -7)$: take, for example, $x=-10$;
then $x^2+6x-7 = (x-1)(x+7) = (-11)(-3)= 33 > 0$.  It follows that $(-\infty, -7)$ is part of the solution.  Next, take $x=0$ to test the interval $(-7,1)$: it
follows that $x^2+6x-7 = (x-1)(x+7) = (-1)(7) = -7 < 0$; thus $(-7, 1)$ is not a part of the solution.  Analogously, we check the interval $(1,\infty)$.  In the
end, we include $-7$ and $1$, since they make the expression $x^2+6x-7$ equal to $0$, and the given inequality allows for that possibility to happen.
\end{proof}


\begin{example}
Solve the inequality $\displaystyle \frac{4}{3x-2} \leq 2$.
\end{example}

\begin{proof}[Solution]
It is convenient to have an inequality with one side zero, so we move $2$ to the left and compute the common denominator:
\begin{align*}
	\frac{4}{3x-2} & \leq 2\\
	\frac{4}{3x-2} - 2 & \leq 0\\
	\frac{4}{3x-2} - \frac{2(3x-2)}{3x-2} &\leq 0\\
	\frac{-6x+8}{3x-2} & \leq 0
\end{align*}

The sign of the fraction depends on the signs of both the numerator and the denominator.  From $-6x+8 = 0$, we get $x = -8/(-6) = 4/3$.  From $3x-2=0$, we get $x=2/3$.
Thus we have to check the intervals $(-\infty, 2/3)$, $(2/3, 4/3)$, and $(4/3, \infty)$.  On each interval, we use test values.

Testing the interval $(-\infty, 2/3)$: when $x=0$, $\frac{-6x+8}{3x-2} = \frac{8}{-2} = -4 < 0$.  Testing the interval $(2/3, 4/3)$: when $x=1$,
$\frac{-6x+8}{3x-2} = \frac{2}{1} = 2 > 0$.  Testing the interval $(4/3, \infty)$: when $x=2$, $\frac{-6x+8}{3x-2}=\frac{-4}{4} = -1 < 0$.
It follows that the solutions consists of $(-\infty, 2/3) \cup [4/3, \infty)$.  The reason why we included $x=4/3$ is that it makes the fraction equal to
zero (since its numerator is zero), and the given inequality allows for that possibility to occur.
\end{proof}

\section{Equations and Inequalities Involving Absolute Values}
To recall the definition of the absolute value, we consider the following example.
\begin{example}
Rewrite $|5x-2|$ without the absolute value signs.
\end{example}

\begin{proof}[Solution]
	By definition,
	\[ |5x-2| = \begin{cases} 5x-2 & \textrm{ if } 5x-2 \geq 0\\ -(5x-2) & \textrm{ if } 5x-2 < 0\end{cases}
		= \begin{cases} 5x-2 & \textrm{ if } x \geq 2/5 \\ -5x+2 & \textrm{ if } x < 2/5 \end{cases} \]
\end{proof}

In working with absolute values, the following statements might be helpful.
\begin{theorem}
  Let $A$ be any expression (e.g. $4$ or $x+2$), and let $a>0$. Then:
  \begin{itemize}
    \item $|A| = a$ if and only if $A = a$ or $A = -a$
    \item $|A| < a$ if and only if $-a < A < a$
    \item $|A| > a$ if and only if $A > a$ or $A < -a$
  \end{itemize}
\end{theorem}

A convenient way to visualize $|A|$ is to think of it as $|A| = |A - 0|$; i.e., to interpret it as the distance between $A$ and the origin.  Now $|A| < a$ means that
we need all numbers $A$ whose distance from the origin is smaller than $a$; thus, $-a < A < a$.  The inequality $|A| > a$ is interpreted similarly.

\begin{example}
Solve $|4x-3| = 2$.
\end{example}
\begin{proof}[Solution]
By definition,
\[ |4x-3| = \begin{cases} 4x-3 & \textrm{ if } 4x-3 \geq 0\\ -(4x-3) &\textrm{ if } 4x-3 < 0\end{cases}
		= \begin{cases} 4x-3 & \textrm{ if } x \geq 3/4\\ -(4x-3) &\textrm{ if } x < 3/4\end{cases} \]

Thus, the given equation breaks up into two equations.  If $x \geq 3/4$, it reads as $4x-3=2$ (and the solution is $x = 5/4$).  If $x < 3/4$, it reads
$-4x+3 = 2$ (and the solution is $x=1/4$).  Consequently, there are two solutions, $x=5/4$ and $x=1/4$.
\end{proof}

\begin{example}
Solve the following inequalities.
\begin{enumerate}
	\item $|2x+1| \leq 4$.
	\item $|3x-4| > 1$.
\end{enumerate}
\end{example}
\begin{proof}[Solution]
\begin{enumerate}
	\item{The given inequality is equivalent to
		\begin{align*}
			-4 &\leq 2x+1 \leq 4\\
			-5 &\leq 2x \leq 3\\
			-5/2 &\leq x \leq 3/2.
		\end{align*}
		Thus, the solution consists of the interval $[-5/2, 3/2]$.
	}
	\item{
		The given inequality is equivalent to $3x-4>1$ or $3x-4 < -1$.  Solving $3x-4 > 1$, we get $3x>5$ and $x > 5/3$.  Solving $3x-4 < -1$, we get $3x < 3$ and $x < 1$.
		Thus, the solution is $(-\infty, 1) \cup (5/3, \infty)$.
	}
\end{enumerate}
\end{proof}

\section{Systems of Equations}
In the two examples below, we review the most common methods of solving systems of two equations with two unknowns.

\begin{example}
	Solve the system $2x+y = 10$, $4x-y = 2$.
\end{example}
\begin{proof}[Solution]
We use the substitution method.  The idea is to eliminate one variable, so that we end up with one equation and one unknown.  Computing $y$ from $2x+y = 10$, we get
$y = 10-2x$.  Substituting $y$ into the second equation $4x-y=2$ we get:
\begin{align*}
	4x - y &= 2\\
	4x - (10 - 2x) &= 2\\
	6x &= 12\\
	x &= 2
\end{align*}
The corresponding value for $y$ is $y = 10 - 2x = 10 - 4 = 6$.   Expressing our solution as an ordered pair, we obtain $(2, 6)$.

Alternatively, adding up the two equations, we get
\[ (2x+y) + (4x - y) = 10 + 2. \]
Thus, $6x = 12$ and $x = 2$.  Using either of the two equations, we get $y = 6$.
\end{proof}

\begin{example}
	Solve the system $2x - y = -5$, $y = x^2 + 2$.
\end{example}
\begin{proof}[Solution]
From $2x-y = 5$, we get $y = 2x+5$.  Substituting $y$ into the second equation, we get
\begin{align*}
	y &= x^2 + 2\\
	2x+5 &= x^2 + 2\\
	x^2 - 2x - 3 &= 0\\
	(x-3)(x+1) &= 0.
\end{align*}
Consequently, there are two solutions for $x$, $x = -1$ and $x=3$.  When $x = -1$, $y = 2(-1)+5 = 3$.  When $x = 3$, $y = 2(3)+5 = 11$.
Thus there are two solutions $(-1,3)$, and $(3, 11)$.
\end{proof}


\end{document}
