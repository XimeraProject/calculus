\documentclass{ximera}

%\usepackage{todonotes}
%\usepackage{mathtools} %% Required for wide table Curl and Greens
%\usepackage{cuted} %% Required for wide table Curl and Greens
\newcommand{\todo}{}

\usepackage{multicol}

\usepackage{esint} % for \oiint
\ifxake%%https://math.meta.stackexchange.com/questions/9973/how-do-you-render-a-closed-surface-double-integral
\renewcommand{\oiint}{{\large\bigcirc}\kern-1.56em\iint}
\fi

\graphicspath{
  {./}
  {ximeraTutorial/}
  {basicPhilosophy/}
  {functionsOfSeveralVariables/}
  {normalVectors/}
  {lagrangeMultipliers/}
  {vectorFields/}
  {greensTheorem/}
  {shapeOfThingsToCome/}
  {dotProducts/}
  {partialDerivativesAndTheGradientVector/}
  {../ximeraTutorial/}
  {../productAndQuotientRules/exercises/}
  {../motionAndPathsInSpace/exercises/}
  {../normalVectors/exercisesParametricPlots/}
  {../continuityOfFunctionsOfSeveralVariables/exercises/}
  {../partialDerivativesAndTheGradientVector/exercises/}
  {../directionalDerivativeAndChainRule/exercises/}
  {../commonCoordinates/exercisesCylindricalCoordinates/}
  {../commonCoordinates/exercisesSphericalCoordinates/}
  {../greensTheorem/exercisesCurlAndLineIntegrals/}
  {../greensTheorem/exercisesDivergenceAndLineIntegrals/}
  {../shapeOfThingsToCome/exercisesDivergenceTheorem/}
  {../greensTheorem/}
  {../shapeOfThingsToCome/}
  {../separableDifferentialEquations/exercises/}
  {../dotproducts/}
  {../functionsOfSeveralVariables/}
  {../lagrangeMultipliers/}
  {../partialDerivativesAndTheGradientVector/}
  {../normalVectors/}
  {../vectorFields/}
}
\def\xmNotExpandableAsAccordion{true}

\newcommand{\mooculus}{\textsf{\textbf{MOOC}\textnormal{\textsf{ULUS}}}}

\usepackage{tkz-euclide}\usepackage{tikz}
\usepackage{tikz-cd}
\usetikzlibrary{arrows}
\tikzset{>=stealth,commutative diagrams/.cd,
  arrow style=tikz,diagrams={>=stealth}} %% cool arrow head
\tikzset{shorten <>/.style={ shorten >=#1, shorten <=#1 } } %% allows shorter vectors

\usetikzlibrary{backgrounds} %% for boxes around graphs
\usetikzlibrary{shapes,positioning}  %% Clouds and stars
\usetikzlibrary{matrix} %% for matrix
\usepgfplotslibrary{polar} %% for polar plots
\usepgfplotslibrary{fillbetween} %% to shade area between curves in TikZ
%\usetkzobj{all} %% obsolete

\usepackage[makeroom]{cancel} %% for strike outs
%\usepackage{mathtools} %% for pretty underbrace % Breaks Ximera
%\usepackage{multicol}
\usepackage{pgffor} %% required for integral for loops

\usepackage{tkz-tab}  %% for sign charts

%% http://tex.stackexchange.com/questions/66490/drawing-a-tikz-arc-specifying-the-center
%% Draws beach ball
\tikzset{pics/carc/.style args={#1:#2:#3}{code={\draw[pic actions] (#1:#3) arc(#1:#2:#3);}}}



\usepackage{array}
\setlength{\extrarowheight}{+.1cm}
\newdimen\digitwidth
\settowidth\digitwidth{9}
\def\divrule#1#2{
\noalign{\moveright#1\digitwidth
\vbox{\hrule width#2\digitwidth}}}





\newcommand{\RR}{\mathbb R}
\newcommand{\R}{\mathbb R}
\newcommand{\N}{\mathbb N}
\newcommand{\Z}{\mathbb Z}

\newcommand{\sagemath}{\textsf{SageMath}}


\renewcommand{\d}{\,d}
%\def\d{\mathop{}\!d}
%\def\d{\,d}

\AddToHook{begindocument}{%
  \renewcommand{\d}{\,d}     % lualatex redefines \d to underdot !!!
}

\pgfplotsset{
    every axis/.style={
        scale only axis,
        enlargelimits=false,
        trim axis left,
        trim axis right,
        clip=true,
    }
}

\newcommand{\dd}[2][]{\frac{\d #1}{\d #2}}
\newcommand{\pp}[2][]{\frac{\partial #1}{\partial #2}}
\renewcommand{\l}{\ell}
\newcommand{\ddx}{\frac{d}{\d x}}

\newcommand{\zeroOverZero}{\ensuremath{\boldsymbol{\tfrac{0}{0}}}}
\newcommand{\inftyOverInfty}{\ensuremath{\boldsymbol{\tfrac{\infty}{\infty}}}}
\newcommand{\zeroOverInfty}{\ensuremath{\boldsymbol{\tfrac{0}{\infty}}}}
\newcommand{\zeroTimesInfty}{\ensuremath{\small\boldsymbol{0\cdot \infty}}}
\newcommand{\inftyMinusInfty}{\ensuremath{\small\boldsymbol{\infty - \infty}}}
\newcommand{\oneToInfty}{\ensuremath{\boldsymbol{1^\infty}}}
\newcommand{\zeroToZero}{\ensuremath{\boldsymbol{0^0}}}
\newcommand{\inftyToZero}{\ensuremath{\boldsymbol{\infty^0}}}



\newcommand{\numOverZero}{\ensuremath{\boldsymbol{\tfrac{\#}{0}}}}
\newcommand{\dfn}{\textbf}
%\newcommand{\unit}{\,\mathrm}
\newcommand{\unit}{\mathop{}\!\mathrm}
\newcommand{\eval}[1]{\bigg[ #1 \bigg]}
\newcommand{\seq}[1]{\left( #1 \right)}
\renewcommand{\epsilon}{\varepsilon}
\renewcommand{\phi}{\varphi}


\renewcommand{\iff}{\Leftrightarrow}

\DeclareMathOperator{\arccot}{arccot}
\DeclareMathOperator{\arcsec}{arcsec}
\DeclareMathOperator{\arccsc}{arccsc}
\DeclareMathOperator{\si}{Si}
\DeclareMathOperator{\scal}{scal}
\DeclareMathOperator{\sign}{sign}


%% \newcommand{\tightoverset}[2]{% for arrow vec
%%   \mathop{#2}\limits^{\vbox to -.5ex{\kern-0.75ex\hbox{$#1$}\vss}}}
\newcommand{\arrowvec}[1]{{\overset{\rightharpoonup}{#1}}}
%\renewcommand{\vec}[1]{\arrowvec{\mathbf{#1}}}
\renewcommand{\vec}[1]{{\overset{\boldsymbol{\rightharpoonup}}{\mathbf{#1}}}\hspace{0in}}

\newcommand{\point}[1]{\left(#1\right)} %this allows \vector{ to be changed to \vector{ with a quick find and replace
\newcommand{\pt}[1]{\mathbf{#1}} %this allows \vec{ to be changed to \vec{ with a quick find and replace
\newcommand{\Lim}[2]{\lim_{\point{#1} \to \point{#2}}} %Bart, I changed this to point since I want to use it.  It runs through both of the exercise and exerciseE files in limits section, which is why it was in each document to start with.

\DeclareMathOperator{\proj}{\mathbf{proj}}
\newcommand{\veci}{{\boldsymbol{\hat{\imath}}}}
\newcommand{\vecj}{{\boldsymbol{\hat{\jmath}}}}
\newcommand{\veck}{{\boldsymbol{\hat{k}}}}
\newcommand{\vecl}{\vec{\boldsymbol{\l}}}
\newcommand{\uvec}[1]{\mathbf{\hat{#1}}}
\newcommand{\utan}{\mathbf{\hat{t}}}
\newcommand{\unormal}{\mathbf{\hat{n}}}
\newcommand{\ubinormal}{\mathbf{\hat{b}}}

\newcommand{\dotp}{\bullet}
\newcommand{\cross}{\boldsymbol\times}
\newcommand{\grad}{\boldsymbol\nabla}
\newcommand{\divergence}{\grad\dotp}
\newcommand{\curl}{\grad\cross}
%\DeclareMathOperator{\divergence}{divergence}
%\DeclareMathOperator{\curl}[1]{\grad\cross #1}
\newcommand{\lto}{\mathop{\longrightarrow\,}\limits}

\renewcommand{\bar}{\overline}

\colorlet{textColor}{black}
\colorlet{background}{white}
\colorlet{penColor}{blue!50!black} % Color of a curve in a plot
\colorlet{penColor2}{red!50!black}% Color of a curve in a plot
\colorlet{penColor3}{red!50!blue} % Color of a curve in a plot
\colorlet{penColor4}{green!50!black} % Color of a curve in a plot
\colorlet{penColor5}{orange!80!black} % Color of a curve in a plot
\colorlet{penColor6}{yellow!70!black} % Color of a curve in a plot
\colorlet{fill1}{penColor!20} % Color of fill in a plot
\colorlet{fill2}{penColor2!20} % Color of fill in a plot
\colorlet{fillp}{fill1} % Color of positive area
\colorlet{filln}{penColor2!20} % Color of negative area
\colorlet{fill3}{penColor3!20} % Fill
\colorlet{fill4}{penColor4!20} % Fill
\colorlet{fill5}{penColor5!20} % Fill
\colorlet{gridColor}{gray!50} % Color of grid in a plot

\newcommand{\surfaceColor}{violet}
\newcommand{\surfaceColorTwo}{redyellow}
\newcommand{\sliceColor}{greenyellow}




\pgfmathdeclarefunction{gauss}{2}{% gives gaussian
  \pgfmathparse{1/(#2*sqrt(2*pi))*exp(-((x-#1)^2)/(2*#2^2))}%
}


%%%%%%%%%%%%%
%% Vectors
%%%%%%%%%%%%%

%% Simple horiz vectors
\renewcommand{\vector}[1]{\left\langle #1\right\rangle}


%% %% Complex Horiz Vectors with angle brackets
%% \makeatletter
%% \renewcommand{\vector}[2][ , ]{\left\langle%
%%   \def\nextitem{\def\nextitem{#1}}%
%%   \@for \el:=#2\do{\nextitem\el}\right\rangle%
%% }
%% \makeatother

%% %% Vertical Vectors
%% \def\vector#1{\begin{bmatrix}\vecListA#1,,\end{bmatrix}}
%% \def\vecListA#1,{\if,#1,\else #1\cr \expandafter \vecListA \fi}

%%%%%%%%%%%%%
%% End of vectors
%%%%%%%%%%%%%

%\newcommand{\fullwidth}{}
%\newcommand{\normalwidth}{}



%% makes a snazzy t-chart for evaluating functions
%\newenvironment{tchart}{\rowcolors{2}{}{background!90!textColor}\array}{\endarray}

%%This is to help with formatting on future title pages.
\newenvironment{sectionOutcomes}{}{}



%% Flowchart stuff
%\tikzstyle{startstop} = [rectangle, rounded corners, minimum width=3cm, minimum height=1cm,text centered, draw=black]
%\tikzstyle{question} = [rectangle, minimum width=3cm, minimum height=1cm, text centered, draw=black]
%\tikzstyle{decision} = [trapezium, trapezium left angle=70, trapezium right angle=110, minimum width=3cm, minimum height=1cm, text centered, draw=black]
%\tikzstyle{question} = [rectangle, rounded corners, minimum width=3cm, minimum height=1cm,text centered, draw=black]
%\tikzstyle{process} = [rectangle, minimum width=3cm, minimum height=1cm, text centered, draw=black]
%\tikzstyle{decision} = [trapezium, trapezium left angle=70, trapezium right angle=110, minimum width=3cm, minimum height=1cm, text centered, draw=black]


\author{Jim Talamo}
\license{Creative Commons 3.0 By-bC}


\outcome{}


\begin{document}
\begin{exercise}
There are many different tests that can be used to test whether a given series converges or diverges.  As a result, it is important to develop a good sense for quickly narrowing down which tests make sense to apply to a given series just by inspecting the given series.  A fundamentally important fact allows us to gain some insight into the Ratio Test:

\begin{fact}
Let $p(x)$ be a polynomial.  Then:

\[
\lim_{n \to \infty} \frac{p(n+1)}{p(n)} = 1.
\]
\end{fact}

\begin{exercise}
To gain a little practice, suppose that $p(x) = x^5$.  Then the limit to evaluate is:

\[
\lim_{n \to \infty} \frac{p(n+1)}{p(n)} = \answer{\frac{(n+1)^5}{n^5}}
\]
The numerator is a polynomial of degree $\answer{5}$, and the coefficient of the $n^5$ term is $\answer{1}$.

The denominator is a polynomial of degree $\answer{5}$, and the coefficient of the $n^5$ term is $\answer{1}$.

Hence, the limit is $\answer{1}$.

\begin{exercise}
To explore this a little more, now set $p(x) = x^5+4x^3+7$.  Then:
\[
\lim_{n \to \infty} \frac{p(n+1)}{p(n)} = \answer{\frac{(n+1)^5+4(n+1)^3+7}{n^5+4n^3+7}}
\]

Note that $(n+1)^5$ is a polynomial of degree $\answer{5}$, $(n+1)^3$ is a polynomial of degree $\answer{3}$, so without expanding any of these powers out, the numerator is a polynomial of degree $\answer{5}$, and the coefficient of the $n^5$ term is $\answer{1}$.

The denominator is a polynomial of degree $\answer{5}$, and the coefficient of the $n^5$ term is $\answer{1}$.

Hence, the limit is $\answer{1}$.

\end{exercise}


\begin{exercise}
One of the major consequences of this is that multiplying or dividing a given \emph{sequence} by a polynomial will not affect the limit required in the Ratio Test.  Indeed, consider the series $\sum_{k=1}^{\infty} \frac{1}{2^k}$.  Note that this is a geometric series, but we could establish its convergence using the Ratio Test.  Indeed, here:

\[
L = \lim_{n \to \infty} \frac{a_{n+1}}{a_n} = \lim_{n \to \infty} \frac{1}{2^{n+1}}\cdot 2^n = \answer{\frac{1}{2}}
\]

Now, let's multiply the original sequence defined by the rule $a_n = \left(\frac{1}{2}\right)^n$ by the polynomial $p(n) = n^5$ and try to sum its terms.  That is, we want to consider the series $\sum_{k=1}^{\infty} \frac{k^5}{2^k}$. We find:

\[
L =  \lim_{n \to \infty} \frac{(n+1)^5}{2^{n+1}} \cdot \frac{2^n}{n^5} =  \lim_{n \to \infty} \frac{2^n}{2^{n+1}} \cdot \frac{(n+1)^5}{n^5} =  \answer{\frac{1}{2}} \cdot \answer{1}
\]

Introducing the $n^5$ term into the numerator did not affect the value of $L$.

Indeed, for the series $\sum_{k=1}^{\infty} \frac{1}{2^k}$, we found that $L=\answer{\frac{1}{2}}$ and for the series $\sum_{k=1}^{\infty} \frac{k^5}{2^k}$, we found $L=\answer{\frac{1}{2}}$. 

\begin{exercise}
One of the major consequences of this now is that a series must have a term that grows at least exponentially in order for the Ratio Test to have a chance to be conclusive!  More explicitly, let $p, q > 0$, $a>1$, and consider the growth rates results for \emph{sequences}:

\begin{image}
  \begin{tikzpicture}
        \node at (0,0) {
          $(\ln n)^p << n^q << \underbrace{a^n << n! << n^n}$
        };
        \node at (2.5,-.5) {\small{A series should have a term that}};
         \node at (2.6,-.95) {\small{grows at least as quickly as one of}};
         \node at (2.7,-1.4) {\small{these if you want to use Ratio Test! }};
      \end{tikzpicture}
  \end{image}

Without performing any calculations, determine which of the following series would the Ratio Test would be conclusive.  That is, which of the following would either converge or diverge as a consequence of the Ratio Test?
\begin{selectAll}
\choice{$\sum_{k=1}^{\infty} \frac{k^3+2k-1}{k^2+2}$}
\choice[correct]{$\sum_{k=1}^{\infty} \frac{\ln(k) +k^2}{k!}$}
\choice[correct]{$\sum_{k=1}^{\infty} \frac{k^{10}}{5^k}$}
\choice{$\sum_{k=1}^{\infty} \frac{(\ln k)^{80}}{k^5}$}
\choice{$\sum_{k=1}^{\infty} k^{2018}$}
\choice[correct]{$\sum_{k=1}^{\infty} \frac{2^k}{k^k}$}
\end{selectAll}

\end{exercise}
\end{exercise}
\end{exercise}
\end{exercise}
\end{document}
