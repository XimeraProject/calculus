\documentclass[noauthor,nooutcomes]{ximera}

%\usepackage{todonotes}
%\usepackage{mathtools} %% Required for wide table Curl and Greens
%\usepackage{cuted} %% Required for wide table Curl and Greens
\newcommand{\todo}{}

\usepackage{multicol}

\usepackage{esint} % for \oiint
\ifxake%%https://math.meta.stackexchange.com/questions/9973/how-do-you-render-a-closed-surface-double-integral
\renewcommand{\oiint}{{\large\bigcirc}\kern-1.56em\iint}
\fi

\graphicspath{
  {./}
  {ximeraTutorial/}
  {basicPhilosophy/}
  {functionsOfSeveralVariables/}
  {normalVectors/}
  {lagrangeMultipliers/}
  {vectorFields/}
  {greensTheorem/}
  {shapeOfThingsToCome/}
  {dotProducts/}
  {partialDerivativesAndTheGradientVector/}
  {../ximeraTutorial/}
  {../productAndQuotientRules/exercises/}
  {../motionAndPathsInSpace/exercises/}
  {../normalVectors/exercisesParametricPlots/}
  {../continuityOfFunctionsOfSeveralVariables/exercises/}
  {../partialDerivativesAndTheGradientVector/exercises/}
  {../directionalDerivativeAndChainRule/exercises/}
  {../commonCoordinates/exercisesCylindricalCoordinates/}
  {../commonCoordinates/exercisesSphericalCoordinates/}
  {../greensTheorem/exercisesCurlAndLineIntegrals/}
  {../greensTheorem/exercisesDivergenceAndLineIntegrals/}
  {../shapeOfThingsToCome/exercisesDivergenceTheorem/}
  {../greensTheorem/}
  {../shapeOfThingsToCome/}
  {../separableDifferentialEquations/exercises/}
  {../dotproducts/}
  {../functionsOfSeveralVariables/}
  {../lagrangeMultipliers/}
  {../partialDerivativesAndTheGradientVector/}
  {../normalVectors/}
  {../vectorFields/}
}
\def\xmNotExpandableAsAccordion{true}

\newcommand{\mooculus}{\textsf{\textbf{MOOC}\textnormal{\textsf{ULUS}}}}

\usepackage{tkz-euclide}\usepackage{tikz}
\usepackage{tikz-cd}
\usetikzlibrary{arrows}
\tikzset{>=stealth,commutative diagrams/.cd,
  arrow style=tikz,diagrams={>=stealth}} %% cool arrow head
\tikzset{shorten <>/.style={ shorten >=#1, shorten <=#1 } } %% allows shorter vectors

\usetikzlibrary{backgrounds} %% for boxes around graphs
\usetikzlibrary{shapes,positioning}  %% Clouds and stars
\usetikzlibrary{matrix} %% for matrix
\usepgfplotslibrary{polar} %% for polar plots
\usepgfplotslibrary{fillbetween} %% to shade area between curves in TikZ
%\usetkzobj{all} %% obsolete

\usepackage[makeroom]{cancel} %% for strike outs
%\usepackage{mathtools} %% for pretty underbrace % Breaks Ximera
%\usepackage{multicol}
\usepackage{pgffor} %% required for integral for loops

\usepackage{tkz-tab}  %% for sign charts

%% http://tex.stackexchange.com/questions/66490/drawing-a-tikz-arc-specifying-the-center
%% Draws beach ball
\tikzset{pics/carc/.style args={#1:#2:#3}{code={\draw[pic actions] (#1:#3) arc(#1:#2:#3);}}}



\usepackage{array}
\setlength{\extrarowheight}{+.1cm}
\newdimen\digitwidth
\settowidth\digitwidth{9}
\def\divrule#1#2{
\noalign{\moveright#1\digitwidth
\vbox{\hrule width#2\digitwidth}}}





\newcommand{\RR}{\mathbb R}
\newcommand{\R}{\mathbb R}
\newcommand{\N}{\mathbb N}
\newcommand{\Z}{\mathbb Z}

\newcommand{\sagemath}{\textsf{SageMath}}


\renewcommand{\d}{\,d}
%\def\d{\mathop{}\!d}
%\def\d{\,d}

\AddToHook{begindocument}{%
  \renewcommand{\d}{\,d}     % lualatex redefines \d to underdot !!!
}

\pgfplotsset{
    every axis/.style={
        scale only axis,
        enlargelimits=false,
        trim axis left,
        trim axis right,
        clip=true,
    }
}

\newcommand{\dd}[2][]{\frac{\d #1}{\d #2}}
\newcommand{\pp}[2][]{\frac{\partial #1}{\partial #2}}
\renewcommand{\l}{\ell}
\newcommand{\ddx}{\frac{d}{\d x}}

\newcommand{\zeroOverZero}{\ensuremath{\boldsymbol{\tfrac{0}{0}}}}
\newcommand{\inftyOverInfty}{\ensuremath{\boldsymbol{\tfrac{\infty}{\infty}}}}
\newcommand{\zeroOverInfty}{\ensuremath{\boldsymbol{\tfrac{0}{\infty}}}}
\newcommand{\zeroTimesInfty}{\ensuremath{\small\boldsymbol{0\cdot \infty}}}
\newcommand{\inftyMinusInfty}{\ensuremath{\small\boldsymbol{\infty - \infty}}}
\newcommand{\oneToInfty}{\ensuremath{\boldsymbol{1^\infty}}}
\newcommand{\zeroToZero}{\ensuremath{\boldsymbol{0^0}}}
\newcommand{\inftyToZero}{\ensuremath{\boldsymbol{\infty^0}}}



\newcommand{\numOverZero}{\ensuremath{\boldsymbol{\tfrac{\#}{0}}}}
\newcommand{\dfn}{\textbf}
%\newcommand{\unit}{\,\mathrm}
\newcommand{\unit}{\mathop{}\!\mathrm}
\newcommand{\eval}[1]{\bigg[ #1 \bigg]}
\newcommand{\seq}[1]{\left( #1 \right)}
\renewcommand{\epsilon}{\varepsilon}
\renewcommand{\phi}{\varphi}


\renewcommand{\iff}{\Leftrightarrow}

\DeclareMathOperator{\arccot}{arccot}
\DeclareMathOperator{\arcsec}{arcsec}
\DeclareMathOperator{\arccsc}{arccsc}
\DeclareMathOperator{\si}{Si}
\DeclareMathOperator{\scal}{scal}
\DeclareMathOperator{\sign}{sign}


%% \newcommand{\tightoverset}[2]{% for arrow vec
%%   \mathop{#2}\limits^{\vbox to -.5ex{\kern-0.75ex\hbox{$#1$}\vss}}}
\newcommand{\arrowvec}[1]{{\overset{\rightharpoonup}{#1}}}
%\renewcommand{\vec}[1]{\arrowvec{\mathbf{#1}}}
\renewcommand{\vec}[1]{{\overset{\boldsymbol{\rightharpoonup}}{\mathbf{#1}}}\hspace{0in}}

\newcommand{\point}[1]{\left(#1\right)} %this allows \vector{ to be changed to \vector{ with a quick find and replace
\newcommand{\pt}[1]{\mathbf{#1}} %this allows \vec{ to be changed to \vec{ with a quick find and replace
\newcommand{\Lim}[2]{\lim_{\point{#1} \to \point{#2}}} %Bart, I changed this to point since I want to use it.  It runs through both of the exercise and exerciseE files in limits section, which is why it was in each document to start with.

\DeclareMathOperator{\proj}{\mathbf{proj}}
\newcommand{\veci}{{\boldsymbol{\hat{\imath}}}}
\newcommand{\vecj}{{\boldsymbol{\hat{\jmath}}}}
\newcommand{\veck}{{\boldsymbol{\hat{k}}}}
\newcommand{\vecl}{\vec{\boldsymbol{\l}}}
\newcommand{\uvec}[1]{\mathbf{\hat{#1}}}
\newcommand{\utan}{\mathbf{\hat{t}}}
\newcommand{\unormal}{\mathbf{\hat{n}}}
\newcommand{\ubinormal}{\mathbf{\hat{b}}}

\newcommand{\dotp}{\bullet}
\newcommand{\cross}{\boldsymbol\times}
\newcommand{\grad}{\boldsymbol\nabla}
\newcommand{\divergence}{\grad\dotp}
\newcommand{\curl}{\grad\cross}
%\DeclareMathOperator{\divergence}{divergence}
%\DeclareMathOperator{\curl}[1]{\grad\cross #1}
\newcommand{\lto}{\mathop{\longrightarrow\,}\limits}

\renewcommand{\bar}{\overline}

\colorlet{textColor}{black}
\colorlet{background}{white}
\colorlet{penColor}{blue!50!black} % Color of a curve in a plot
\colorlet{penColor2}{red!50!black}% Color of a curve in a plot
\colorlet{penColor3}{red!50!blue} % Color of a curve in a plot
\colorlet{penColor4}{green!50!black} % Color of a curve in a plot
\colorlet{penColor5}{orange!80!black} % Color of a curve in a plot
\colorlet{penColor6}{yellow!70!black} % Color of a curve in a plot
\colorlet{fill1}{penColor!20} % Color of fill in a plot
\colorlet{fill2}{penColor2!20} % Color of fill in a plot
\colorlet{fillp}{fill1} % Color of positive area
\colorlet{filln}{penColor2!20} % Color of negative area
\colorlet{fill3}{penColor3!20} % Fill
\colorlet{fill4}{penColor4!20} % Fill
\colorlet{fill5}{penColor5!20} % Fill
\colorlet{gridColor}{gray!50} % Color of grid in a plot

\newcommand{\surfaceColor}{violet}
\newcommand{\surfaceColorTwo}{redyellow}
\newcommand{\sliceColor}{greenyellow}




\pgfmathdeclarefunction{gauss}{2}{% gives gaussian
  \pgfmathparse{1/(#2*sqrt(2*pi))*exp(-((x-#1)^2)/(2*#2^2))}%
}


%%%%%%%%%%%%%
%% Vectors
%%%%%%%%%%%%%

%% Simple horiz vectors
\renewcommand{\vector}[1]{\left\langle #1\right\rangle}


%% %% Complex Horiz Vectors with angle brackets
%% \makeatletter
%% \renewcommand{\vector}[2][ , ]{\left\langle%
%%   \def\nextitem{\def\nextitem{#1}}%
%%   \@for \el:=#2\do{\nextitem\el}\right\rangle%
%% }
%% \makeatother

%% %% Vertical Vectors
%% \def\vector#1{\begin{bmatrix}\vecListA#1,,\end{bmatrix}}
%% \def\vecListA#1,{\if,#1,\else #1\cr \expandafter \vecListA \fi}

%%%%%%%%%%%%%
%% End of vectors
%%%%%%%%%%%%%

%\newcommand{\fullwidth}{}
%\newcommand{\normalwidth}{}



%% makes a snazzy t-chart for evaluating functions
%\newenvironment{tchart}{\rowcolors{2}{}{background!90!textColor}\array}{\endarray}

%%This is to help with formatting on future title pages.
\newenvironment{sectionOutcomes}{}{}



%% Flowchart stuff
%\tikzstyle{startstop} = [rectangle, rounded corners, minimum width=3cm, minimum height=1cm,text centered, draw=black]
%\tikzstyle{question} = [rectangle, minimum width=3cm, minimum height=1cm, text centered, draw=black]
%\tikzstyle{decision} = [trapezium, trapezium left angle=70, trapezium right angle=110, minimum width=3cm, minimum height=1cm, text centered, draw=black]
%\tikzstyle{question} = [rectangle, rounded corners, minimum width=3cm, minimum height=1cm,text centered, draw=black]
%\tikzstyle{process} = [rectangle, minimum width=3cm, minimum height=1cm, text centered, draw=black]
%\tikzstyle{decision} = [trapezium, trapezium left angle=70, trapezium right angle=110, minimum width=3cm, minimum height=1cm, text centered, draw=black]


\author{Bart Snapp}

\title[Collaborate:]{Breaking math}

\begin{document}
\begin{abstract}
  Two calculus students attempt to ``break'' math.
\end{abstract}
\maketitle

\textbf{Work in groups of 3--4, writing your answers on a separate
  sheet of paper.}


Check out this dialogue between two calculus students (based on a true
story):

\begin{dialogue}
\item[Devyn] Riley, I have something very important to say.
\item[Riley] Yeah?  Hit me with it.
\item[Devyn] I think I just broke math.
\item[Riley] I've suspected for ages that all this calculus stuff was razzmatazz. Lay it on me.
\item[Devyn] Consider the vector field:
  \[
  \vector{\frac{-y}{x^2+y^2},\frac{x}{x^2+y^2}}
  \]
\item[Riley] Got it. It looks like a whirlpool.
\item[Devyn] I know! Now compute its curl.
\item[Riley] OK--I get zero curl.
\item[Devyn] I know! Now consider Green's Theorem.
\item[Riley] You mean:
  \[
  \iint_R \curl \vec{F} \d A = \oint_{C} \vec{F}\dotp\d\vec{p}
  \]
  What's $R$? What's $C$?
\item[Devyn] Let $R$ be the unit disk centered at the origin.
\item[Riley] OK, so $C$ is the unit circle centered at the origin.
\item[Devyn] Right. Now here's the deal\dots The left-hand side of the
  equation is zero because the curl of our vector field is zero.
\item[Riley] Oh. And the right-hand side of the equation cannot be
  zero because our vector field looks like a whirlpool.
\item[Devyn] And now we have zero equals something not zero, and kablamy, math is in ruins.
\end{dialogue}


\begin{problem}
  What just happened? Explain why Devin and Riley think math is
  broken. Try to take it step-by-step.
\end{problem}

Let's investigate further. To really understand this, we're going to
have to check each of their claims.

\begin{problem}
  Consider the vector field
  \[
  \vector{\frac{-y}{x^2+y^2},\frac{x}{x^2+y^2}}
  \]
  Plot some vectors on the grid below. Focus on getting the directions
  of the vectors correct, and don't worry to much about the
  magnitudes.
  \begin{image}
    \begin{tikzpicture}
      \begin{axis}%
        [
	  ymin=-4,ymax=4,
	  xmin=-4,xmax=4,
          axis lines =middle, xlabel=$x$, ylabel=$y$,
          every axis y label/.style={at=(current axis.above origin),anchor=south},
          every axis x label/.style={at=(current axis.right of origin),anchor=west},
          grid=both,
          grid style={dashed, gridColor},
          xtick={-6,...,6},
          ytick={-6,...,6},
	]
      \end{axis}
     \end{tikzpicture}
  \end{image}
  Does it look like a ``whirlpool?''
\end{problem}

\begin{problem}
  Letting $\vec{F}(x,y) = \vector{\frac{-y}{x^2+y^2},\frac{x}{x^2+y^2}}$,
  explain why someone might believe that
  \[
  \oint_{C} \vec{F}\dotp\d\vec{p}
  \]
  is nonzero. No computation are necessary at this point.
\end{problem}


\begin{problem}
  Parametrize the unit circle $U$ centered at the origin and compute:
  \[
  \oint_{U} \vec{F}\dotp\d\vec{p}
  \]
\end{problem}

\begin{problem}
  Parametrize a circle $C$ of radius $r$ centered at the origin and
  compute:
  \[
  \oint_{C} \vec{F}\dotp\d\vec{p}
  \]
\end{problem}

\begin{problem}
    As a gesture of friendship, we reveal that:
  \begin{align*}
    \grad \arctan(y/x) &= \vector{\frac{-y}{x^2+y^2},\frac{x}{x^2+y^2}}\\
    \grad \arctan(-x/y) &= \vector{\frac{-y}{x^2+y^2},\frac{x}{x^2+y^2}}
  \end{align*}
  Confirm these equations.
\end{problem}

\begin{problem}
  Use the Fundamental Theorem of Line Integrals to compute
  \[
  \oint_S \frac{-y}{x^2+y^2}\d x + \frac{x}{x^2+y^2}\d y
  \]
  where $S$ is the square with vertices $(1,1)$, $(-1,1)$, $(-1,-1)$,
  and $(1,-1)$ drawn in a counterclockwise fashion.
  \begin{hint}
    For the fundamental theorem to apply, the chosen path must be in
    the domain of the potential function.
  \end{hint}
\end{problem}

\begin{problem}
  Use the Fundamental Theorem of Line Integrals to compute
  \[
  \oint_Q \frac{-y}{x^2+y^2}\d x + \frac{x}{x^2+y^2}\d y
  \]
  where $Q$ is the square with vertices $(a,a)$, $(-a,a)$, $(-a,-a)$,
  and $(a,-a)$ drawn in a counterclockwise fashion, where $a>0$.
  \begin{hint}
    For the fundamental theorem to apply, the chosen path must be in
    the domain of the potential function.
  \end{hint}
\end{problem}

\begin{problem}
  Use the Fundamental Theorem of Line Integrals to compute
  \[
  \oint_Y \frac{-y}{x^2+y^2}\d x + \frac{x}{x^2+y^2}\d y
  \]
  where $Y$ is a polygonal path you choose for yourself that contains
  the point $(0,0)$ in the \textbf{interior}. A square of some sort
  would probably be easiest.
  \begin{hint}
    For the fundamental theorem to apply, the chosen path must be in
    the domain of the potential function.
  \end{hint}
\end{problem}

At this point, you might suspect that something strange is going
on\dots

\begin{problem}
  Again letting $\vec{F}(x,y) =
  \vector{\frac{-y}{x^2+y^2},\frac{x}{x^2+y^2}}$, compute:
  \[
  \curl \vec{F}
  \]
  Is $\curl \vec{F}$ zero?
  \begin{hint}
    It is not \textbf{always} zero.
  \end{hint}
\end{problem}

\begin{problem}
  What would you say to Devin and Riley to assure them that
  mathematics is not ``broken?''
\end{problem}


\newpage


\section{The take-away}

Here we presented you with a field where the (scalar) curl was zero
everywhere except at the origin.  At the origin the (scalar) curl was
undefined; hence, Green's Theorem does not apply.

What is remarkable is that in this case, where the (scalar) curl is
zero except for a point, any path $C$ around the point where the field
is undefined will yield the same value for:
\[
\oint_C \vec{F}\dotp\d\vec{p}
\]



\end{document}
