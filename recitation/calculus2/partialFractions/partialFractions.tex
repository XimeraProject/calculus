\documentclass[noauthor, handout]{ximera}
%handout:  for handout version with no solutions or instructor notes
%handout,instructornotes:  for instructor version with just problems and notes, no solutions
%noinstructornotes:  shows only problem and solutions

%% handout
%% space
%% newpage
%% numbers
%% nooutcomes

%I added the commands here so that I would't have to keep looking them up
%\newcommand{\RR}{\mathbb R}
%\renewcommand{\d}{\,d}
%\newcommand{\dd}[2][]{\frac{d #1}{d #2}}
%\renewcommand{\l}{\ell}
%\newcommand{\ddx}{\frac{d}{dx}}
%\everymath{\displaystyle}
%\newcommand{\dfn}{\textbf}
%\newcommand{\eval}[1]{\bigg[ #1 \bigg]}

%\begin{image}
%\includegraphics[trim= 170 420 250 180]{Figure1.pdf}
%\end{image}

%add a ``.'' below when used in a specific directory.

%\usepackage{todonotes}
%\usepackage{mathtools} %% Required for wide table Curl and Greens
%\usepackage{cuted} %% Required for wide table Curl and Greens
\newcommand{\todo}{}

\usepackage{multicol}

\usepackage{esint} % for \oiint
\ifxake%%https://math.meta.stackexchange.com/questions/9973/how-do-you-render-a-closed-surface-double-integral
\renewcommand{\oiint}{{\large\bigcirc}\kern-1.56em\iint}
\fi

\graphicspath{
  {./}
  {ximeraTutorial/}
  {basicPhilosophy/}
  {functionsOfSeveralVariables/}
  {normalVectors/}
  {lagrangeMultipliers/}
  {vectorFields/}
  {greensTheorem/}
  {shapeOfThingsToCome/}
  {dotProducts/}
  {partialDerivativesAndTheGradientVector/}
  {../ximeraTutorial/}
  {../productAndQuotientRules/exercises/}
  {../motionAndPathsInSpace/exercises/}
  {../normalVectors/exercisesParametricPlots/}
  {../continuityOfFunctionsOfSeveralVariables/exercises/}
  {../partialDerivativesAndTheGradientVector/exercises/}
  {../directionalDerivativeAndChainRule/exercises/}
  {../commonCoordinates/exercisesCylindricalCoordinates/}
  {../commonCoordinates/exercisesSphericalCoordinates/}
  {../greensTheorem/exercisesCurlAndLineIntegrals/}
  {../greensTheorem/exercisesDivergenceAndLineIntegrals/}
  {../shapeOfThingsToCome/exercisesDivergenceTheorem/}
  {../greensTheorem/}
  {../shapeOfThingsToCome/}
  {../separableDifferentialEquations/exercises/}
  {../dotproducts/}
  {../functionsOfSeveralVariables/}
  {../lagrangeMultipliers/}
  {../partialDerivativesAndTheGradientVector/}
  {../normalVectors/}
  {../vectorFields/}
}
\def\xmNotExpandableAsAccordion{true}

\newcommand{\mooculus}{\textsf{\textbf{MOOC}\textnormal{\textsf{ULUS}}}}

\usepackage{tkz-euclide}\usepackage{tikz}
\usepackage{tikz-cd}
\usetikzlibrary{arrows}
\tikzset{>=stealth,commutative diagrams/.cd,
  arrow style=tikz,diagrams={>=stealth}} %% cool arrow head
\tikzset{shorten <>/.style={ shorten >=#1, shorten <=#1 } } %% allows shorter vectors

\usetikzlibrary{backgrounds} %% for boxes around graphs
\usetikzlibrary{shapes,positioning}  %% Clouds and stars
\usetikzlibrary{matrix} %% for matrix
\usepgfplotslibrary{polar} %% for polar plots
\usepgfplotslibrary{fillbetween} %% to shade area between curves in TikZ
%\usetkzobj{all} %% obsolete

\usepackage[makeroom]{cancel} %% for strike outs
%\usepackage{mathtools} %% for pretty underbrace % Breaks Ximera
%\usepackage{multicol}
\usepackage{pgffor} %% required for integral for loops

\usepackage{tkz-tab}  %% for sign charts

%% http://tex.stackexchange.com/questions/66490/drawing-a-tikz-arc-specifying-the-center
%% Draws beach ball
\tikzset{pics/carc/.style args={#1:#2:#3}{code={\draw[pic actions] (#1:#3) arc(#1:#2:#3);}}}



\usepackage{array}
\setlength{\extrarowheight}{+.1cm}
\newdimen\digitwidth
\settowidth\digitwidth{9}
\def\divrule#1#2{
\noalign{\moveright#1\digitwidth
\vbox{\hrule width#2\digitwidth}}}





\newcommand{\RR}{\mathbb R}
\newcommand{\R}{\mathbb R}
\newcommand{\N}{\mathbb N}
\newcommand{\Z}{\mathbb Z}

\newcommand{\sagemath}{\textsf{SageMath}}


\renewcommand{\d}{\,d}
%\def\d{\mathop{}\!d}
%\def\d{\,d}

\AddToHook{begindocument}{%
  \renewcommand{\d}{\,d}     % lualatex redefines \d to underdot !!!
}

\pgfplotsset{
    every axis/.style={
        scale only axis,
        enlargelimits=false,
        trim axis left,
        trim axis right,
        clip=true,
    }
}

\newcommand{\dd}[2][]{\frac{\d #1}{\d #2}}
\newcommand{\pp}[2][]{\frac{\partial #1}{\partial #2}}
\renewcommand{\l}{\ell}
\newcommand{\ddx}{\frac{d}{\d x}}

\newcommand{\zeroOverZero}{\ensuremath{\boldsymbol{\tfrac{0}{0}}}}
\newcommand{\inftyOverInfty}{\ensuremath{\boldsymbol{\tfrac{\infty}{\infty}}}}
\newcommand{\zeroOverInfty}{\ensuremath{\boldsymbol{\tfrac{0}{\infty}}}}
\newcommand{\zeroTimesInfty}{\ensuremath{\small\boldsymbol{0\cdot \infty}}}
\newcommand{\inftyMinusInfty}{\ensuremath{\small\boldsymbol{\infty - \infty}}}
\newcommand{\oneToInfty}{\ensuremath{\boldsymbol{1^\infty}}}
\newcommand{\zeroToZero}{\ensuremath{\boldsymbol{0^0}}}
\newcommand{\inftyToZero}{\ensuremath{\boldsymbol{\infty^0}}}



\newcommand{\numOverZero}{\ensuremath{\boldsymbol{\tfrac{\#}{0}}}}
\newcommand{\dfn}{\textbf}
%\newcommand{\unit}{\,\mathrm}
\newcommand{\unit}{\mathop{}\!\mathrm}
\newcommand{\eval}[1]{\bigg[ #1 \bigg]}
\newcommand{\seq}[1]{\left( #1 \right)}
\renewcommand{\epsilon}{\varepsilon}
\renewcommand{\phi}{\varphi}


\renewcommand{\iff}{\Leftrightarrow}

\DeclareMathOperator{\arccot}{arccot}
\DeclareMathOperator{\arcsec}{arcsec}
\DeclareMathOperator{\arccsc}{arccsc}
\DeclareMathOperator{\si}{Si}
\DeclareMathOperator{\scal}{scal}
\DeclareMathOperator{\sign}{sign}


%% \newcommand{\tightoverset}[2]{% for arrow vec
%%   \mathop{#2}\limits^{\vbox to -.5ex{\kern-0.75ex\hbox{$#1$}\vss}}}
\newcommand{\arrowvec}[1]{{\overset{\rightharpoonup}{#1}}}
%\renewcommand{\vec}[1]{\arrowvec{\mathbf{#1}}}
\renewcommand{\vec}[1]{{\overset{\boldsymbol{\rightharpoonup}}{\mathbf{#1}}}\hspace{0in}}

\newcommand{\point}[1]{\left(#1\right)} %this allows \vector{ to be changed to \vector{ with a quick find and replace
\newcommand{\pt}[1]{\mathbf{#1}} %this allows \vec{ to be changed to \vec{ with a quick find and replace
\newcommand{\Lim}[2]{\lim_{\point{#1} \to \point{#2}}} %Bart, I changed this to point since I want to use it.  It runs through both of the exercise and exerciseE files in limits section, which is why it was in each document to start with.

\DeclareMathOperator{\proj}{\mathbf{proj}}
\newcommand{\veci}{{\boldsymbol{\hat{\imath}}}}
\newcommand{\vecj}{{\boldsymbol{\hat{\jmath}}}}
\newcommand{\veck}{{\boldsymbol{\hat{k}}}}
\newcommand{\vecl}{\vec{\boldsymbol{\l}}}
\newcommand{\uvec}[1]{\mathbf{\hat{#1}}}
\newcommand{\utan}{\mathbf{\hat{t}}}
\newcommand{\unormal}{\mathbf{\hat{n}}}
\newcommand{\ubinormal}{\mathbf{\hat{b}}}

\newcommand{\dotp}{\bullet}
\newcommand{\cross}{\boldsymbol\times}
\newcommand{\grad}{\boldsymbol\nabla}
\newcommand{\divergence}{\grad\dotp}
\newcommand{\curl}{\grad\cross}
%\DeclareMathOperator{\divergence}{divergence}
%\DeclareMathOperator{\curl}[1]{\grad\cross #1}
\newcommand{\lto}{\mathop{\longrightarrow\,}\limits}

\renewcommand{\bar}{\overline}

\colorlet{textColor}{black}
\colorlet{background}{white}
\colorlet{penColor}{blue!50!black} % Color of a curve in a plot
\colorlet{penColor2}{red!50!black}% Color of a curve in a plot
\colorlet{penColor3}{red!50!blue} % Color of a curve in a plot
\colorlet{penColor4}{green!50!black} % Color of a curve in a plot
\colorlet{penColor5}{orange!80!black} % Color of a curve in a plot
\colorlet{penColor6}{yellow!70!black} % Color of a curve in a plot
\colorlet{fill1}{penColor!20} % Color of fill in a plot
\colorlet{fill2}{penColor2!20} % Color of fill in a plot
\colorlet{fillp}{fill1} % Color of positive area
\colorlet{filln}{penColor2!20} % Color of negative area
\colorlet{fill3}{penColor3!20} % Fill
\colorlet{fill4}{penColor4!20} % Fill
\colorlet{fill5}{penColor5!20} % Fill
\colorlet{gridColor}{gray!50} % Color of grid in a plot

\newcommand{\surfaceColor}{violet}
\newcommand{\surfaceColorTwo}{redyellow}
\newcommand{\sliceColor}{greenyellow}




\pgfmathdeclarefunction{gauss}{2}{% gives gaussian
  \pgfmathparse{1/(#2*sqrt(2*pi))*exp(-((x-#1)^2)/(2*#2^2))}%
}


%%%%%%%%%%%%%
%% Vectors
%%%%%%%%%%%%%

%% Simple horiz vectors
\renewcommand{\vector}[1]{\left\langle #1\right\rangle}


%% %% Complex Horiz Vectors with angle brackets
%% \makeatletter
%% \renewcommand{\vector}[2][ , ]{\left\langle%
%%   \def\nextitem{\def\nextitem{#1}}%
%%   \@for \el:=#2\do{\nextitem\el}\right\rangle%
%% }
%% \makeatother

%% %% Vertical Vectors
%% \def\vector#1{\begin{bmatrix}\vecListA#1,,\end{bmatrix}}
%% \def\vecListA#1,{\if,#1,\else #1\cr \expandafter \vecListA \fi}

%%%%%%%%%%%%%
%% End of vectors
%%%%%%%%%%%%%

%\newcommand{\fullwidth}{}
%\newcommand{\normalwidth}{}



%% makes a snazzy t-chart for evaluating functions
%\newenvironment{tchart}{\rowcolors{2}{}{background!90!textColor}\array}{\endarray}

%%This is to help with formatting on future title pages.
\newenvironment{sectionOutcomes}{}{}



%% Flowchart stuff
%\tikzstyle{startstop} = [rectangle, rounded corners, minimum width=3cm, minimum height=1cm,text centered, draw=black]
%\tikzstyle{question} = [rectangle, minimum width=3cm, minimum height=1cm, text centered, draw=black]
%\tikzstyle{decision} = [trapezium, trapezium left angle=70, trapezium right angle=110, minimum width=3cm, minimum height=1cm, text centered, draw=black]
%\tikzstyle{question} = [rectangle, rounded corners, minimum width=3cm, minimum height=1cm,text centered, draw=black]
%\tikzstyle{process} = [rectangle, minimum width=3cm, minimum height=1cm, text centered, draw=black]
%\tikzstyle{decision} = [trapezium, trapezium left angle=70, trapezium right angle=110, minimum width=3cm, minimum height=1cm, text centered, draw=black]




\author{Tom Needham and Jim Talamo}

\outcome{Evaluate integrals using partial fraction decompositions.}

\title[]{Partial Fractions}

\begin{document}
\begin{abstract}
\end{abstract}
\maketitle

\vspace{-0.9in}

\section{Discussion Questions}

\begin{problem}
Classify each of the following expressions as an irreducible quadratic, a product of distinct linear factors, or a product of repeated linear factors.
\begin{center}
\begin{tabular}{lll}
I. $x^2+5x+1$ \hspace{.4in} II. $x^2$ \hspace{.4in} III. $x^2+x+1$
\end{tabular}
\end{center}
\end{problem}

\begin{freeResponse}
By definition, a quadratic polynomial $ax^2+bx+c$ is irreducible if it cannot be factored into a product of degree one polynomials with real coefficients.  Sometimes, it is easy to spot a factorization of the polynomial, but there is a difference between not knowing how to factor something and claiming that it cannot be factored.  Thankfully, there's a theorem for that.

\begin{theorem}
A quadratic polynomial $ax^2+bx+c$ is irreducible if and only if the equation $ax^2+bx+c=0$ has no real roots.  
\end{theorem}

We have the quadratic formula to aid us in solving quadratic equations.  Indeed, if $ax^2+bx+c=0$, then

\[
x= \frac{-b \pm \sqrt{b^2-4ac}}{2a}.
\]

We can use the \emph{discriminant} $b^2-4ac$ to determine if the roots are real.

I. Although it may not be immediately obvious how to factor it, this quadratic is reducible. To test this, one checks the discriminant $b^2-4ac$; the discriminant is  $\geq 0$ if and only if the polynomial can be factored, and is strictly greater than zero if and only if the factors are distinct. In this case the discriminant is $5^2-4 \cdot 1 \cdot 1 = 21 > 0$. The expression is therefore a product of distinct linear factors.

II. This expression is a product of the repeated linear factor $x$.  Alternatively, it is not irreducible since the equation $x^2=0$ has a repeated, real root $x=0$.

III. The discriminant is $1^2-4 \cdot 1 \cdot 1 = -3 < 0$, so the quadratic is irreducible.
\end{freeResponse}

\begin{problem}
A student claims that the partial fraction decomposition for $\frac{6x^2+7x+1}{x(2x+1)^2}$ is $\frac{Ax+B}{(2x+1)^2}+\frac{C}{x}$ and justifies his or her response by finding that $A=2$, $B=3$ and $C=1$.

\begin{itemize}
\item[I.] Is it true that $\frac{6x^2+7x+1}{x(2x+1)^2} = \frac{2x+3}{(2x+1)^2}+\frac{1}{x}$?
\item[II.] Is this the correct partial fraction decomposition?
\end{itemize}
\end{problem}

\begin{freeResponse}
I. This is true, as we can readily check:
\begin{align*}
\frac{2x+3}{(2x+1)^2}+\frac{1}{x} &= \frac{2x^2+3x}{x(2x+1)^2} + \frac{1}{x} \cdot \frac{(2x+1)^2}{(2x+1)^2} \\
&= \frac{2x^2+3x+4x^2+4x+1}{x(2x+1)^2} \\
&=\frac{6x^2+7x+1}{x(2x+1)^2}.
\end{align*}

II. Remember that the whole point of the method of partial fraction decomposition is to rewrite the rational function in an equivalent way which \emph{we know how to integrate}. The student has provided an equivalent expression for the rational function, but their expression includes the term  $\frac{2x+3}{(2x+1)^2}$, and it is not clear how to find the antiderivatives of this term immediately. The correct partial fraction decomposition is of the form
$$
\frac{6x^2+7x+1}{x(2x+1)^2} = \frac{A}{x} + \frac{B}{2x+1} + \frac{C}{(2x+1)^2}.
$$
Note that each term in the resulting definition is something that we know how to integrate. 

While you are not asked to do so, you can use the usual method for finding the constants to see that $A=1$, $B=1$ and $C=2$, so that

\[
\frac{6x^2+7x+1}{x(2x+1)^2} = \frac{1}{x} + \frac{1}{2x+1} + \frac{2}{(2x+1)^2}.
\]

Note that if we have to find $\int \frac{6x^2+7x+1}{x(2x+1)^2} \d x$, it is much easier to compute $\int \frac{1}{x} + \frac{1}{2x+1} + \frac{2}{(2x+1)^2} \d x$ than it is to compute $\int \frac{2x+3}{(2x+1)^2}+\frac{1}{x} \d x$.

\begin{remark}
For students curious to see how we should integrate $\frac{2x+3}{(2x+1)^2}$, this could be done via the substitution $u=2x+1$ and some algebra, or by noting

\[
\frac{2x+3}{(2x+1)^2} =\frac{2x+1+2}{(2x+1)^2}=\frac{2x+1}{(2x+1)^2}+\frac{2}{(2x+1)^2} = \frac{1}{2x+1}+\frac{2}{(2x+1)^2}.
\]

This is exactly what is obtained by the correct partial fraction decomposition, so it's much better to start with the partial fraction decomposition rather than the other algebraically equivalent form!

\end{remark}
\end{freeResponse}

\section{Group Work}

\begin{problem}
Determine the form of the partial fraction decomposition of each expression, but don't solve for constants.
\begin{center}
\begin{tabular}{lll}
I. $\frac{2x+1}{x^2+5x+6}$ \hspace{.2in} II. $\frac{2x^2+x+3}{(x^2-2x+1)(x+2)}$ \hspace{.2in} III. $\frac{3x^2+x+1}{3x^3+3x^2+6x}$
\end{tabular}
\end{center}
\end{problem}

\begin{freeResponse}
The partial fraction decompositions take the forms:

I. $\frac{2x+1}{x^2+5x+6} = \frac{2x+1}{(x+2)(x+3)} = \frac{A}{x+2} + \frac{B}{x+3}$

II. $\frac{2x^2+x+3}{(x^2-2x+1)(x+2)} = \frac{2x^2+x+3}{(x-1)^2(x+2)} = \frac{A}{x-1} + \frac{B}{(x-1)^2} + \frac{C}{x+2}$

III. $\frac{3x^2+x+1}{3x^3+3x^2+6x} = \frac{3x^2 + x + 1}{3x(x^2+x + 2)} $

In order to continue, we need to determine if $x^2+x + 2$ as irreducible.  The discriminant is $b^2-4ac = 1-4(2)(1) <0$, so this is irreducible.  The partial fraction decomposition is thus

\[
\frac{3x^2 + x + 1}{3x(x^2+x + 2)} = \frac{A}{3x}+\frac{Bx+C}{x^2+x+2}.
\]
\end{freeResponse}

\begin{problem}
Evaluate the integral
$$
\int \frac{1}{x^3-x^2} \d x.
$$
\end{problem}

\begin{freeResponse}
The partial fraction decomposition of the integrand is given by 
$$
\frac{1}{x^3-x^2} = \frac{1}{x^2(x-1)} = \frac{A}{x} + \frac{B}{x^2} + \frac{C}{x-1}.
$$
Clearing denominators in this equation, we have
$$
1 = Ax(x-1) + B(x-1) + Cx^2 = (A + C)x^2 + (-A + B) x + (-B).
$$
Comparing terms by degree, it must be that $-B = 1$, or $B=-1$. We also have $-A+B = 0$, or $A=B=-1$. Finally, $A+C=0$ implies $C = -A = 1$. We conclude that 
$$
\frac{1}{x^3-x^2} = -\frac{1}{x} - \frac{1}{x^2} + \frac{1}{x-1}.
$$
Therefore
\begin{align*}
\int \frac{1}{x^3-x^2} \d x &= \int \frac{1}{x^3-x^2} = \int -\frac{1}{x} - \frac{1}{x^2} + \frac{1}{x-1}\d x  \\
&= -\ln |x| + \frac{1}{x} + \ln |x-1| + C.
\end{align*} 
\end{freeResponse}

\begin{problem}
Consider the integral
$$
\int_2^3 \frac{1}{x^2-1} \d x.
$$
Evaluate the integral in two ways: first by partial fraction decomposition , then via trigonometric substitution with $x=\sec(\theta)$. For the trigonometric substitution method, it may be useful to recall that
$$
\int \csc \theta \d \theta = -\ln | \csc \theta + \cot \theta | + C.
$$
If we change the bounds of integration to $x=0$ to $x=1/2$, would both methods still be valid?
\end{problem}

\begin{freeResponse}
 
First, the partial fraction decomposition of the integrand is 
$$
\frac{1}{x^2-1} = \frac{1}{(x+1)(x-1)} = \frac{A}{x+1} + \frac{B}{x-1}.
$$
Clearing denominators, we have
$$
1 = A(x-1) + B(x+1) = (A+B)x + (-A+B).
$$
Comparing terms of the same degree, we have $-A+B=1$ and $A+B = 0$. Solving the second equation yields $A=-B$, and substituting this into the first equation yields $2B=1$, or $B=1/2$. Therefore $A=-1/2$, and 
$$
\frac{1}{x^2-1} = -\frac{1/2}{x+1} + \frac{1/2}{x-1}.
$$
The integral is therefore given by
\begin{align*}
\int_2^3 \frac{1}{x^2-1} \d x &= \int_2^3  -\frac{1/2}{x+1} + \frac{1/2}{x-1} \d x \\
&= \eval{-\frac{1}{2} \ln |x+1| + \frac{1}{2} \ln |x-1| }_2^3 \\
&= \frac{1}{2}\left(- \ln 4 + \ln 2 + \ln 3 - \ln 1\right) \\
&=  -\frac{1}{2} \ln 2 + \frac{1}{2} \ln 3.
\end{align*}

On the other hand, for the trigonometric substitution, let $x=\sec(\theta)$, so that $\d x = \sec(\theta) \tan \theta \d \theta$. Then
\begin{align*}
\int \frac{1}{x^2-1} \d x &= \int \frac{1}{\tan^2 \theta} \sec(\theta) \tan \theta \d \theta \\
&= \int \frac{\sec(\theta)}{\tan \theta} \d \theta \\
&= \int \frac{1}{\cos \theta} \frac{\cos \theta}{\sin \theta} \d \theta \\
&= \int \csc \theta \d \theta \\
&=  - \ln |\csc \theta + \cot \theta| + C \\
&= - \ln \left|\frac{x}{\sqrt{x^2-1}} + \frac{1}{\sqrt{x^2-1}}\right|+C,
\end{align*}
with the last line derived via a reference triangle. Therefore
\begin{align*}
\int_2^3 \frac{1}{x^2-1} \d x &= \eval{-\ln \left|\frac{x}{\sqrt{x^2-1}} + \frac{1}{\sqrt{x^2-1}}\right|}_2^3 \\
&= -\ln \left|\frac{3}{\sqrt{8}} + \frac{1}{\sqrt{8}}\right| + \ln \left| \frac{2}{\sqrt{3}} + \frac{1}{\sqrt{3}}\right|\\
&= -\ln \left(\frac{4}{\sqrt{8}}\right) + \ln \left(\frac{3}{\sqrt{3}}\right) \\
\end{align*}

To try to see how both answers are actually the same, recall the the important algebraic rules of logarithms below.

\begin{itemize}
\item $\ln(a)+\ln(b) = \ln(ab)$ 
\item $ \ln(a)-\ln(b) = \ln\left(\frac{a}{b}\right)$
\item $b \ln(a) = \ln\left(a^b\right)$
\end{itemize}

Thus, we can rewrite the answer above.

\begin{align*}
-\ln \left(\frac{4}{\sqrt{8}}\right) + \ln \left(\frac{3}{\sqrt{3}}\right) &= -\ln(4)+\ln(\sqrt{8}) + \ln(3)-\ln(\sqrt{3}) \\
& = -\ln(4)+\frac{1}{2} \ln(4\cdot 2) + \ln(3)-\frac{1}{2}\ln(3) \\
& = -\ln(4)+\frac{1}{2} \ln(4) +\frac{1}{2} \ln(2) + \frac{1}{2}\ln(3) \\
& = -\frac{1}{2} \ln(4) +\frac{1}{2} \ln(2) + \frac{1}{2}\ln(3) \\
& = -\frac{1}{2} \ln(2^2) +\frac{1}{2} \ln(2) + \frac{1}{2}\ln(3) \\
& = -\cancel{\frac{1}{2}} \cdot \cancel{2} \ln(2) +\frac{1}{2} \ln(2) + \frac{1}{2}\ln(3) \\
& = -\frac{1}{2} \ln(2) + \frac{1}{2}\ln(3) \\
\end{align*}

We therefore obtain the same answer as we did using partial fractions, but with considerably more effort. Moreover, making the substitution $x=\sec( \theta)$ implicitly assumes that $|x| \geq 1$ (since this is the range of the secant function). This means that the trigonometric substitution method would not be valid if we integrated over any region containing $x$ with $|x| < 1$. 
\end{freeResponse}


\end{document}
