\documentclass[handout]{ximera}
%handout:  for handout version with no solutions or instructor notes
%handout,instructornotes:  for instructor version with just problems and notes, no solutions
%noinstructornotes:  shows only problem and solutions

%% handout
%% space
%% newpage
%% numbers
%% nooutcomes

%I added the commands here so that I would't have to keep looking them up
%\newcommand{\RR}{\mathbb R}
%\renewcommand{\d}{\,d}
%\newcommand{\dd}[2][]{\frac{d #1}{d #2}}
%\renewcommand{\l}{\ell}
%\newcommand{\ddx}{\frac{d}{dx}}
%\everymath{\displaystyle}
%\newcommand{\dfn}{\textbf}
%\newcommand{\eval}[1]{\bigg[ #1 \bigg]}

%\begin{image}
%\includegraphics[trim= 170 420 250 180]{Figure1.pdf}
%\end{image}

%add a ``.'' below when used in a specific directory.

%\usepackage{todonotes}
%\usepackage{mathtools} %% Required for wide table Curl and Greens
%\usepackage{cuted} %% Required for wide table Curl and Greens
\newcommand{\todo}{}

\usepackage{multicol}

\usepackage{esint} % for \oiint
\ifxake%%https://math.meta.stackexchange.com/questions/9973/how-do-you-render-a-closed-surface-double-integral
\renewcommand{\oiint}{{\large\bigcirc}\kern-1.56em\iint}
\fi

\graphicspath{
  {./}
  {ximeraTutorial/}
  {basicPhilosophy/}
  {functionsOfSeveralVariables/}
  {normalVectors/}
  {lagrangeMultipliers/}
  {vectorFields/}
  {greensTheorem/}
  {shapeOfThingsToCome/}
  {dotProducts/}
  {partialDerivativesAndTheGradientVector/}
  {../ximeraTutorial/}
  {../productAndQuotientRules/exercises/}
  {../motionAndPathsInSpace/exercises/}
  {../normalVectors/exercisesParametricPlots/}
  {../continuityOfFunctionsOfSeveralVariables/exercises/}
  {../partialDerivativesAndTheGradientVector/exercises/}
  {../directionalDerivativeAndChainRule/exercises/}
  {../commonCoordinates/exercisesCylindricalCoordinates/}
  {../commonCoordinates/exercisesSphericalCoordinates/}
  {../greensTheorem/exercisesCurlAndLineIntegrals/}
  {../greensTheorem/exercisesDivergenceAndLineIntegrals/}
  {../shapeOfThingsToCome/exercisesDivergenceTheorem/}
  {../greensTheorem/}
  {../shapeOfThingsToCome/}
  {../separableDifferentialEquations/exercises/}
  {../dotproducts/}
  {../functionsOfSeveralVariables/}
  {../lagrangeMultipliers/}
  {../partialDerivativesAndTheGradientVector/}
  {../normalVectors/}
  {../vectorFields/}
}
\def\xmNotExpandableAsAccordion{true}

\newcommand{\mooculus}{\textsf{\textbf{MOOC}\textnormal{\textsf{ULUS}}}}

\usepackage{tkz-euclide}\usepackage{tikz}
\usepackage{tikz-cd}
\usetikzlibrary{arrows}
\tikzset{>=stealth,commutative diagrams/.cd,
  arrow style=tikz,diagrams={>=stealth}} %% cool arrow head
\tikzset{shorten <>/.style={ shorten >=#1, shorten <=#1 } } %% allows shorter vectors

\usetikzlibrary{backgrounds} %% for boxes around graphs
\usetikzlibrary{shapes,positioning}  %% Clouds and stars
\usetikzlibrary{matrix} %% for matrix
\usepgfplotslibrary{polar} %% for polar plots
\usepgfplotslibrary{fillbetween} %% to shade area between curves in TikZ
%\usetkzobj{all} %% obsolete

\usepackage[makeroom]{cancel} %% for strike outs
%\usepackage{mathtools} %% for pretty underbrace % Breaks Ximera
%\usepackage{multicol}
\usepackage{pgffor} %% required for integral for loops

\usepackage{tkz-tab}  %% for sign charts

%% http://tex.stackexchange.com/questions/66490/drawing-a-tikz-arc-specifying-the-center
%% Draws beach ball
\tikzset{pics/carc/.style args={#1:#2:#3}{code={\draw[pic actions] (#1:#3) arc(#1:#2:#3);}}}



\usepackage{array}
\setlength{\extrarowheight}{+.1cm}
\newdimen\digitwidth
\settowidth\digitwidth{9}
\def\divrule#1#2{
\noalign{\moveright#1\digitwidth
\vbox{\hrule width#2\digitwidth}}}





\newcommand{\RR}{\mathbb R}
\newcommand{\R}{\mathbb R}
\newcommand{\N}{\mathbb N}
\newcommand{\Z}{\mathbb Z}

\newcommand{\sagemath}{\textsf{SageMath}}


\renewcommand{\d}{\,d}
%\def\d{\mathop{}\!d}
%\def\d{\,d}

\AddToHook{begindocument}{%
  \renewcommand{\d}{\,d}     % lualatex redefines \d to underdot !!!
}

\pgfplotsset{
    every axis/.style={
        scale only axis,
        enlargelimits=false,
        trim axis left,
        trim axis right,
        clip=true,
    }
}

\newcommand{\dd}[2][]{\frac{\d #1}{\d #2}}
\newcommand{\pp}[2][]{\frac{\partial #1}{\partial #2}}
\renewcommand{\l}{\ell}
\newcommand{\ddx}{\frac{d}{\d x}}

\newcommand{\zeroOverZero}{\ensuremath{\boldsymbol{\tfrac{0}{0}}}}
\newcommand{\inftyOverInfty}{\ensuremath{\boldsymbol{\tfrac{\infty}{\infty}}}}
\newcommand{\zeroOverInfty}{\ensuremath{\boldsymbol{\tfrac{0}{\infty}}}}
\newcommand{\zeroTimesInfty}{\ensuremath{\small\boldsymbol{0\cdot \infty}}}
\newcommand{\inftyMinusInfty}{\ensuremath{\small\boldsymbol{\infty - \infty}}}
\newcommand{\oneToInfty}{\ensuremath{\boldsymbol{1^\infty}}}
\newcommand{\zeroToZero}{\ensuremath{\boldsymbol{0^0}}}
\newcommand{\inftyToZero}{\ensuremath{\boldsymbol{\infty^0}}}



\newcommand{\numOverZero}{\ensuremath{\boldsymbol{\tfrac{\#}{0}}}}
\newcommand{\dfn}{\textbf}
%\newcommand{\unit}{\,\mathrm}
\newcommand{\unit}{\mathop{}\!\mathrm}
\newcommand{\eval}[1]{\bigg[ #1 \bigg]}
\newcommand{\seq}[1]{\left( #1 \right)}
\renewcommand{\epsilon}{\varepsilon}
\renewcommand{\phi}{\varphi}


\renewcommand{\iff}{\Leftrightarrow}

\DeclareMathOperator{\arccot}{arccot}
\DeclareMathOperator{\arcsec}{arcsec}
\DeclareMathOperator{\arccsc}{arccsc}
\DeclareMathOperator{\si}{Si}
\DeclareMathOperator{\scal}{scal}
\DeclareMathOperator{\sign}{sign}


%% \newcommand{\tightoverset}[2]{% for arrow vec
%%   \mathop{#2}\limits^{\vbox to -.5ex{\kern-0.75ex\hbox{$#1$}\vss}}}
\newcommand{\arrowvec}[1]{{\overset{\rightharpoonup}{#1}}}
%\renewcommand{\vec}[1]{\arrowvec{\mathbf{#1}}}
\renewcommand{\vec}[1]{{\overset{\boldsymbol{\rightharpoonup}}{\mathbf{#1}}}\hspace{0in}}

\newcommand{\point}[1]{\left(#1\right)} %this allows \vector{ to be changed to \vector{ with a quick find and replace
\newcommand{\pt}[1]{\mathbf{#1}} %this allows \vec{ to be changed to \vec{ with a quick find and replace
\newcommand{\Lim}[2]{\lim_{\point{#1} \to \point{#2}}} %Bart, I changed this to point since I want to use it.  It runs through both of the exercise and exerciseE files in limits section, which is why it was in each document to start with.

\DeclareMathOperator{\proj}{\mathbf{proj}}
\newcommand{\veci}{{\boldsymbol{\hat{\imath}}}}
\newcommand{\vecj}{{\boldsymbol{\hat{\jmath}}}}
\newcommand{\veck}{{\boldsymbol{\hat{k}}}}
\newcommand{\vecl}{\vec{\boldsymbol{\l}}}
\newcommand{\uvec}[1]{\mathbf{\hat{#1}}}
\newcommand{\utan}{\mathbf{\hat{t}}}
\newcommand{\unormal}{\mathbf{\hat{n}}}
\newcommand{\ubinormal}{\mathbf{\hat{b}}}

\newcommand{\dotp}{\bullet}
\newcommand{\cross}{\boldsymbol\times}
\newcommand{\grad}{\boldsymbol\nabla}
\newcommand{\divergence}{\grad\dotp}
\newcommand{\curl}{\grad\cross}
%\DeclareMathOperator{\divergence}{divergence}
%\DeclareMathOperator{\curl}[1]{\grad\cross #1}
\newcommand{\lto}{\mathop{\longrightarrow\,}\limits}

\renewcommand{\bar}{\overline}

\colorlet{textColor}{black}
\colorlet{background}{white}
\colorlet{penColor}{blue!50!black} % Color of a curve in a plot
\colorlet{penColor2}{red!50!black}% Color of a curve in a plot
\colorlet{penColor3}{red!50!blue} % Color of a curve in a plot
\colorlet{penColor4}{green!50!black} % Color of a curve in a plot
\colorlet{penColor5}{orange!80!black} % Color of a curve in a plot
\colorlet{penColor6}{yellow!70!black} % Color of a curve in a plot
\colorlet{fill1}{penColor!20} % Color of fill in a plot
\colorlet{fill2}{penColor2!20} % Color of fill in a plot
\colorlet{fillp}{fill1} % Color of positive area
\colorlet{filln}{penColor2!20} % Color of negative area
\colorlet{fill3}{penColor3!20} % Fill
\colorlet{fill4}{penColor4!20} % Fill
\colorlet{fill5}{penColor5!20} % Fill
\colorlet{gridColor}{gray!50} % Color of grid in a plot

\newcommand{\surfaceColor}{violet}
\newcommand{\surfaceColorTwo}{redyellow}
\newcommand{\sliceColor}{greenyellow}




\pgfmathdeclarefunction{gauss}{2}{% gives gaussian
  \pgfmathparse{1/(#2*sqrt(2*pi))*exp(-((x-#1)^2)/(2*#2^2))}%
}


%%%%%%%%%%%%%
%% Vectors
%%%%%%%%%%%%%

%% Simple horiz vectors
\renewcommand{\vector}[1]{\left\langle #1\right\rangle}


%% %% Complex Horiz Vectors with angle brackets
%% \makeatletter
%% \renewcommand{\vector}[2][ , ]{\left\langle%
%%   \def\nextitem{\def\nextitem{#1}}%
%%   \@for \el:=#2\do{\nextitem\el}\right\rangle%
%% }
%% \makeatother

%% %% Vertical Vectors
%% \def\vector#1{\begin{bmatrix}\vecListA#1,,\end{bmatrix}}
%% \def\vecListA#1,{\if,#1,\else #1\cr \expandafter \vecListA \fi}

%%%%%%%%%%%%%
%% End of vectors
%%%%%%%%%%%%%

%\newcommand{\fullwidth}{}
%\newcommand{\normalwidth}{}



%% makes a snazzy t-chart for evaluating functions
%\newenvironment{tchart}{\rowcolors{2}{}{background!90!textColor}\array}{\endarray}

%%This is to help with formatting on future title pages.
\newenvironment{sectionOutcomes}{}{}



%% Flowchart stuff
%\tikzstyle{startstop} = [rectangle, rounded corners, minimum width=3cm, minimum height=1cm,text centered, draw=black]
%\tikzstyle{question} = [rectangle, minimum width=3cm, minimum height=1cm, text centered, draw=black]
%\tikzstyle{decision} = [trapezium, trapezium left angle=70, trapezium right angle=110, minimum width=3cm, minimum height=1cm, text centered, draw=black]
%\tikzstyle{question} = [rectangle, rounded corners, minimum width=3cm, minimum height=1cm,text centered, draw=black]
%\tikzstyle{process} = [rectangle, minimum width=3cm, minimum height=1cm, text centered, draw=black]
%\tikzstyle{decision} = [trapezium, trapezium left angle=70, trapezium right angle=110, minimum width=3cm, minimum height=1cm, text centered, draw=black]


\author{Jim Talamo}

\outcome{Use definite integrals to find areas of surfaces of revolution.}


\title[]{Surface Area of Revolution}

\begin{document}
\begin{abstract}
\end{abstract}
\maketitle

\vspace{-0.9in}

\section{Discussion Questions}

\begin{problem}
The segment of the curve $y=x^3$ from $x=0$ to $x=2$ is revolved about the $y$-axis.  

\begin{enumerate}
\item[I.] Sketch the curve. 
\item[II.] Show that the the length element $ds$ in terms of $x$ and $\d x$ is: \[ds = \sqrt{1+9x^4} \d x.\]
\item[III.] Write an expression for the length element $ds$ in terms of $y$ and $\d y$.
\item[IV.] Which of the following integrals gives the area of the surface of revolution?

\begin{tabular}{ll}
A. $\int_0^2 2\pi x \sqrt{1+9x^4} \d x$ & B. $\int_0^2 2\pi x^3 \sqrt{1+9x^4} \d x$  \\[4ex]
C. $\int_0^8 2\pi y \sqrt{1+\frac{1}{9}y^{-4/3}} \d y$ \qquad \qquad & D. $\int_0^8 2\pi y^{1/3} \sqrt{1+\frac{1}{9}y^{-4/3}} \d y$
\end{tabular}
\end{enumerate}
\end{problem}

\begin{freeResponse}

\end{freeResponse}

\section{Group Work}

\begin{problem}
The curve $C$ is the segment of $y=\cos(2x)$ from $x=0$ to $x=\pi/4$.  Set up an integral with respect to $x$ and an integral with respect to $y$ that gives the area of the surface of revolution when the curve is revolved about the following axes: 

\begin{tabular}{lll}
I. The $x$-axis. \qquad \qquad II. The line $x=-4$.  \qquad \qquad III. The line $y=5$.
\end{tabular}

\begin{freeResponse}

\end{freeResponse}

\end{problem}

\begin{problem}
Find the surface area of the surface generated by revolving the curve given by
	\begin{enumerate}
			\item  $x = 2y^3$ from $\left( 0, 0 \right)$ to $\left( 2, 1 \right)$ about the $y$-axis.
		\begin{freeResponse}
		The formula for the surface area is
			\[
			\text{{\color{red} Surface Area}} = \int_0^{1} 2 \pi f(y) \sqrt{1+f'(y)^2} \d y.
			\]
		Since $x = f(y) = 2y^3$, 
		we know that $f'(y) = 6y^2$.  
		Note that
			\begin{align*}
			\sqrt{1+f'(y)^2} \d y  &= \sqrt{1+ \left( 6y^2 \right)^2}  \\
			&=  \sqrt{1+ 36y^4}  \\
			\end{align*}
		and so
			\begin{align*}
			\text{{\color{red} Surface Area}} &= \int_0^{1} 2 \pi \left(2y^3\right) \left(  \sqrt{1+ 36y^4} \right) \d y  \\
			&= \int_0^{1} 4 \pi y^3  \sqrt{1+ 36y^4}  \d y \\
			& \\
			u&=1+ 36y^4 \\
			du &= 144y^3dy \\
			\frac{du}{144} &= y^3dy \\
			& \\
			u(0)&=1+36(0)^4 = 1\\
			u(1)&= 1+ 36(1)^4) = 37 \\
			& \\
			&= \frac{4 \pi}{144} \int_1^{37}  \sqrt{u}  \d u \\
			&= \frac{4 \pi}{144} \eval{\frac{2}{3}u^{\frac{3}{2}}}_1^{37}\\
			&= \frac{4 \pi}{144} \left[ \left( \frac{2}{3}(37)^{\frac{3}{2}} \right) - \left( \frac{2}{3}(1)^{\frac{3}{2}} \right) \right]  \\
			&= \frac{(37)^{\frac{3}{2}} -1 }{54}\pi
			\end{align*}
	
		\end{freeResponse}
	
	
		\item  $y = \frac{1}{6} x^3 + \frac{1}{2x}$ from $\left( 2, \frac{19}{12} \right)$ to $\left( 3, \frac{14}{3} \right)$ about the $x$-axis.
		\begin{freeResponse}
		The formula for the surface area is
			\[
			\text{{\color{red} Surface Area}} = \int_2^3 2 \pi f(x) \sqrt{1+f'(x)^2} \d x.
			\]
		Since $y = f(x) = \frac{1}{6} x^3 + \frac{1}{2x}$, we know that $f'(x) = \frac{1}{2} x^2 - \frac{1}{2} x^{-2}$.  
		Note that
			\begin{align*}
			\sqrt{1+f'(x)^2} &= \sqrt{1+ \left( \frac{1}{2}x^2 - \frac{1}{2}x^{-2} \right)^2}  \\
			&= \sqrt{1+ \left( \frac{1}{4}x^4 - \frac{1}{2} + \frac{1}{4}x^{-4} \right)}  \\
			&= \sqrt{\frac{1}{4}x^4 + \frac{1}{2} + \frac{1}{4}x^{-4}}  \\
			&= \sqrt{\left( \frac{1}{2}x^2 + \frac{1}{2}x^{-2} \right)^2}  \\
			&= \left( \frac{1}{2}x^2 + \frac{1}{2}x^{-2} \right)
			\end{align*}
		and so
			\begin{align*}
			\text{{\color{red} Surface Area}} &= \int_2^3 2 \pi \left( \frac{1}{6} x^3 + \frac{1}{2} x^{-1} \right) \left( \frac{1}{2} x^2 + \frac{1}{2} x^{-2} \right) \d x  \\
			&= 2 \pi \int_2^3 \left( \frac{1}{12} x^5 + \frac{1}{12} x + \frac{1}{4} x + \frac{1}{4} x^{-3} \right) \d x  \\
			&= 2 \pi \int_2^3 \left( \frac{1}{12} x^5 + \frac{1}{3} x + \frac{1}{4} x^{-3} \right) \d x  \\
			&= 2 \pi \eval{\frac{1}{72}x^6 + \frac{1}{6}x^2 - \frac{1}{8}x^{-2}}_2^3  \\
			&= 2\pi \left[ \left( \frac{81}{8} + \frac{3}{2} - \frac{1}{72} \right) - \left( \frac{8}{9} + \frac{2}{3} - \frac{1}{32} \right) \right]  \\
			&= 2\pi \left( \frac{2916 + 432 - 4 - 256 - 192 + 9}{288} \right)  \\
			&= \frac{2905 \pi}{144}.
			\end{align*}
		\end{freeResponse}
		
		
		

	\end{enumerate}
	
\end{problem}

\end{document}
