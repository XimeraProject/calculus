\documentclass[noauthor, handout]{ximera}
%handout:  for handout version with no solutions or instructor notes
%handout,instructornotes:  for instructor version with just problems and notes, no solutions
%noinstructornotes:  shows only problem and solutions

%% handout
%% space
%% newpage
%% numbers
%% nooutcomes

%I added the commands here so that I would't have to keep looking them up
%\newcommand{\RR}{\mathbb R}
%\renewcommand{\d}{\,d}
%\newcommand{\dd}[2][]{\frac{d #1}{d #2}}
%\renewcommand{\l}{\ell}
%\newcommand{\ddx}{\frac{d}{dx}}
%\everymath{\displaystyle}
%\newcommand{\dfn}{\textbf}
%\newcommand{\eval}[1]{\bigg[ #1 \bigg]}

%\begin{image}
%\includegraphics[trim= 170 420 250 180]{Figure1.pdf}
%\end{image}

%add a ``.'' below when used in a specific directory.

%\usepackage{todonotes}
%\usepackage{mathtools} %% Required for wide table Curl and Greens
%\usepackage{cuted} %% Required for wide table Curl and Greens
\newcommand{\todo}{}

\usepackage{multicol}

\usepackage{esint} % for \oiint
\ifxake%%https://math.meta.stackexchange.com/questions/9973/how-do-you-render-a-closed-surface-double-integral
\renewcommand{\oiint}{{\large\bigcirc}\kern-1.56em\iint}
\fi

\graphicspath{
  {./}
  {ximeraTutorial/}
  {basicPhilosophy/}
  {functionsOfSeveralVariables/}
  {normalVectors/}
  {lagrangeMultipliers/}
  {vectorFields/}
  {greensTheorem/}
  {shapeOfThingsToCome/}
  {dotProducts/}
  {partialDerivativesAndTheGradientVector/}
  {../ximeraTutorial/}
  {../productAndQuotientRules/exercises/}
  {../motionAndPathsInSpace/exercises/}
  {../normalVectors/exercisesParametricPlots/}
  {../continuityOfFunctionsOfSeveralVariables/exercises/}
  {../partialDerivativesAndTheGradientVector/exercises/}
  {../directionalDerivativeAndChainRule/exercises/}
  {../commonCoordinates/exercisesCylindricalCoordinates/}
  {../commonCoordinates/exercisesSphericalCoordinates/}
  {../greensTheorem/exercisesCurlAndLineIntegrals/}
  {../greensTheorem/exercisesDivergenceAndLineIntegrals/}
  {../shapeOfThingsToCome/exercisesDivergenceTheorem/}
  {../greensTheorem/}
  {../shapeOfThingsToCome/}
  {../separableDifferentialEquations/exercises/}
  {../dotproducts/}
  {../functionsOfSeveralVariables/}
  {../lagrangeMultipliers/}
  {../partialDerivativesAndTheGradientVector/}
  {../normalVectors/}
  {../vectorFields/}
}
\def\xmNotExpandableAsAccordion{true}

\newcommand{\mooculus}{\textsf{\textbf{MOOC}\textnormal{\textsf{ULUS}}}}

\usepackage{tkz-euclide}\usepackage{tikz}
\usepackage{tikz-cd}
\usetikzlibrary{arrows}
\tikzset{>=stealth,commutative diagrams/.cd,
  arrow style=tikz,diagrams={>=stealth}} %% cool arrow head
\tikzset{shorten <>/.style={ shorten >=#1, shorten <=#1 } } %% allows shorter vectors

\usetikzlibrary{backgrounds} %% for boxes around graphs
\usetikzlibrary{shapes,positioning}  %% Clouds and stars
\usetikzlibrary{matrix} %% for matrix
\usepgfplotslibrary{polar} %% for polar plots
\usepgfplotslibrary{fillbetween} %% to shade area between curves in TikZ
%\usetkzobj{all} %% obsolete

\usepackage[makeroom]{cancel} %% for strike outs
%\usepackage{mathtools} %% for pretty underbrace % Breaks Ximera
%\usepackage{multicol}
\usepackage{pgffor} %% required for integral for loops

\usepackage{tkz-tab}  %% for sign charts

%% http://tex.stackexchange.com/questions/66490/drawing-a-tikz-arc-specifying-the-center
%% Draws beach ball
\tikzset{pics/carc/.style args={#1:#2:#3}{code={\draw[pic actions] (#1:#3) arc(#1:#2:#3);}}}



\usepackage{array}
\setlength{\extrarowheight}{+.1cm}
\newdimen\digitwidth
\settowidth\digitwidth{9}
\def\divrule#1#2{
\noalign{\moveright#1\digitwidth
\vbox{\hrule width#2\digitwidth}}}





\newcommand{\RR}{\mathbb R}
\newcommand{\R}{\mathbb R}
\newcommand{\N}{\mathbb N}
\newcommand{\Z}{\mathbb Z}

\newcommand{\sagemath}{\textsf{SageMath}}


\renewcommand{\d}{\,d}
%\def\d{\mathop{}\!d}
%\def\d{\,d}

\AddToHook{begindocument}{%
  \renewcommand{\d}{\,d}     % lualatex redefines \d to underdot !!!
}

\pgfplotsset{
    every axis/.style={
        scale only axis,
        enlargelimits=false,
        trim axis left,
        trim axis right,
        clip=true,
    }
}

\newcommand{\dd}[2][]{\frac{\d #1}{\d #2}}
\newcommand{\pp}[2][]{\frac{\partial #1}{\partial #2}}
\renewcommand{\l}{\ell}
\newcommand{\ddx}{\frac{d}{\d x}}

\newcommand{\zeroOverZero}{\ensuremath{\boldsymbol{\tfrac{0}{0}}}}
\newcommand{\inftyOverInfty}{\ensuremath{\boldsymbol{\tfrac{\infty}{\infty}}}}
\newcommand{\zeroOverInfty}{\ensuremath{\boldsymbol{\tfrac{0}{\infty}}}}
\newcommand{\zeroTimesInfty}{\ensuremath{\small\boldsymbol{0\cdot \infty}}}
\newcommand{\inftyMinusInfty}{\ensuremath{\small\boldsymbol{\infty - \infty}}}
\newcommand{\oneToInfty}{\ensuremath{\boldsymbol{1^\infty}}}
\newcommand{\zeroToZero}{\ensuremath{\boldsymbol{0^0}}}
\newcommand{\inftyToZero}{\ensuremath{\boldsymbol{\infty^0}}}



\newcommand{\numOverZero}{\ensuremath{\boldsymbol{\tfrac{\#}{0}}}}
\newcommand{\dfn}{\textbf}
%\newcommand{\unit}{\,\mathrm}
\newcommand{\unit}{\mathop{}\!\mathrm}
\newcommand{\eval}[1]{\bigg[ #1 \bigg]}
\newcommand{\seq}[1]{\left( #1 \right)}
\renewcommand{\epsilon}{\varepsilon}
\renewcommand{\phi}{\varphi}


\renewcommand{\iff}{\Leftrightarrow}

\DeclareMathOperator{\arccot}{arccot}
\DeclareMathOperator{\arcsec}{arcsec}
\DeclareMathOperator{\arccsc}{arccsc}
\DeclareMathOperator{\si}{Si}
\DeclareMathOperator{\scal}{scal}
\DeclareMathOperator{\sign}{sign}


%% \newcommand{\tightoverset}[2]{% for arrow vec
%%   \mathop{#2}\limits^{\vbox to -.5ex{\kern-0.75ex\hbox{$#1$}\vss}}}
\newcommand{\arrowvec}[1]{{\overset{\rightharpoonup}{#1}}}
%\renewcommand{\vec}[1]{\arrowvec{\mathbf{#1}}}
\renewcommand{\vec}[1]{{\overset{\boldsymbol{\rightharpoonup}}{\mathbf{#1}}}\hspace{0in}}

\newcommand{\point}[1]{\left(#1\right)} %this allows \vector{ to be changed to \vector{ with a quick find and replace
\newcommand{\pt}[1]{\mathbf{#1}} %this allows \vec{ to be changed to \vec{ with a quick find and replace
\newcommand{\Lim}[2]{\lim_{\point{#1} \to \point{#2}}} %Bart, I changed this to point since I want to use it.  It runs through both of the exercise and exerciseE files in limits section, which is why it was in each document to start with.

\DeclareMathOperator{\proj}{\mathbf{proj}}
\newcommand{\veci}{{\boldsymbol{\hat{\imath}}}}
\newcommand{\vecj}{{\boldsymbol{\hat{\jmath}}}}
\newcommand{\veck}{{\boldsymbol{\hat{k}}}}
\newcommand{\vecl}{\vec{\boldsymbol{\l}}}
\newcommand{\uvec}[1]{\mathbf{\hat{#1}}}
\newcommand{\utan}{\mathbf{\hat{t}}}
\newcommand{\unormal}{\mathbf{\hat{n}}}
\newcommand{\ubinormal}{\mathbf{\hat{b}}}

\newcommand{\dotp}{\bullet}
\newcommand{\cross}{\boldsymbol\times}
\newcommand{\grad}{\boldsymbol\nabla}
\newcommand{\divergence}{\grad\dotp}
\newcommand{\curl}{\grad\cross}
%\DeclareMathOperator{\divergence}{divergence}
%\DeclareMathOperator{\curl}[1]{\grad\cross #1}
\newcommand{\lto}{\mathop{\longrightarrow\,}\limits}

\renewcommand{\bar}{\overline}

\colorlet{textColor}{black}
\colorlet{background}{white}
\colorlet{penColor}{blue!50!black} % Color of a curve in a plot
\colorlet{penColor2}{red!50!black}% Color of a curve in a plot
\colorlet{penColor3}{red!50!blue} % Color of a curve in a plot
\colorlet{penColor4}{green!50!black} % Color of a curve in a plot
\colorlet{penColor5}{orange!80!black} % Color of a curve in a plot
\colorlet{penColor6}{yellow!70!black} % Color of a curve in a plot
\colorlet{fill1}{penColor!20} % Color of fill in a plot
\colorlet{fill2}{penColor2!20} % Color of fill in a plot
\colorlet{fillp}{fill1} % Color of positive area
\colorlet{filln}{penColor2!20} % Color of negative area
\colorlet{fill3}{penColor3!20} % Fill
\colorlet{fill4}{penColor4!20} % Fill
\colorlet{fill5}{penColor5!20} % Fill
\colorlet{gridColor}{gray!50} % Color of grid in a plot

\newcommand{\surfaceColor}{violet}
\newcommand{\surfaceColorTwo}{redyellow}
\newcommand{\sliceColor}{greenyellow}




\pgfmathdeclarefunction{gauss}{2}{% gives gaussian
  \pgfmathparse{1/(#2*sqrt(2*pi))*exp(-((x-#1)^2)/(2*#2^2))}%
}


%%%%%%%%%%%%%
%% Vectors
%%%%%%%%%%%%%

%% Simple horiz vectors
\renewcommand{\vector}[1]{\left\langle #1\right\rangle}


%% %% Complex Horiz Vectors with angle brackets
%% \makeatletter
%% \renewcommand{\vector}[2][ , ]{\left\langle%
%%   \def\nextitem{\def\nextitem{#1}}%
%%   \@for \el:=#2\do{\nextitem\el}\right\rangle%
%% }
%% \makeatother

%% %% Vertical Vectors
%% \def\vector#1{\begin{bmatrix}\vecListA#1,,\end{bmatrix}}
%% \def\vecListA#1,{\if,#1,\else #1\cr \expandafter \vecListA \fi}

%%%%%%%%%%%%%
%% End of vectors
%%%%%%%%%%%%%

%\newcommand{\fullwidth}{}
%\newcommand{\normalwidth}{}



%% makes a snazzy t-chart for evaluating functions
%\newenvironment{tchart}{\rowcolors{2}{}{background!90!textColor}\array}{\endarray}

%%This is to help with formatting on future title pages.
\newenvironment{sectionOutcomes}{}{}



%% Flowchart stuff
%\tikzstyle{startstop} = [rectangle, rounded corners, minimum width=3cm, minimum height=1cm,text centered, draw=black]
%\tikzstyle{question} = [rectangle, minimum width=3cm, minimum height=1cm, text centered, draw=black]
%\tikzstyle{decision} = [trapezium, trapezium left angle=70, trapezium right angle=110, minimum width=3cm, minimum height=1cm, text centered, draw=black]
%\tikzstyle{question} = [rectangle, rounded corners, minimum width=3cm, minimum height=1cm,text centered, draw=black]
%\tikzstyle{process} = [rectangle, minimum width=3cm, minimum height=1cm, text centered, draw=black]
%\tikzstyle{decision} = [trapezium, trapezium left angle=70, trapezium right angle=110, minimum width=3cm, minimum height=1cm, text centered, draw=black]




\author{Tom Needham \and Jim Talamo}

\outcome{Recognize and evaluate geometric series.}
\outcome{Use properties of a sequence to determine properties of its sequence of partial sums.}

\title[]{Sums of Sequences}

\begin{document}
\begin{abstract}
\end{abstract}
\maketitle

\vspace{-0.9in}

\section{Discussion Questions}

\begin{problem}
Let $\{a_n\}_{n=1}^\infty$ be the sequence given by the formula
$$
a_n = \frac{n!}{3^n}.
$$
Find the first three terms in the sequence of partial sums associated to $\{a_n\}$. 
\end{problem}

\begin{freeResponse}
The first three terms in the partial sum sequence are 
\begin{align*}
s_1 &= a_1 = \frac{1}{3} \\
s_2 &= a_1 + a_2 = \frac{1}{3} + \frac{2}{9} = \frac{5}{9} \\
s_3 &= s_2 + a_3 = \frac{5}{9} + \frac{6}{27} = \frac{7}{9}.
\end{align*}
\end{freeResponse}

%%%%%%%%%%%%%%%%%%%%%%%%%%%%%%%%%%%%%%%%%%%%%%%

\begin{problem}
Determine which of the following are geometric series.
\begin{center}
\begin{tabular}{lll}
I. $\sum_{n=10}^\infty 3 \cdot 2^{1+n}$ \hspace{.2in} II. $\sum_{n=0}^\infty \left(\frac{1}{3n}\right)^n$ \hspace{.2in} III. $\sum_{k=2}^\infty 5 \cdot \frac{1}{k^2}$ \hspace{.2in} IV. $\sum_{k=2}^\infty 5 \cdot \frac{1}{2^k}$
\end{tabular}
\end{center}
\end{problem}

\begin{freeResponse}
A geometric series is a series of the form
$$
\sum_{n=n_0}^N A \cdot r^n
$$
for some integer $n_0$ and some constants $A$ and $r$. This is equivalent to the statement that the consecutive terms in the series have a common constant ratio for all $n$. 

I. This series is clearly geometric.

II. This is not a geometric series. We can see this by inspection, but to be thorough we should check the ratio of the terms. Since
$$
\frac{\left(\frac{1}{3(n+1)}\right)^{n+1}}{\left(\frac{1}{3n}\right)^n} = \left(\frac{n}{n+1}\right)^n \frac{1}{3(n+1)}
$$
is not constant (check a few terms, to be sure!), the series is not geometric.

III. This series is not geometric. Checking ratios of consecutive terms, we see that
$$
\frac{5 \frac{1}{(k+1)^2}}{ 5 \frac{1}{k^2}} = \frac{k^2}{(k+1)^2}
$$
is nonconstant.

IV. This series is geometric, since a slight algebraic manipulation puts it into exactly the standard form:
$$
\sum_{k=2}^\infty 5 \cdot \frac{1}{2^k} = \sum_{k=2}^\infty 5 \cdot \left(\frac{1}{2}\right)^k.
$$
\end{freeResponse}

%%%%%%%%%%%%%%%%%%%%%%%%%%%%%%%%%%%%%%%%%%%%%%%

\begin{problem}
Suppose that $\{a_n\}_{n=0}^\infty$ is a sequence with sequence of partial sums $\{s_n\}_{n=0}^\infty$ given by the explicit formula
$$
s_n = \left(\frac{1}{3}\right)^n.
$$
Find the values of the following series.
\begin{center}
\begin{tabular}{ll}
I. $\sum_{n=0}^\infty s_n$ \hspace{.6in} II. $\sum_{n=0}^\infty a_n$
\end{tabular}
\end{center}
\end{problem}

\begin{freeResponse}
I. Using our result on sums of geometric series, we have
$$
\sum_{n=0}^\infty s_n = \sum_{n=0}^\infty \left(\frac{1}{3}\right)^n = \frac{1}{1-1/3} = \frac{3}{2}.
$$

II. By definition,
$$
\sum_{n=0}^\infty a_n = \lim_{n \rightarrow \infty} s_n = 0.
$$
\end{freeResponse}

%%%%%%%%%%%%%%%%%%%%%%%%%%%%%%%%%%%%%%%%%%%%%%%

\begin{problem}
Let $\{a_n\}_{n=1}^\infty$ be a sequence and $\{s_n\}_{n=1}^\infty$ its sequence of partial sums. Fill in the missing values in the following table.

\begin{center}
\begin{tabular}{ |c|c|c| } 
 \hline
$n$ & $a_n$ & $s_n$ \\ 
 \hline
 \hline
  3 & 3 & 2 \\ 
  \hline
    4 & 5 &  \\ 
 \hline
   5 & 1 & 8 \\ 
 \hline
   6 &  & 10 \\ 
 \hline
\end{tabular}
\end{center}

\end{problem}

\begin{freeResponse}
We need to find $s_4$ and $a_6$. To find $s_4$, note that 
$$
s_4 = s_3 + a_4 = 2 + 5 = 7.
$$
Similarly, to find $a_6$, note that
$$
s_6 = s_5 + a_ 6,
$$
or 
$$
10 = 8 + a_6,
$$
whence $a_6 = 2$. 
\end{freeResponse}


%%%%%%%%%%%%%%%%%%%%%%%%%%%%%%%%%%%%%%%%%%%%%%%


\section{Group Work}

\begin{problem}
Let $\{a_n\}_{n=1}^\infty$ be a sequence with sequence of partial sums $\{s_n\}_{n=1}^\infty$ given explicitly by the formula
$$
s_n = \frac{n^3-5n^2}{(1+2n)^3}.
$$
Determine if the following series converge or diverge.  If they converge, find the value to which they converge.
\begin{center}
\begin{tabular}{ll}
I. $\sum_{n=1}^\infty a_n$ \hspace{.6in} II. $\sum_{n=4}^\infty a_n$ \hspace{.6in} 
\end{tabular}
\end{center}
\end{problem}

\begin{freeResponse}
I. By definition, 
$$
\sum_{n=1}^\infty a_n = \lim_{n\rightarrow \infty} s_n = \lim_{n\rightarrow \infty}\frac{n^3-5n^2}{(1+2n)^3} = \frac{1}{8},
$$
where we have computed the limit using the method of growth rate comparison.

II. We know that changing the lower index will not affect whether the series converges, but it might affect the value.  Note that the series here is related to the series from the last part since

\begin{image}
  \begin{tikzpicture}
        \node at (0,0) {
          $\underbrace{\sum_{k=1}^{\infty} a_k}=\overbrace{a_1+a_2+a_3}+ \underbrace{a_4+a_5 + \ldots}$};
        \node at (1.8,-.7) {\small{This is the series}};
        \node at (1.8,-1) {\small{we want to find.}};
        
        \node at (-2.5,-.9) {\small{This is $1/8$ from }};
        \node at (-2.5,-1.2) {\small{the last part.}};    
        
        \node at (-.6,.9) {\small{This is $s_3$. }};
        
      \end{tikzpicture}
  \end{image}
  
 Note that $s_3 = \frac{(3)^3-5(3)^2}{(1+2(3))^3} = -\frac{18}{343}$.
  
We thus have
\begin{align*}
\sum_{n=1}^\infty a_n &= s_3+\sum_{n=4}^\infty a_n  \\
 \frac{1}{8} &=  \frac{-18}{343}+\sum_{n=4}^\infty a_n \\
 \sum_{n=4}^\infty a_n   &=  \frac{1}{8} +\frac{18}{343}
\end{align*}

where the last two terms are computed using part I and the explicit formula for $s_n$, respectively.

\end{freeResponse}

%%%%%%%%%%%%%%%%%%%%%%%%%%%%%%%%%%%%%%%%%%%%%%%

\begin{problem}
Given that 
$$
\sum_{k=2}^\infty \frac{3}{k^2 + k} = \frac{3}{2},
$$
determine the value of 
$$
\sum_{k=4}^\infty \frac{9}{k^2 + k}.
$$
\end{problem}

\begin{freeResponse}
We have
$$
\sum_{k=4}^\infty \frac{3}{k^2 + k} = \sum_{k=2}^\infty \frac{3}{k^2 + k} - \sum_{k=2}^3 \frac{3}{k^2 + k} = \frac{3}{2} - \frac{3}{6} - \frac{3}{12} = \frac{5}{6},
$$
and it follows that 
$$
\sum_{k=4}^\infty \frac{9}{k^2 + k} = 3 \sum_{k=4}^\infty \frac{3}{k^2 + k} = 3 \cdot \frac{5}{6} = \frac{5}{2}.
$$
\end{freeResponse}

%%%%%%%%%%%%%%%%%%%%%%%%%%%%%%%%%%%%%%%%%%%%%%%

\begin{problem}
Determine whether each of the following series converges or diverges. If it converges, find its value.
\begin{center}
\begin{tabular}{lll}
I. $\sum_{n=0}^\infty 2^{1-2n}$ \hspace{.5in} II. $\sum_{n=0}^\infty \frac{5 \cdot 2^{n+3}}{3^{2n}}$ \hspace{.5in} III. $\sum_{n=0}^\infty 7 \cdot \left(\frac{5}{9}\right)^{2-3n}$ \hspace{.2in} %IV. $\sum_{k=-3}^\infty \left(\frac{1}{4}\right)^{k+1}$
\end{tabular}
\end{center}
\end{problem}

\begin{freeResponse}
The general strategy is to manipulate the terms in each sum to put them into the standard form for a geometric series.

I. The first series can be rewritten as 
$$
\sum_{n=0}^\infty 2^{1-2n} = \sum_{n=0}^\infty 2 \cdot 2^{-2n} = \sum_{n=0}^\infty 2 \cdot \left(\frac{1}{4}\right)^n = \frac{2}{1-1/4} = \frac{8}{3}.
$$

II. A similar approach works here, and we have
$$
\sum_{n=0}^\infty \frac{5 \cdot 2^{n+3}}{3^{2n}} = \sum_{n=0}^\infty \frac{5 \cdot 2^3 \cdot 2^n}{9^n} = \sum_{n=0}^\infty 40 \cdot \left(\frac{2}{9}\right)^n = \frac{40}{1-2/9} = \frac{360}{7}.
$$

III. Continuing with this approach, we have
$$
\sum_{n=0}^\infty 7 \cdot \left(\frac{5}{9}\right)^{2-3n} = \sum_{n=0}^\infty 7 \cdot \left(\frac{5}{9}\right)^2 \left(\frac{5}{9}\right)^{-3n} = \sum_{n=0}^\infty 7 \cdot \left(\frac{5}{9}\right)^2 \left(\frac{9^3}{5^3}\right)^{n}.
$$
In this case, the ratio $9^3/5^3$ is larger than $1$ and we conclude that this series diverges.

%IV. Finally, this example requires us to manipulate the series by separating off the piece which doesn't correspond to the standard indexing. That is, 
%\begin{align*}
%\sum_{k=-3}^\infty \left(\frac{1}{4}\right)^{k+1} &= \sum_{k=-3}^{-1} \left(\frac{1}{4}\right)^{k+1} + \sum_{k=0}^\infty \left(\frac{1}{4}\right)^{k+1} \\
%&= \sum_{k=-3}^{-1} \left(\frac{1}{4}\right)^{k+1} + \sum_{k=0}^\infty \frac{1}{4} \left(\frac{1}{4}\right)^{k} \\
%&= 4^2 + 4 + 1 + \frac{1/4}{1-1/4} = \frac{64}{3}.
%\end{align*}
\end{freeResponse}
\end{document}
