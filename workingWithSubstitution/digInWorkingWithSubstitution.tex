\documentclass{ximera}

%\usepackage{todonotes}
%\usepackage{mathtools} %% Required for wide table Curl and Greens
%\usepackage{cuted} %% Required for wide table Curl and Greens
\newcommand{\todo}{}

\usepackage{multicol}

\usepackage{esint} % for \oiint
\ifxake%%https://math.meta.stackexchange.com/questions/9973/how-do-you-render-a-closed-surface-double-integral
\renewcommand{\oiint}{{\large\bigcirc}\kern-1.56em\iint}
\fi

\graphicspath{
  {./}
  {ximeraTutorial/}
  {basicPhilosophy/}
  {functionsOfSeveralVariables/}
  {normalVectors/}
  {lagrangeMultipliers/}
  {vectorFields/}
  {greensTheorem/}
  {shapeOfThingsToCome/}
  {dotProducts/}
  {partialDerivativesAndTheGradientVector/}
  {../ximeraTutorial/}
  {../productAndQuotientRules/exercises/}
  {../motionAndPathsInSpace/exercises/}
  {../normalVectors/exercisesParametricPlots/}
  {../continuityOfFunctionsOfSeveralVariables/exercises/}
  {../partialDerivativesAndTheGradientVector/exercises/}
  {../directionalDerivativeAndChainRule/exercises/}
  {../commonCoordinates/exercisesCylindricalCoordinates/}
  {../commonCoordinates/exercisesSphericalCoordinates/}
  {../greensTheorem/exercisesCurlAndLineIntegrals/}
  {../greensTheorem/exercisesDivergenceAndLineIntegrals/}
  {../shapeOfThingsToCome/exercisesDivergenceTheorem/}
  {../greensTheorem/}
  {../shapeOfThingsToCome/}
  {../separableDifferentialEquations/exercises/}
  {../dotproducts/}
  {../functionsOfSeveralVariables/}
  {../lagrangeMultipliers/}
  {../partialDerivativesAndTheGradientVector/}
  {../normalVectors/}
  {../vectorFields/}
}
\def\xmNotExpandableAsAccordion{true}

\newcommand{\mooculus}{\textsf{\textbf{MOOC}\textnormal{\textsf{ULUS}}}}

\usepackage{tkz-euclide}\usepackage{tikz}
\usepackage{tikz-cd}
\usetikzlibrary{arrows}
\tikzset{>=stealth,commutative diagrams/.cd,
  arrow style=tikz,diagrams={>=stealth}} %% cool arrow head
\tikzset{shorten <>/.style={ shorten >=#1, shorten <=#1 } } %% allows shorter vectors

\usetikzlibrary{backgrounds} %% for boxes around graphs
\usetikzlibrary{shapes,positioning}  %% Clouds and stars
\usetikzlibrary{matrix} %% for matrix
\usepgfplotslibrary{polar} %% for polar plots
\usepgfplotslibrary{fillbetween} %% to shade area between curves in TikZ
%\usetkzobj{all} %% obsolete

\usepackage[makeroom]{cancel} %% for strike outs
%\usepackage{mathtools} %% for pretty underbrace % Breaks Ximera
%\usepackage{multicol}
\usepackage{pgffor} %% required for integral for loops

\usepackage{tkz-tab}  %% for sign charts

%% http://tex.stackexchange.com/questions/66490/drawing-a-tikz-arc-specifying-the-center
%% Draws beach ball
\tikzset{pics/carc/.style args={#1:#2:#3}{code={\draw[pic actions] (#1:#3) arc(#1:#2:#3);}}}



\usepackage{array}
\setlength{\extrarowheight}{+.1cm}
\newdimen\digitwidth
\settowidth\digitwidth{9}
\def\divrule#1#2{
\noalign{\moveright#1\digitwidth
\vbox{\hrule width#2\digitwidth}}}





\newcommand{\RR}{\mathbb R}
\newcommand{\R}{\mathbb R}
\newcommand{\N}{\mathbb N}
\newcommand{\Z}{\mathbb Z}

\newcommand{\sagemath}{\textsf{SageMath}}


\renewcommand{\d}{\,d}
%\def\d{\mathop{}\!d}
%\def\d{\,d}

\AddToHook{begindocument}{%
  \renewcommand{\d}{\,d}     % lualatex redefines \d to underdot !!!
}

\pgfplotsset{
    every axis/.style={
        scale only axis,
        enlargelimits=false,
        trim axis left,
        trim axis right,
        clip=true,
    }
}

\newcommand{\dd}[2][]{\frac{\d #1}{\d #2}}
\newcommand{\pp}[2][]{\frac{\partial #1}{\partial #2}}
\renewcommand{\l}{\ell}
\newcommand{\ddx}{\frac{d}{\d x}}

\newcommand{\zeroOverZero}{\ensuremath{\boldsymbol{\tfrac{0}{0}}}}
\newcommand{\inftyOverInfty}{\ensuremath{\boldsymbol{\tfrac{\infty}{\infty}}}}
\newcommand{\zeroOverInfty}{\ensuremath{\boldsymbol{\tfrac{0}{\infty}}}}
\newcommand{\zeroTimesInfty}{\ensuremath{\small\boldsymbol{0\cdot \infty}}}
\newcommand{\inftyMinusInfty}{\ensuremath{\small\boldsymbol{\infty - \infty}}}
\newcommand{\oneToInfty}{\ensuremath{\boldsymbol{1^\infty}}}
\newcommand{\zeroToZero}{\ensuremath{\boldsymbol{0^0}}}
\newcommand{\inftyToZero}{\ensuremath{\boldsymbol{\infty^0}}}



\newcommand{\numOverZero}{\ensuremath{\boldsymbol{\tfrac{\#}{0}}}}
\newcommand{\dfn}{\textbf}
%\newcommand{\unit}{\,\mathrm}
\newcommand{\unit}{\mathop{}\!\mathrm}
\newcommand{\eval}[1]{\bigg[ #1 \bigg]}
\newcommand{\seq}[1]{\left( #1 \right)}
\renewcommand{\epsilon}{\varepsilon}
\renewcommand{\phi}{\varphi}


\renewcommand{\iff}{\Leftrightarrow}

\DeclareMathOperator{\arccot}{arccot}
\DeclareMathOperator{\arcsec}{arcsec}
\DeclareMathOperator{\arccsc}{arccsc}
\DeclareMathOperator{\si}{Si}
\DeclareMathOperator{\scal}{scal}
\DeclareMathOperator{\sign}{sign}


%% \newcommand{\tightoverset}[2]{% for arrow vec
%%   \mathop{#2}\limits^{\vbox to -.5ex{\kern-0.75ex\hbox{$#1$}\vss}}}
\newcommand{\arrowvec}[1]{{\overset{\rightharpoonup}{#1}}}
%\renewcommand{\vec}[1]{\arrowvec{\mathbf{#1}}}
\renewcommand{\vec}[1]{{\overset{\boldsymbol{\rightharpoonup}}{\mathbf{#1}}}\hspace{0in}}

\newcommand{\point}[1]{\left(#1\right)} %this allows \vector{ to be changed to \vector{ with a quick find and replace
\newcommand{\pt}[1]{\mathbf{#1}} %this allows \vec{ to be changed to \vec{ with a quick find and replace
\newcommand{\Lim}[2]{\lim_{\point{#1} \to \point{#2}}} %Bart, I changed this to point since I want to use it.  It runs through both of the exercise and exerciseE files in limits section, which is why it was in each document to start with.

\DeclareMathOperator{\proj}{\mathbf{proj}}
\newcommand{\veci}{{\boldsymbol{\hat{\imath}}}}
\newcommand{\vecj}{{\boldsymbol{\hat{\jmath}}}}
\newcommand{\veck}{{\boldsymbol{\hat{k}}}}
\newcommand{\vecl}{\vec{\boldsymbol{\l}}}
\newcommand{\uvec}[1]{\mathbf{\hat{#1}}}
\newcommand{\utan}{\mathbf{\hat{t}}}
\newcommand{\unormal}{\mathbf{\hat{n}}}
\newcommand{\ubinormal}{\mathbf{\hat{b}}}

\newcommand{\dotp}{\bullet}
\newcommand{\cross}{\boldsymbol\times}
\newcommand{\grad}{\boldsymbol\nabla}
\newcommand{\divergence}{\grad\dotp}
\newcommand{\curl}{\grad\cross}
%\DeclareMathOperator{\divergence}{divergence}
%\DeclareMathOperator{\curl}[1]{\grad\cross #1}
\newcommand{\lto}{\mathop{\longrightarrow\,}\limits}

\renewcommand{\bar}{\overline}

\colorlet{textColor}{black}
\colorlet{background}{white}
\colorlet{penColor}{blue!50!black} % Color of a curve in a plot
\colorlet{penColor2}{red!50!black}% Color of a curve in a plot
\colorlet{penColor3}{red!50!blue} % Color of a curve in a plot
\colorlet{penColor4}{green!50!black} % Color of a curve in a plot
\colorlet{penColor5}{orange!80!black} % Color of a curve in a plot
\colorlet{penColor6}{yellow!70!black} % Color of a curve in a plot
\colorlet{fill1}{penColor!20} % Color of fill in a plot
\colorlet{fill2}{penColor2!20} % Color of fill in a plot
\colorlet{fillp}{fill1} % Color of positive area
\colorlet{filln}{penColor2!20} % Color of negative area
\colorlet{fill3}{penColor3!20} % Fill
\colorlet{fill4}{penColor4!20} % Fill
\colorlet{fill5}{penColor5!20} % Fill
\colorlet{gridColor}{gray!50} % Color of grid in a plot

\newcommand{\surfaceColor}{violet}
\newcommand{\surfaceColorTwo}{redyellow}
\newcommand{\sliceColor}{greenyellow}




\pgfmathdeclarefunction{gauss}{2}{% gives gaussian
  \pgfmathparse{1/(#2*sqrt(2*pi))*exp(-((x-#1)^2)/(2*#2^2))}%
}


%%%%%%%%%%%%%
%% Vectors
%%%%%%%%%%%%%

%% Simple horiz vectors
\renewcommand{\vector}[1]{\left\langle #1\right\rangle}


%% %% Complex Horiz Vectors with angle brackets
%% \makeatletter
%% \renewcommand{\vector}[2][ , ]{\left\langle%
%%   \def\nextitem{\def\nextitem{#1}}%
%%   \@for \el:=#2\do{\nextitem\el}\right\rangle%
%% }
%% \makeatother

%% %% Vertical Vectors
%% \def\vector#1{\begin{bmatrix}\vecListA#1,,\end{bmatrix}}
%% \def\vecListA#1,{\if,#1,\else #1\cr \expandafter \vecListA \fi}

%%%%%%%%%%%%%
%% End of vectors
%%%%%%%%%%%%%

%\newcommand{\fullwidth}{}
%\newcommand{\normalwidth}{}



%% makes a snazzy t-chart for evaluating functions
%\newenvironment{tchart}{\rowcolors{2}{}{background!90!textColor}\array}{\endarray}

%%This is to help with formatting on future title pages.
\newenvironment{sectionOutcomes}{}{}



%% Flowchart stuff
%\tikzstyle{startstop} = [rectangle, rounded corners, minimum width=3cm, minimum height=1cm,text centered, draw=black]
%\tikzstyle{question} = [rectangle, minimum width=3cm, minimum height=1cm, text centered, draw=black]
%\tikzstyle{decision} = [trapezium, trapezium left angle=70, trapezium right angle=110, minimum width=3cm, minimum height=1cm, text centered, draw=black]
%\tikzstyle{question} = [rectangle, rounded corners, minimum width=3cm, minimum height=1cm,text centered, draw=black]
%\tikzstyle{process} = [rectangle, minimum width=3cm, minimum height=1cm, text centered, draw=black]
%\tikzstyle{decision} = [trapezium, trapezium left angle=70, trapezium right angle=110, minimum width=3cm, minimum height=1cm, text centered, draw=black]


\outcome{To be able to use the method of substitution to solve more difficult types of integrals.}
\outcome{To be able to both correctly identify what to substitute for and to be able to successfully carry out the process to correctly solve the problem.}

\title[Dig-In:]{Working with substitution}

\begin{document}
\begin{abstract}
We explore more difficult problems involving substitution.
\end{abstract}
\maketitle

We begin by restating the substitution formula.

\begin{theorem}[Integral Substitution Formula] 
If $g$ is differentiable on the interval $[a,b]$ and $f$ is
differentiable on the interval $[g(a),g(b)]$, and if they have continuous derivatives, then
\[
\int_a^b f'(g(x)) g'(x) \d x =\int_{g(a)}^{g(b)} f'(u) \d u.
\]
\end{theorem}


We will work out a few more examples.


\begin{example}
Compute:
\[
\int_{2}^{3} \frac{1}{x\ln(x)} \d x
\]
\begin{explanation}
  Let
  \[
  u =\answer[given]{\ln(x)},
  \]
  computing $\d u$, we find
  \[
  \d u =\answer[given]{\frac{1}{x}}\d x
  \]
  and solving for $\d x$ we find
  \[
  \d x = \answer[given]{x} \d u.
  \]
  Now
\begin{align*}
\int_{2}^{3} \frac{1}{x\ln(x)} \d x &= \int_{g(2)}^{g(3)} \frac{1}{x \cdot u} x \d u  \\
&= \int_{\answer[given]{\ln(2)}}^{\answer[given]{\ln(3)}} \answer[given]{\frac{1}{u}} \d u\\
&= \eval{\answer[given]{\ln(u)}}_{\answer[given]{\ln(2)}}^{\answer[given]{\ln(3)}}\\
& = \ln(\ln(3)) - \ln(\ln(2)).
\end{align*}
\end{explanation}
\end{example}


The next example requires a new technique.

\begin{example}
  Compute:
\[
\int x^3\sqrt{1-x^2}\d x
\]
\begin{explanation}
Here it is not apparent that the chain rule is involved. However, if
it was involved, perhaps a good guess for $g$ would be
\[
u = \answer[given]{1-x^2}
\]
and then
\begin{align*}
  \d u &= \answer[given]{-2x} \d x, \\
  \d x &= \answer[given]{\frac{-1}{2x}} \d u.
\end{align*}
Now we consider the integral we are trying to compute
\[
\int x^3\sqrt{1-x^2}\d x
\]
and we substitute using our work above. Write with me
\begin{align*}
  \int x^3\sqrt{1-x^2}\d x &= \int x^3 \answer[given]{\sqrt{u}} \left( \answer[given]{\frac{-1}{2x}} \right) \d u \\
  &= \int \frac{-x^2 \answer[given]{\sqrt{u}}}{2}\d u.
\end{align*}
However, we cannot continue until each $x$ is replaced. We know that
\begin{align*}
u &= 1-x^2 \\
\Rightarrow \qquad u -1 &= -x^2\\
\Rightarrow \qquad \answer[given]{1- u} &= x^2
\end{align*}
so now we may replace $x^2$
\[
\int x^3\sqrt{1-x^2}\d x = \int -\frac{\answer[given]{(1-u)} \sqrt{u}}{2}\d u.
\]
At this point, we are close to being done. Write
\begin{align*}
\int -\frac{(1-u)\sqrt{u}}{2}\d u &= \int \left(\frac{u\sqrt{u}}{2} - \frac{\sqrt{u}}{2}\right) \d u \\
&= \int \frac{\answer[given]{u^{3/2}}}{2} \d u - \int \frac{\sqrt{u}}{2} \d u \\
&= \answer[given]{\frac{u^{5/2}}{5}} - \answer[given]{\frac{u^{3/2}}{3}}.
\end{align*}
Now recall that $u = 1-x^2$. Hence our final answer is
\[
\int x^3\sqrt{1-x^2}\d x = \answer[given]{\frac{(1-x^2)^{5/2}}{5}} - \frac{(1-x^2)^{3/2}}{3}+C.
\]
\end{explanation}
\end{example}


Sometimes it is not obvious how a fraction could have been obtained
using the chain rule.  A common trick though is to substitute for the
\textit{denominator} of a fraction.  Like all tricks, this technique
does not always work. Regardless the next two examples present how
this technique can be used.


\begin{example}
Compute:
\[
\int \frac{\sec(y) \tan(y) + \sec^2(y)}{\sec(y) + \tan(y)} \d y
\]
\begin{explanation}
We substitute
\[
u = \sec(y) + \tan(y)
\]
and we immediately see that
\begin{align*}
  \d u &= \answer[given]{(\sec(y) \tan(y) + \sec^2(y))} \d y,\\
  \d y &= \frac{1}{\answer[given]{\sec(y) \tan(y) + \sec^2(y)}} \d u.
\end{align*}
But this cancels perfectly with the numerator!  
So we have that
\begin{align*}
\int \frac{\sec(y) \tan(y) + \sec^2(y)}{\sec(y) + \tan(y)} \d y &= \int \frac{1}{u} \d u  \\
&= \ln(|u|) + C  \\
&= \ln(| \sec(y) + \tan(y) |) + C.
\end{align*}
\end{explanation}
\end{example}



Notice that 
\[
\frac{\sec(x) \tan(x) + \sec^2(x)}{\sec(x) + \tan(x)} = \frac{\sec(x) (\tan(x) + \sec(x))}{\sec(x) + \tan(x)} = \sec(x)
\]
when $\sec(x) \neq - \tan(x)$.  
So in a very contrived way, we have just proved

\begin{theorem}
\[
\int \sec(x) \d x = \ln(|\sec(x) + \tan(x)|) + C.
\]
\end{theorem}


Notice the variable in this next example.


\begin{example}\label{key example}
Compute:
\[
\int \frac{u}{1-u^2} \d u
\]
\begin{explanation}
We want to substitute for $1-u^2$.  
But the variable ``$u$'' has already been used\dots OH NO!
Never fear! We can substitute with whatever variable that we want.  
In particular, let us use ``$w$'' for this problem.  
So we let
\[
w = 1 - u^2
\]
and then
\begin{align*}
  \d w &= \answer[given]{-2u} \d u,\\
  \d u &= \answer[given]{\frac{-1}{2u}} \d w.
\end{align*}
Thus
\begin{align*}
\int \frac{u}{1-u^2} \d u &= \int \frac{u}{w} \left( \answer[given]{\frac{-1}{2u}} \right) \d w  \\
&= - \frac{1}{2} \int \frac{1}{w} \d w  \\
&= - \frac{1}{2} \ln(|w|) + C  \\
&= - \frac{1}{2} \ln(|1-u^2|) + C.
\end{align*}
\end{explanation}
Recall:
\[
\int_{g(a)}^{g(b)} f'(u) \d u=\int_{g(a)}^{g(b)} f'(w) \d w=\int_{g(a)}^{g(b)} f'(s) \d s.
\]
So, we can call a variable $u$ by any other name and ``it would smell as sweet".

\end{example}




%This last example is a little more involved, and instead we will state the problem as a Theorem.
%The example is really the proof of the Theorem.

\begin{example}\label{example tan}
Compute:
\[
\int \tan(x) \d x
\]
\begin{explanation}
We begin by writing
\begin{align*}
\tan(x) &= \frac{\sin(x)}{\cos(x)}  \\
&= \frac{\sin(x)}{\cos(x)} \cdot \frac{\cos(x)}{\cos(x)}  \\
&= \frac{\sin(x) \cos(x)}{1 - \sin^2(x)}.
\end{align*}
We then make the substitution
\[
s = \sin(x)
\]
and so
\begin{align*}
\d s &= \answer[given]{\cos(x)} \d x,\\
\d x &= \frac{1}{\answer[given]{\cos(x)}} \d s.
\end{align*}
Then
\begin{align*}
\int \tan(x) \d x &= \int \frac{\sin(x) \cos(x)}{1 - \sin^2(x)} \d x  \\
&= \int \frac{s \cos(x)}{1 - s^2} \cdot \frac{1}{\answer[given]{\cos(x)}} \d s  \\
&= \int \frac{s}{1-s^2} \d s.
\end{align*}
But this is the same problem as Example \ref{key example}!  
And so we know that
\begin{align*}
\int \tan(x) \d x &= - \frac{1}{2} \ln(|1-s^2|) + C  \\
&= - \frac{1}{2} \ln(|1-\sin^2 (x)|) + C  \\
&= - \frac{1}{2} \ln(|\cos^2(x)|) + C  \\
&= \ln\left(|\cos^2(x)|^{- \frac{1}{2}}\right) + C  \\
&= \ln(|\sec(x)|) + C.
\end{align*}
\end{explanation}
\end{example}

We have just proved

\begin{theorem}
\[
\int \tan(x) \d x = \ln(|\sec(x)|) + C.
\]
\end{theorem}

Note that in Example \ref{example tan}, we could have instead made the substitution
\[
u = 1-\sin^2(x).
\]
This would have gotten us to the answer quicker and without using Example \ref{key example}.  
You are encouraged to work this out on your own right now!

We end this section with two more difficult examples.



\begin{example}
Compute:
\[
\int \frac{e^{2x}}{1 - e^{2x}} \d x
\]
\begin{explanation}
Maybe the biggest key to solving this problem is to recall that
\[
e^{2x} = (\answer[given]{e^x})^2.
\]
So we can rewrite the problem
\[
\int \frac{e^{2x}}{1 - e^{2x}} \d x = \int \frac{(\answer[given]{e^x})^2}{1 - (\answer[given]{e^x})^2} \d x.
\]
Now, if we make the substitution $u = e^x$, we have that
\begin{align*}
\d u &= \answer[given]{e^x} \d x,\\
\d x &= \frac{1}{\answer[given]{e^x}} \d u,
\end{align*}
and
\begin{align*}
\int \frac{e^{2x}}{1 - e^{2x}} \d x &= \int \frac{(e^x)^2}{1 - u^2} \cdot \frac{1}{\answer[given]{e^x}} \d u  \\
&= \int \frac{e^x}{1-u^2} \d u  \\
&= \int \frac{u}{1-u^2} \d u.
\end{align*}
But now we are back to Example \ref{key example}, and so we know that
\begin{align*}
\int \frac{e^{2x}}{1 - e^{2x}} \d x &= - \frac{1}{2} \ln(|1-u^2|) + C  \\
&= - \frac{1}{2} \ln(|1 - e^{2x}|) + C.
\end{align*}
\end{explanation}
\end{example}


Again, in the previous example we could have instead made the substitution 
\[
u = 1 - e^{2x}
\]
and avoided using Example \ref{key example}.  
In general, any time that you make two successive substitutions in a problem, you could have instead just made one substitution.  
This one substitution is the \textit{composition} of the two original substitutions.  
But sometimes it may not be obvious to make one clever substitution, and so two substitutions makes more sense.  
The next example helps to demonstrate this.


\begin{example}
Compute:
\[
\int_0^{16} \sqrt{4 - \sqrt{x}} \d x
\]
\begin{explanation}
While it is not obvious at all, let us try the substitution
\[
u = \sqrt{x}.
\]
Then
\begin{align*}
\d u &= \answer[given]{\frac{1}{2 \sqrt{x}}} \d x,\\
\d x &= 2 \sqrt{x} \d u = 2u \d u,
\end{align*}
and so
\begin{align*}
\int_0^{16} \sqrt{4 - \sqrt{x}} \d x &= \int_{g(0)}^{g(16)} \answer[given]{\sqrt{4-u} \cdot 2u} \d u  \\
&= \int_{\answer[given]{0}}^{\answer[given]{4}} \answer[given]{2u \sqrt{4-u}} \d u.
\end{align*}
From here we now make the second (and more obvious) substitution
\[
w = 4-u.
\]
Then $u = 4-w$, and
\begin{align*}
\d w &= - \d u,\\
\d u &= - \d w.
\end{align*}
So
\begin{align*}
\int_0^{16} \sqrt{4 - \sqrt{x}} \d x &= \int_0^4 2u \sqrt{4-u} \d u  \\
&= \int_{w(0)}^{w(4)} 2 (4-w) \sqrt{w} (-1) \d w  \\
&= - \int_{\answer[given]{4}}^{\answer[given]{0}} ( 8w^{\frac{1}{2}} - 2w^{\frac{3}{2}} ) \d w  \\
&= \int_{\answer[given]{0}}^{\answer[given]{4}} ( 8w^{\frac{1}{2}} - 2w^{\frac{3}{2}} ) \d w  \\
&= \eval{ \answer[given]{8 \left( \frac{2}{3} \right) w^{\frac{3}{2}} - 2 \left( \frac{2}{5} \right) w^{\frac{5}{2}}} }_{0}^{4}  \\
&= \left( \frac{16}{3} (4)^{\frac{3}{2}} - \frac{4}{5} (4)^{\frac{5}{2}} \right) - \left( 0 - 0 \right)  \\
&= \frac{128}{3} - \frac{128}{5}   \\
&= \frac{256}{15}.
\end{align*}
\end{explanation}
\end{example}
\end{document}
