\documentclass{ximera}

%\usepackage{todonotes}
%\usepackage{mathtools} %% Required for wide table Curl and Greens
%\usepackage{cuted} %% Required for wide table Curl and Greens
\newcommand{\todo}{}

\usepackage{multicol}

\usepackage{esint} % for \oiint
\ifxake%%https://math.meta.stackexchange.com/questions/9973/how-do-you-render-a-closed-surface-double-integral
\renewcommand{\oiint}{{\large\bigcirc}\kern-1.56em\iint}
\fi

\graphicspath{
  {./}
  {ximeraTutorial/}
  {basicPhilosophy/}
  {functionsOfSeveralVariables/}
  {normalVectors/}
  {lagrangeMultipliers/}
  {vectorFields/}
  {greensTheorem/}
  {shapeOfThingsToCome/}
  {dotProducts/}
  {partialDerivativesAndTheGradientVector/}
  {../ximeraTutorial/}
  {../productAndQuotientRules/exercises/}
  {../motionAndPathsInSpace/exercises/}
  {../normalVectors/exercisesParametricPlots/}
  {../continuityOfFunctionsOfSeveralVariables/exercises/}
  {../partialDerivativesAndTheGradientVector/exercises/}
  {../directionalDerivativeAndChainRule/exercises/}
  {../commonCoordinates/exercisesCylindricalCoordinates/}
  {../commonCoordinates/exercisesSphericalCoordinates/}
  {../greensTheorem/exercisesCurlAndLineIntegrals/}
  {../greensTheorem/exercisesDivergenceAndLineIntegrals/}
  {../shapeOfThingsToCome/exercisesDivergenceTheorem/}
  {../greensTheorem/}
  {../shapeOfThingsToCome/}
  {../separableDifferentialEquations/exercises/}
  {../dotproducts/}
  {../functionsOfSeveralVariables/}
  {../lagrangeMultipliers/}
  {../partialDerivativesAndTheGradientVector/}
  {../normalVectors/}
  {../vectorFields/}
}
\def\xmNotExpandableAsAccordion{true}

\newcommand{\mooculus}{\textsf{\textbf{MOOC}\textnormal{\textsf{ULUS}}}}

\usepackage{tkz-euclide}\usepackage{tikz}
\usepackage{tikz-cd}
\usetikzlibrary{arrows}
\tikzset{>=stealth,commutative diagrams/.cd,
  arrow style=tikz,diagrams={>=stealth}} %% cool arrow head
\tikzset{shorten <>/.style={ shorten >=#1, shorten <=#1 } } %% allows shorter vectors

\usetikzlibrary{backgrounds} %% for boxes around graphs
\usetikzlibrary{shapes,positioning}  %% Clouds and stars
\usetikzlibrary{matrix} %% for matrix
\usepgfplotslibrary{polar} %% for polar plots
\usepgfplotslibrary{fillbetween} %% to shade area between curves in TikZ
%\usetkzobj{all} %% obsolete

\usepackage[makeroom]{cancel} %% for strike outs
%\usepackage{mathtools} %% for pretty underbrace % Breaks Ximera
%\usepackage{multicol}
\usepackage{pgffor} %% required for integral for loops

\usepackage{tkz-tab}  %% for sign charts

%% http://tex.stackexchange.com/questions/66490/drawing-a-tikz-arc-specifying-the-center
%% Draws beach ball
\tikzset{pics/carc/.style args={#1:#2:#3}{code={\draw[pic actions] (#1:#3) arc(#1:#2:#3);}}}



\usepackage{array}
\setlength{\extrarowheight}{+.1cm}
\newdimen\digitwidth
\settowidth\digitwidth{9}
\def\divrule#1#2{
\noalign{\moveright#1\digitwidth
\vbox{\hrule width#2\digitwidth}}}





\newcommand{\RR}{\mathbb R}
\newcommand{\R}{\mathbb R}
\newcommand{\N}{\mathbb N}
\newcommand{\Z}{\mathbb Z}

\newcommand{\sagemath}{\textsf{SageMath}}


\renewcommand{\d}{\,d}
%\def\d{\mathop{}\!d}
%\def\d{\,d}

\AddToHook{begindocument}{%
  \renewcommand{\d}{\,d}     % lualatex redefines \d to underdot !!!
}

\pgfplotsset{
    every axis/.style={
        scale only axis,
        enlargelimits=false,
        trim axis left,
        trim axis right,
        clip=true,
    }
}

\newcommand{\dd}[2][]{\frac{\d #1}{\d #2}}
\newcommand{\pp}[2][]{\frac{\partial #1}{\partial #2}}
\renewcommand{\l}{\ell}
\newcommand{\ddx}{\frac{d}{\d x}}

\newcommand{\zeroOverZero}{\ensuremath{\boldsymbol{\tfrac{0}{0}}}}
\newcommand{\inftyOverInfty}{\ensuremath{\boldsymbol{\tfrac{\infty}{\infty}}}}
\newcommand{\zeroOverInfty}{\ensuremath{\boldsymbol{\tfrac{0}{\infty}}}}
\newcommand{\zeroTimesInfty}{\ensuremath{\small\boldsymbol{0\cdot \infty}}}
\newcommand{\inftyMinusInfty}{\ensuremath{\small\boldsymbol{\infty - \infty}}}
\newcommand{\oneToInfty}{\ensuremath{\boldsymbol{1^\infty}}}
\newcommand{\zeroToZero}{\ensuremath{\boldsymbol{0^0}}}
\newcommand{\inftyToZero}{\ensuremath{\boldsymbol{\infty^0}}}



\newcommand{\numOverZero}{\ensuremath{\boldsymbol{\tfrac{\#}{0}}}}
\newcommand{\dfn}{\textbf}
%\newcommand{\unit}{\,\mathrm}
\newcommand{\unit}{\mathop{}\!\mathrm}
\newcommand{\eval}[1]{\bigg[ #1 \bigg]}
\newcommand{\seq}[1]{\left( #1 \right)}
\renewcommand{\epsilon}{\varepsilon}
\renewcommand{\phi}{\varphi}


\renewcommand{\iff}{\Leftrightarrow}

\DeclareMathOperator{\arccot}{arccot}
\DeclareMathOperator{\arcsec}{arcsec}
\DeclareMathOperator{\arccsc}{arccsc}
\DeclareMathOperator{\si}{Si}
\DeclareMathOperator{\scal}{scal}
\DeclareMathOperator{\sign}{sign}


%% \newcommand{\tightoverset}[2]{% for arrow vec
%%   \mathop{#2}\limits^{\vbox to -.5ex{\kern-0.75ex\hbox{$#1$}\vss}}}
\newcommand{\arrowvec}[1]{{\overset{\rightharpoonup}{#1}}}
%\renewcommand{\vec}[1]{\arrowvec{\mathbf{#1}}}
\renewcommand{\vec}[1]{{\overset{\boldsymbol{\rightharpoonup}}{\mathbf{#1}}}\hspace{0in}}

\newcommand{\point}[1]{\left(#1\right)} %this allows \vector{ to be changed to \vector{ with a quick find and replace
\newcommand{\pt}[1]{\mathbf{#1}} %this allows \vec{ to be changed to \vec{ with a quick find and replace
\newcommand{\Lim}[2]{\lim_{\point{#1} \to \point{#2}}} %Bart, I changed this to point since I want to use it.  It runs through both of the exercise and exerciseE files in limits section, which is why it was in each document to start with.

\DeclareMathOperator{\proj}{\mathbf{proj}}
\newcommand{\veci}{{\boldsymbol{\hat{\imath}}}}
\newcommand{\vecj}{{\boldsymbol{\hat{\jmath}}}}
\newcommand{\veck}{{\boldsymbol{\hat{k}}}}
\newcommand{\vecl}{\vec{\boldsymbol{\l}}}
\newcommand{\uvec}[1]{\mathbf{\hat{#1}}}
\newcommand{\utan}{\mathbf{\hat{t}}}
\newcommand{\unormal}{\mathbf{\hat{n}}}
\newcommand{\ubinormal}{\mathbf{\hat{b}}}

\newcommand{\dotp}{\bullet}
\newcommand{\cross}{\boldsymbol\times}
\newcommand{\grad}{\boldsymbol\nabla}
\newcommand{\divergence}{\grad\dotp}
\newcommand{\curl}{\grad\cross}
%\DeclareMathOperator{\divergence}{divergence}
%\DeclareMathOperator{\curl}[1]{\grad\cross #1}
\newcommand{\lto}{\mathop{\longrightarrow\,}\limits}

\renewcommand{\bar}{\overline}

\colorlet{textColor}{black}
\colorlet{background}{white}
\colorlet{penColor}{blue!50!black} % Color of a curve in a plot
\colorlet{penColor2}{red!50!black}% Color of a curve in a plot
\colorlet{penColor3}{red!50!blue} % Color of a curve in a plot
\colorlet{penColor4}{green!50!black} % Color of a curve in a plot
\colorlet{penColor5}{orange!80!black} % Color of a curve in a plot
\colorlet{penColor6}{yellow!70!black} % Color of a curve in a plot
\colorlet{fill1}{penColor!20} % Color of fill in a plot
\colorlet{fill2}{penColor2!20} % Color of fill in a plot
\colorlet{fillp}{fill1} % Color of positive area
\colorlet{filln}{penColor2!20} % Color of negative area
\colorlet{fill3}{penColor3!20} % Fill
\colorlet{fill4}{penColor4!20} % Fill
\colorlet{fill5}{penColor5!20} % Fill
\colorlet{gridColor}{gray!50} % Color of grid in a plot

\newcommand{\surfaceColor}{violet}
\newcommand{\surfaceColorTwo}{redyellow}
\newcommand{\sliceColor}{greenyellow}




\pgfmathdeclarefunction{gauss}{2}{% gives gaussian
  \pgfmathparse{1/(#2*sqrt(2*pi))*exp(-((x-#1)^2)/(2*#2^2))}%
}


%%%%%%%%%%%%%
%% Vectors
%%%%%%%%%%%%%

%% Simple horiz vectors
\renewcommand{\vector}[1]{\left\langle #1\right\rangle}


%% %% Complex Horiz Vectors with angle brackets
%% \makeatletter
%% \renewcommand{\vector}[2][ , ]{\left\langle%
%%   \def\nextitem{\def\nextitem{#1}}%
%%   \@for \el:=#2\do{\nextitem\el}\right\rangle%
%% }
%% \makeatother

%% %% Vertical Vectors
%% \def\vector#1{\begin{bmatrix}\vecListA#1,,\end{bmatrix}}
%% \def\vecListA#1,{\if,#1,\else #1\cr \expandafter \vecListA \fi}

%%%%%%%%%%%%%
%% End of vectors
%%%%%%%%%%%%%

%\newcommand{\fullwidth}{}
%\newcommand{\normalwidth}{}



%% makes a snazzy t-chart for evaluating functions
%\newenvironment{tchart}{\rowcolors{2}{}{background!90!textColor}\array}{\endarray}

%%This is to help with formatting on future title pages.
\newenvironment{sectionOutcomes}{}{}



%% Flowchart stuff
%\tikzstyle{startstop} = [rectangle, rounded corners, minimum width=3cm, minimum height=1cm,text centered, draw=black]
%\tikzstyle{question} = [rectangle, minimum width=3cm, minimum height=1cm, text centered, draw=black]
%\tikzstyle{decision} = [trapezium, trapezium left angle=70, trapezium right angle=110, minimum width=3cm, minimum height=1cm, text centered, draw=black]
%\tikzstyle{question} = [rectangle, rounded corners, minimum width=3cm, minimum height=1cm,text centered, draw=black]
%\tikzstyle{process} = [rectangle, minimum width=3cm, minimum height=1cm, text centered, draw=black]
%\tikzstyle{decision} = [trapezium, trapezium left angle=70, trapezium right angle=110, minimum width=3cm, minimum height=1cm, text centered, draw=black]


\outcome{View the chain rule in terms of the gradient.}

\outcome{Compute the derivative of the compostion of a function of
  several variables with a vector-valued function.}

\outcome{Use the chain rule of several variables to implicitly differentiate a curve.}

\title[Dig-In:]{The chain rule}

\begin{document}
\begin{abstract}
  We investigate the chain rule for functions of several variables.
\end{abstract}
\maketitle

The chain rule states that
\[
\ddx\Big(f\big(g(x)\big)\Big) = f'\big(g(x)\big)g'(x).
\]
If $t=g(x)$, we can express the chain rule as
\[
\dd[f]{x} = \dd[f]{t}\dd[t]{x}.
\]
In this section we extend the chain rule to functions of more than one
variable.

\begin{theorem}
  Let $F:\R^n\to\R$ be a differentiable function and let
  \[
  \vec{x}(t) = \vector{x_1(t),x_2(t),\dots,x_n(t)}
  \]
  be a differentiable vector-valued function from $\R\to\R^n$. Then
  \[
  \dd[F]{t} = \grad F(\vec{x}(t)) \dotp \vec{x}'(t) 
  \]
\end{theorem}

It is good to understand what the situation of $F(x,y)$, $\vec{x}(t) =
\vector{x(t),y(t)}$ describes. We know that $F(x,y)$ describes a
surface; we also recognize that $\vec{x}(t)$ describes a curve in the
$(x,y)$-plane. Combining these together, we are describing a curve
that lies on the surface described by $F$. The parametric equations
for this curve are $x=x(t)$, $y=y(t)$ and $F\big(x(t),y(t)\big)$.
Consider:
\begin{image}
  \begin{tikzpicture}
    \begin{axis}%
      [
        tick label style={font=\scriptsize},axis on top,
	axis lines=center,
	view={145}{35},
	name=myplot,
	%xtick={1,2,3,4},
	%ytick={1,2,3,4,5,6},
	%ztick=\empty,
	%extra x ticks={1},
	minor x tick num=1,
	minor y tick num=1,
	minor z tick num=1,
	%extra x tick labels={$a$},
	%extra y ticks={1},
	%extra y tick labels={$a$},
	%extra z ticks={1},
	%extra z tick labels={$h$},
	ymin=-.5,ymax=4.9,
	xmin=-.5,xmax=4.9,
	zmin=-.5, zmax=2.9,
	every axis x label/.style={at={(axis cs:\pgfkeysvalueof{/pgfplots/xmax},0,0)},xshift=-1pt,yshift=-4pt},
	xlabel={\scriptsize $x$},
	every axis y label/.style={at={(axis cs:0,\pgfkeysvalueof{/pgfplots/ymax},0)},xshift=5pt,yshift=-3pt},
	ylabel={\scriptsize $y$},
	every axis z label/.style={at={(axis cs:0,0,\pgfkeysvalueof{/pgfplots/zmax})},xshift=0pt,yshift=4pt},
	zlabel={\scriptsize $z$},
        colormap/cool
      ]
      \addplot3[domain=0:360,,y domain=0:4,
        samples=30,smooth,,samples y=0,dashed,very thick,penColor] ({cos(x)+2},{sin(x)+2},{0});
      
      \addplot3[domain=0:4,,y domain=0:4,mesh,samples=15,samples y=15,very thin,z buffer=sort] {-.2*(x-1)^2-.05*y^2+2};
      
      \addplot3[domain=0:360,,y domain=0:4,
        samples=30,smooth,,samples y=0,very thick,penColor] ({cos(x)+2},{sin(x)+2},{-.2*(cos(x)+2-1)^2-.05*(sin(x)+2)^2+2});
    \end{axis}
  \end{tikzpicture}
\end{image}
Here a surface is drawn, along with a dashed curve in the
$(x,y)$-plane. Restricting $F$ to just the points on this circle gives
the curve shown on the surface. The derivative $\dd[F]{t}$ gives the
instantaneous rate of change of $F$ with respect to $t$.

Now try your hand at the chain rule. 
\begin{question}
  Let $F(x,y)=x^2y+x$, where $x(t)=\sin(t)$ and $y(t)=e^{5t}$. Compute $\dd[F]{t}$.
  \begin{prompt}
  \[
  \dd[F]{t} = \answer{(2\sin(t)e^{5t}+1)\cos(t)+5e^{5t}\sin^2(t)}.
  \]
  \end{prompt}
\end{question}

The previous example can make us wonder: if we substituted for $x$ and
$y$ at the end to show that $\dd[F]{t}$ is really just a function of
$t$, why not substitute \textit{before} differentiating, showing
clearly that $F$ is a function of $t$?

That is, $z = x^2y+x = (\sin t)^2e^{5t}+\sin t.$ Applying the chain
and product rules, we have
\[
\dd[F]{t} = 2\sin(t)\cos(t) e^{5t}+ 5\sin^2(t) e^{5t}+\cos(t),
\]
which matches the result from the example.

This may now make one wonder ``What's the point? If we could already
find the derivative, why learn another way of finding it?'' In some
cases, applying this rule makes differentiation simpler, but this is
hardly the power of the chain rule. Rather, the chain rule is
extremely powerful when \textit{we do not know what $F$, $x$ and/or
  $y$ are}. It may be hard to believe, but often in ``the real world''
we know rate-of-change information (information about derivatives)
without explicitly knowing the underlying functions. The chain rule
allows us to combine several rates of change to find another rate of
change.


\begin{question}
  Suppose the curve below has an arc length parameterization given
  by $\vec{p}(s)$.
  \begin{image}
    \begin{tikzpicture}
      \begin{axis}%
        [
	  xmin=-3,xmax=5,
          ymin=-2,ymax=3,
          xlabel=$x$,ylabel=$y$,
          axis lines=center,
          every axis y label/.style={at=(current axis.above origin),anchor=south},
          every axis x label/.style={at=(current axis.right of origin),anchor=west},
          clip=false,
	  grid =major,
          xtick={-3,-2,...,5},
          ytick={-2,-1,...,3},
	]
        \addplot[line join =bevel,penColor,ultra thick] coordinates{
          (-2,2) (2,-1) (2,2) (4,2)
        };
        \addplot[color=penColor,fill=penColor,only marks,mark=*] coordinates{(-2,2)};  %% closed hole
        \addplot[color=penColor,fill=penColor,only marks,mark=*] coordinates{(4,2)};  %% closed hole
        \node[penColor,below] at (axis cs: 4,2) {$\vec{p}(0)$};
        \node[penColor,above] at (axis cs: -2,2) {$\vec{p}(10)$};
      \end{axis}
    \end{tikzpicture}
  \end{image}
  Let $F(x,y) = x^2 - y^3$. Compute:
  \[
  \eval{\dd{s}F(\vec{p}(s))}_{s=1} 
  \begin{prompt}
    = \answer{-6}
  \end{prompt}
  \]
\end{question}




The chain rule also tells us something about the meaning of the
gradient. As we will see, the gradient vector is always orthogonal to
level curves and surfaces.

\begin{example}
  Suppose that a vector-valued function
  $\vec{c}(t)=\vector{x(t),y(t)}$ runs along a level surface for the
  surface $F(x,y)$. Explain how the chain rule shows that the gradient
  is orthogonal to level curves and surfaces.
  \begin{explanation}
    We should ask ourselves: ``What is the change in $F$ as $t$
    varies?''  Since $\vec{c}(t)$ traces out a level curve, the change
    must be $\answer[given]{0}$. Comparing this to the chain rule we see:
    \begin{align*}
    \dd{t} F(\vec{c}(t)) &= \grad F(\vec{c}(t)) \dotp \vec{c}'(t) \\
    &= \answer[given]{0}
    \end{align*}
    this tells us that the gradient is orthogonal to the tangent
    vectors of our level curve. This means that the gradient is
    orthogonal to level curves.
  \end{explanation}
\end{example}

Note that the last explanation works in any dimension. The up-shot?
\begin{quote}
  \textbf{Gradient vectors are orthogonal to level sets.}
\end{quote}

This is a key concept concerning the gradient.







\section{New solutions for old problems}

We can also use our new chain rule to revisit problems from our
previous studies of calculus. Our new tools allow for simpler
solutions to these problems.

\subsection{Differentiating integrals}

Recall the following form of the Fundamental Theorem of Calculus:
\[
\ddx \int_a^x f(t) \d t = f(x) 
\]
\begin{question}
  Compute
  \[
  \dd x \int_0^x \frac{\sin(t)}{t} \d t
  \begin{prompt}
    = \answer{\frac{\sin(x)}{x}}
  \end{prompt}
  \]
\end{question}
It is easy to use the Fundamental Theorem of Calculus to differentiate
integrals, when the limits of integration are a constant and a
variable. However, when the limits are functions, things get more
complicated. The multivariable chain rule helps out in these
situations.
\begin{example}
  Compute:
  \[
  \dd{t} \int_{\sin(t)}^{\cos(t)} e^{(s^2)} \d s
  \]
  \begin{explanation}
    Let
    \begin{align*}
      F(x,y) &= \int_y^x e^{(s^2)} \d s\\
      x(t) &= \cos(t)\\
      y(t) &= \sin(t).
    \end{align*}
    Now,
    \[
    \dd{t} F(t) = \grad F(x(t),y(t)) \dotp \vector{\answer[given]{-\sin(t)},\answer[given]{\cos(t)}}.
    \]
    To compute the partials, we use the Fundamental Theorem of Calculus:
    \begin{align*}
    \pp[F]{x} &= \pp{x} \int_y^x e^{(s^2)} \d s\\
    &= \answer[given]{e^{(x^2)}}
    \end{align*}
    And
    \begin{align*}
      \pp[F]{y} &= \pp{y} \int_y^x e^{(s^2)} \d s\\
      \pp[F]{y} &= \pp{y} \left(-\int_x^y e^{(s^2)} \d s\right)\\
      &= \answer[given]{-e^{(y^2)}}
    \end{align*}
    So
    \begin{align*}
      \dd{t} F(t) &= \vector{e^{\cos^2(t)},-e^{\sin^2(t)}}\dotp \vector{-\sin(t),\cos(t)}\\
      &= \answer[given]{-e^{\cos^2(t)}\sin(t)-e^{\sin^2(t)}\cos(t)}.
    \end{align*}
  \end{explanation}
\end{example}


\begin{question}
  Compute:
  \[
  \dd{t} \int_{t^2}^{t^3} \frac{\sin(s)}{s} \d s
  \begin{prompt}
    = \answer{\frac{-2 t \sin(t^2)}{t^2}+\frac{3t^2\sin(t^3)}{t^3}}
  \end{prompt}
  \]
\end{question}




\subsection{Implicit differentiation}

We've used implicit differentiation to compute $\dd[y]{x}$ when $y$ is
given as an implicit function of $x$. Now we'll revisit this with the
chain rule and give a new, simpler, method of finding $\dd[y]{x}$.

For instance, consider the implicit function $x^2y-xy^3=3$. We learned
to use the following steps to find $\dd[y]{x}$:
\begin{align*}
\ddx\Big(x^2y-xy^3\big) &= \ddx\Big(3\Big) \\
2xy + x^2\dd[y]{x}-y^3-3xy^2\dd[y]{x} &= 0\\
\dd[y]{x} = -\frac{2xy-y^3}{x^2-3xy^2}.
\end{align*}
     
Instead of using this method, consider $z=x^2y-xy^3$. The implicit
function above describes the level curve $z=3$. Considering $x$ and
$y$ as functions of $x$, the chain rule states that
\[
\dd[z]{x} = \pp[z]{x}\dd[x]{x}+\pp[z]{y}\dd[y]{x}.
\]
Since $z$ is constant (in our example, $z=3$), $\dd[z]{x} = 0$. We
also know $\dd[x]{x} = 1$. Write with me,
\begin{align*}
  0 &= \pp[z]{x}(1) + \pp[z]{y}\dd[y]{x} \\
  \dd[y]{x} &= -\pp[z]{x}\Big/\pp[z]{y}\\
  &= -\frac{F^{(1,0)}(x,y)}{F^{(0,1)}(x,y)}.
\end{align*}

Note how our solution for $\dd[y]{x}$ above is just the partial
derivative of $z$, with respect to $x$, divided by the partial
derivative of $z$ with respect to $y$.  We state the above as a
theorem.

\begin{theorem}
  Let $F:\R^2\to\R$ be a differentiable function of $x$ and $y$, where
  $F(x,y)=c$ defines $y$ as an implicit function of $x$, for some
  constant $c$. Then \index{derivative!implicit}\index{implicit
    differentiation}
  \[
  \dd[y]{x} = -\frac{F^{(1,0)}(x,y)}{F^{(0,1)}(x,y)}.
  \]
\end{theorem}

Try your hand at this.

\begin{question}
  Given the implicitly defined function $\sin(x^2y^2)+y^3=x+y$, find
  $y'$.
  \begin{hint}
    Consider $F(x,y) = \sin(x^2y^2)+y^3-x-y$, and find
    $F^{(1,0)}(x,y)$, and $F^{(0,1)}(x,y)$.
  \end{hint}
  \begin{prompt}
  \[
  \dd[y]{x} = \answer{\frac{-2xy^2\cos(x^2y^2)+1}{2x^2y\cos(x^2y^2)+3y^2-1}}
  \]
  \end{prompt}
\end{question}

%\example{ex_implicit5}{Using Implicit Differentiation}{
%Given the implicitly defined function $\sin(x^2y^2)+y^3=x+y$, find $y'$.}
%$$y' = \frac{1 - 2xy^2\cos(x^2y^2)}{2x^2y\cos(x^2y^2)+3y^2-1}.$$


\end{document}
